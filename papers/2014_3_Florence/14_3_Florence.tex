\documentclass{beamer}
\usepackage[latin1]{inputenc}
\usetheme{Montpellier}
%\usecolortheme[RGB={204,51,255}]{structure}
%\usecolortheme[named=purple]{structure}
\usecolortheme[RGB={62,128,62}]{structure}
%\definecolor{dark}{rgb}{0.3,0.15,0.3}
%\definecolor{light}{rgb}{0.8,0.6,0.8}
%\definecolor{reddish}{rgb}{.5,0.15,0.15}
\definecolor{dark}{rgb}{0.4,0.2,0.4}
%\definecolor{light}{rgb}{0.8,0.6,0.8}
\definecolor{reddish}{rgb}{.7,0.25,0.25}
\usepackage{graphicx}
\usepackage{epsfig}
\usepackage{wrapfig}
\newcommand{\btVFill}{\vskip0pt plus 1filll}

\def\cd{\color{dark}}
\def\cb{\color{black}}

\title[Multi-neuron spike trains - distances and information.]{Multi-neuron spike trains - distances and information.}
\author{Conor Houghton}
\institute{Computer Science, U Bristol}
\date{NETT Florence, March 2014}

\begin{document}

\setlength{\unitlength}{1mm}
\begin{frame}
\titlepage
\end{frame}


\section{Thanks for the invite}


\begin{frame}{}
\color{black} 
In 1817 Stendhal reportedly was overcome by the cultural
richness of Florence he encountered when he first visited the Tuscan
city. As he described in his book Naples and Florence: A Journey from
Milan to Reggio:
\begin{quote}
As I emerged from the porch of Santa Croce, I was seized with a fierce palpitation of the heart (that same symptom which, in Berlin, is referred to as an attack of the nerves); the well-spring of life was dried up within me, and I walked in constant fear of falling to the ground.
\end{quote}
\btVFill
\begin{flushright}
\tiny{Wikipedia article \texttt{http://en.wikipedia.org/wiki/Stendhal}.}
\end{flushright}
\end{frame}



\section{Motivation}


\begin{frame}{Overview.}
\color{black}
\begin{itemize}
\item Mutual information for spaces with distances.
\begin{itemize}
\item{} [Rederive a result due to Kraskov, St\"{o}bauer and Grassberger (PRE 2004)]
\end{itemize}
\item Spike trains and sets of spike trains as a space with distance.
\end{itemize} 
\color{black}
\end{frame}

\begin{frame}{Spike trains.}
Spiking responses in the auditory forebrain of zebra finch.
\color{reddish}
\include{raster}
\color{black}
\end{frame}

\begin{frame}{Classical approach I.}
\begin{itemize}
\item Discretize.
\color{reddish}
\begin{center}
\include{discretization}
\end{center}
\color{black}
\item Split into words.
\color{dark}
$$010001000000100\rightarrow 01000,10000,00100$$
\color{black}
\end{itemize}
\btVFill
\begin{flushright}
\tiny{Bialek, de Ruyer van Steveninck, Strong and other coworkers, late 1990s.}
\end{flushright}
\end{frame}

\begin{frame}{Classical approach II.}
\begin{itemize}
\item Estimate probability of words. For example, say $w_8=01000$ then estimate
\color{dark}
$$p(w_8)\approx\frac{\#\mbox{ occurrences of }w_8}{\#\mbox{ words}}$$ 
\color{black}
\item Calculate
\color{dark}
$$H(W)=-\sum_i p(w_i)\log_2 p(w_i) =-\langle\log_2 p(w_i)\rangle$$ 
\color{black}
\end{itemize}
\btVFill
\begin{flushright}
\tiny{Bialek, de Ruyer van Steveninck, Strong and other coworkers, late 1990s.}
\end{flushright}
\end{frame}


\begin{frame}{Classical approach III.}
\begin{itemize}
\item Conditional probability.
\color{reddish}
\begin{center}
\include{conditioned}
\end{center}
\color{black}
\end{itemize}
\btVFill
\begin{flushright}
\tiny{Bialek, de Ruyer van Steveninck, Strong and other coworkers, late 1990s.}
\end{flushright}
\end{frame}


\begin{frame}{Classical approach IV.}
\begin{itemize}
\item Mutual information
\color{dark}
$$H(W|S)=\langle H(W|s_i)\rangle$$
\color{black}
and
\color{dark}
$$I(W;S)=H(W)-H(W|S)$$
\color{black}
\end{itemize}
\btVFill
\begin{flushright}
\tiny{Bialek, de Ruyer van Steveninck, Strong and other coworkers, late 1990s.}
\end{flushright}
\end{frame}


\begin{frame}{ms scale information in blow fly spike trains.}
\begin{center}
\includegraphics[width=6cm]{blow_fly.jpg}
\end{center}
\btVFill
\begin{flushright}
\tiny{Bialek, de Ruyer van Steveninck, Strong and other coworkers, late 1990s.}
\end{flushright}
\end{frame}



\begin{frame}{Difficulties with the classical approach.}
\begin{itemize}
\item Undersampling. 
\begin{itemize}
\item 100 ms words and 2 ms bins gives $2^{50}=1125899906842624$ words.
\item Lots of clever approaches to this, for example Nemenman et al. (PRE 2004, BMC Neuroscience 2007) where a cunning prior is used for $p(w_i)$.
\end{itemize}
\item Sampling bias.
\begin{itemize}
\item An even distribution will never give equal counts for each word,
  giving different $p(w_i)$.
\item Lots of clever approaches to this too, see Panzeri et al. (J Neurophys. 2007).
\end{itemize}
\end{itemize}
\end{frame}

\begin{frame}{Many fixes but still . . . }
\begin{itemize}
\item Neuron - neuron mutual information.
\item Maze - neuron mutual information.
\item Mutual information with multiple units.
\end{itemize}
\end{frame}



\section{KDE for spike trains.}

\begin{frame}{Approaches to probability estimation}
\begin{itemize}
\item Parameteric approach.
\begin{itemize} 
\item See Gillespie and Houghton (JCN 2011).
\item . . . or Yu et al., (Front. in Comp. Neuro. 2010).
\end{itemize}
\item Non-parametric approach
\begin{itemize}
\item Histograms.
\item Kernel density estimation (KDE).
\item $k$th nearest neighbor (kNN).
\end{itemize}
\end{itemize}
We will see that KDE and kNN lead to more-or-less the same formula.
\end{frame}


\begin{frame}{Kernel density estimation.}
\begin{center}
\includegraphics[width=8cm]{Comparison_of_1D_histogram_and_KDE.png}
\end{center}
\color{dark}
$$
p(x)=\frac{1}{n}\sum_ik(x-x_i)
$$
\color{black}
\btVFill
\begin{flushright}
\tiny{Picture from \texttt{http://en.wikipedia.org/wiki/Kernel\_density\_estimation}}
\end{flushright}
\end{frame}

\begin{frame}{Histograms and the classical approach}
\begin{itemize}
\item Histogram:
\color{dark} 
\begin{eqnarray*}
\mbox{binning}: \mathbf{R}&\rightarrow&\mathbf{Z}\cr
                         x&\mapsto& \mbox{int}\,(x/\delta x)\mbox{th bin} 
\end{eqnarray*}
\color{black}
\item Converting spike trains to words:
\color{dark}
\begin{eqnarray*}
\mbox{discretization}: \mbox{space of spike trains}&\rightarrow&\mathbf{Z}\cr
                         r&\mapsto& w
\end{eqnarray*}
\color{black}
\end{itemize}
\end{frame}

\begin{frame}{What we need for kernel density estimation.}
\color{dark}
$$
p(x)=\frac{1}{n}\sum_ik(x-x_i)
$$
\color{black}
So we must have
\color{dark}
$$
\int k(x)dx=1
$$
\color{black}
That means we need to be able to integrate.
\end{frame}


\begin{frame}{Spike train space.}
\color{reddish}
\begin{center}
\include{spiketrains}
\end{center}
\color{black}
\end{frame}


\begin{frame}{Integrating in the space of spike trains I.}
The space of spike trains has no coordinates, but it does have a
measure given by the distribution of spike trains.
\color{dark}
$$
\mbox{vol}(\mathcal{D})=P(r\in \mathcal{D})
$$
\color{black}
which can be estimated by the fraction of responses in \color{dark}$\mathcal{D}$
$$
\mbox{vol}(\mathcal{D})\approx\frac{\#\mbox{ spike trains in }\mathcal{D}}{\#\mbox{ spike trains}} 
$$
\color{black}
\end{frame}
 

\begin{frame}{Integrating in the space of spike trains II.}
\color{reddish}
\begin{center}
\setlength{\unitlength}{2500sp}%
%
\begingroup\makeatletter\ifx\SetFigFont\undefined%
\gdef\SetFigFont#1#2#3#4#5{%
  \reset@font\fontsize{#1}{#2pt}%
  \fontfamily{#3}\fontseries{#4}\fontshape{#5}%
  \selectfont}%
\fi\endgroup%
\begin{picture}(3466,3016)(2130,-2956)
{\color[rgb]{0,0,0}\thinlines
\put(4201,-2061){\circle*{226}}
}%
{\color[rgb]{0,0,0}\put(3701,-1661){\circle*{226}}
}%
{\color[rgb]{0,0,0}\put(4501,-1511){\circle*{226}}}%
{\color[rgb]{0,0,0}\put(3501,-1211){\circle*{226}}}%
{\color[rgb]{0,0,0}\put(4651,-2011){\circle*{226}}
}%
{\color[rgb]{0,0,0}\put(2251,-961){\circle*{226}}}%
{\color[rgb]{0,0,0}\put(2051,-2061){\circle*{226}}}%
{\color[rgb]{0,0,0}\put(3551,-2161){\circle*{226}}}%
{\color[rgb]{0,0,0}\put(3251,-201){\circle*{226}}}%
{\color[rgb]{0,0,0}\put(3051,-1601){\circle*{226}}}%
{\color[rgb]{0,0,0}\put(2451,-1661){\circle*{226}}
}%
{\color[rgb]{0,0,0}\put(3151,-961){\circle*{226}}
}%
{\color[rgb]{0,0,0}\put(2251,-61){\circle*{226}}}%
{\color[rgb]{0,0,0}\put(5051,-21){\circle*{226}}}%
{\color[rgb]{0,0,0}\put(4251,-611){\circle*{226}}}%
{\color[rgb]{0,0,0}\put(5251,-1611){\circle*{226}}}%
{\color[rgb]{0,0,0}\put(5351,-2011){\circle*{226}}}%
{\color[rgb]{0,0,0}\put(4501,-1861){\circle{2000}}}%
\put(4501,-1261){volume$\approx 3/17$}
\end{picture}%

\end{center}
\color{black}
\end{frame}
 
\begin{frame}{Basic idea}
Want to calculate:
\color{dark}
$$
 I(R;S) =  \sum_{s\in\mathcal{S}}\int_\mathcal{R} p(r,s) \log_2 \frac{p(r|s)}{p(r)} dr  
$$ \color{black}
\begin{itemize}
\item Use \color{dark}$p(r)$\color{black} to give a measure.
\item Use KDE to estimate \color{dark}$p(r|s)$\color{black}. 
\item Any integrals will be estimated by counting points. 
\item Kernels will be defined using volume.
\end{itemize}
\btVFill
\begin{flushright}
\tiny{Tobin and Houghton (Entropy 2013)}
\end{flushright}
\end{frame}


\begin{frame}{Result}
After some fiddling this give
\color{dark}
$$
 I(R;S) \approx \frac{1}{n_r}\sum_{i} \log_2{\frac{n_sc(r_i,s_i;n_h)}{n_h}}
$$
\color{black}
\begin{itemize}
\item \color{dark}$n_r$\color{black}{} number of responses.
\item \color{dark}$n_s$\color{black}{} number of stimuli.
\item \color{dark}$n_h$\color{black}{} size of the kernel.
\item \color{dark}$c(r_i,s_i;n_h)$\color{black}{} number of responses to stimulus \color{dark}$s_i$\color{black}{} in the kernel around \color{dark}$r_i$\color{black}.
\end{itemize}
\btVFill
\begin{flushright}
\tiny{Tobin and Houghton (Entropy 2013)}
\end{flushright}
\end{frame}

\begin{frame}{$c(r_i,s_i;n_h)$}
\color{reddish}
\begin{center}
\include{c_fig}
\end{center}
\color{black}
\end{frame}

\begin{frame}{Results I}
We are interested in the structure of spike train space as a metric
space, so simulated test data was constructed using the sort of
distribution of spike trains in metric space observed in Gillespie and
Houghton (JCN 2011).
\end{frame}

\begin{frame}{Results II}
\color{reddish}
\begin{center}
\include{plot_run1}
\color{black}
\color{dark} $n_s=10$\color{black}{} and \color{dark}{} $n_t=10$\color{black}.
\end{center}
\color{black}
\btVFill
\begin{flushright}
\tiny{Tobin and Houghton (Entropy 2013)}
\end{flushright}
\end{frame}


\begin{frame}{Results III}
\color{reddish}
\begin{center}
%\documentclass[12pt,a4paper]{article} % for use with pdfLaTeX only

%\usepackage{graphicx}
%\usepackage{amsfonts}
%\usepackage{amssymb}
%\usepackage{amsmath}
%\usepackage{amsthm}

%\begin{document}

% GNUPLOT: LaTeX picture with Postscript
\vskip -1cm
\begin{tabular}{ll}
\textbf{histogram}&\textbf{KDE}\\[0.5cm]

  \setlength{\unitlength}{0.0350bp}%
  \begin{picture}(4040.00,3228.00)%
      \put(949,937){\makebox(0,0)[r]{\strut{} 0}}%
      \put(949,1402){\makebox(0,0)[r]{\strut{} 0.2}}%
      \put(949,1867){\makebox(0,0)[r]{\strut{} 0.4}}%
      \put(949,2332){\makebox(0,0)[r]{\strut{} 0.6}}%
      \put(949,2798){\makebox(0,0)[r]{\strut{} 0.8}}%
      \put(949,3263){\makebox(0,0)[r]{\strut{} 1}}%
      \put(1314,484){\makebox(0,0){\strut{} 0}}%
      \put(2244,484){\makebox(0,0){\strut{} 0.4}}%
      \put(3175,484){\makebox(0,0){\strut{} 0.8}}%
      \put(179,1983){\rotatebox{-270}{\makebox(0,0){\strut{}estimated (ban)}}}%
      \put(2360,154){\makebox(0,0){\strut{}true (ban)}}
    \put(-500,0){\includegraphics[width=6.35cm]{plot_run1_old}}%
  \end{picture}%
&
  \setlength{\unitlength}{0.0350bp}%
  \begin{picture}(4040.00,3228.00)%
      \put(949,937){\makebox(0,0)[r]{\strut{} 0}}%
      \put(949,1402){\makebox(0,0)[r]{\strut{} 0.2}}%
      \put(949,1867){\makebox(0,0)[r]{\strut{} 0.4}}%
      \put(949,2332){\makebox(0,0)[r]{\strut{} 0.6}}%
      \put(949,2798){\makebox(0,0)[r]{\strut{} 0.8}}%
      \put(949,3263){\makebox(0,0)[r]{\strut{} 1}}%
      \put(1314,484){\makebox(0,0){\strut{} 0}}%
      \put(1779,484){\makebox(0,0){\strut{} 0.2}}%
      \put(2244,484){\makebox(0,0){\strut{} 0.4}}%
      \put(2709,484){\makebox(0,0){\strut{} 0.6}}%
      \put(3175,484){\makebox(0,0){\strut{} 0.8}}%
      \put(3640,484){\makebox(0,0){\strut{} 1}}%
      \put(179,1983){\rotatebox{-270}{\makebox(0,0){\strut{}estimated (ban)}}}%
      \put(2360,154){\makebox(0,0){\strut{}true (ban)}}%
    \put(-500,0){\includegraphics[width=6.35cm]{plot_run1_new}}%
  \end{picture}%
 \end{tabular} 
%\end{document}

\color{black}
\color{dark} $n_s=10$\color{black}{} and \color{dark}{} $n_t=200$\color{black}.
\end{center}
\color{black}
\btVFill
\begin{flushright}
\tiny{Tobin and Houghton (Entropy 2013)}
\end{flushright}
\end{frame}

\begin{frame}{KDE and kNN}
\color{black}
\begin{itemize}
\item KDE
\color{dark}
$$
 I(R;S) \approx \frac{1}{n_r}\sum_{i} \log_2{\frac{n_sc(r_i,s_i;n_h)}{n_h}}
$$
\color{black}
\item kNN
\color{dark}
$$
   I_e(R;S) \approx  \digamma(n_k) + \digamma(n_tn_s)-\digamma(n_t) - \displaystyle \frac{1}{n_r} \sum_{i} \digamma[C(r_i,s_i;n_k)] 
$$
\color{black}
\begin{itemize}
\item They use a Kozachenko and Leonenko estimator and, by a clever choice of how they pick $k$ for different parts of the estimate, they get all the volume-based terms to cancel.
\end{itemize} 
\end{itemize}
\btVFill
\begin{flushright}
\tiny{Kraskov, St\"{o}bauer and Grassberger (PRE 2004)}
\end{flushright}
\end{frame}

\section{Distances}

\begin{frame}{Euclidean metric}
\begin{center}
\includegraphics[width=4.5cm]{U1.png}
\end{center}
\btVFill
\begin{flushright}
\tiny{Picture from Google maps}
\end{flushright}
\end{frame}


\begin{frame}{Non-Euclidean metric}
\begin{center}
\includegraphics[width=4.5cm]{U2.png}
\end{center}
\btVFill
\begin{flushright}
\tiny{Picture from Google maps}
\end{flushright}
\end{frame}


\begin{frame}{Non-metric}
\begin{center}
\includegraphics[width=8cm]{800px-Uffizi_Hallway.jpg}
\end{center}
\btVFill
\begin{flushright}
\tiny{Picture from \texttt{http://en.wikipedia.org/wiki/Uffizi}}
\end{flushright}
\end{frame}


\begin{frame}{van Rossum metric}
\begin{center}
\includegraphics[width=9cm]{filtered.png}
\end{center}
Spike trains mapped to functions and a metric on the space of functions induces a metric on the spike train space.
\btVFill
\begin{flushright}
\tiny{van Rossum (Neural Comp. 2001)}
\end{flushright}
\end{frame}


\begin{frame}{Multi-unit van Rossum metric}
\begin{itemize}
\item There is a multi-unit easily computed version of the van Rossum metric.
\item It relies on a time constant and a population parameter.
\end{itemize}
\btVFill
\begin{flushright}
\tiny{Houghton and Sen (Neural Comp. 2008) / Houghton and Kreuz (Network 2012)}
\end{flushright}
\end{frame}

\begin{frame}{The Victor-Purpura metric I.}
\color{black}Edit one spike train in to the other:
\begin{enumerate}
\item \color{black}Insertion of a spike with a cost of one.
\item \color{black}Deletion of a spike with a cost of one.
\item \color{dark}\textcolor{black}{Moving a spike a distance} $\delta t$ \textcolor{black}{costs} $q|\delta t|$.
\end{enumerate}
\color{black}The distance is the cost of the cheapest edit.
\btVFill
\begin{flushright}
\tiny{Victor and Purpura (J. Neurophys. 1996)}
\end{flushright}
\end{frame}


\begin{frame}{The Victor-Purpura metric II.}
\color{reddish}
\include{VP-diagram}
\color{dark}
\[
d=3+q(\delta t_1+\delta t_2)
\]
\end{frame}

\begin{frame}{Multi-neuron Victor-Purpura I.}
\begin{itemize}
\item \color{black} The Victor-Purpura metric can be extended to measure a
distance between a pair of population responses by adding a cost
\color{dark}$k$\color{black}{} for changing the identity of a spike.
\item \color{black}Edit one spike train in to the other:
\begin{enumerate}
\item \color{black}Insertion of a spike with a cost of one.
\item \color{black}Deletion of a spike with a cost of one.
\item \color{black}Change the neuron label of a spike at a cost \textcolor{dark}{$k$}.
\item \color{dark}\textcolor{black}{Moving a spike a distance} $\delta t$ \textcolor{black}{costs} $q|\delta t|$.
\end{enumerate}
\color{black}The distance is the cost of the cheapest edit.
\end{itemize}
\btVFill
\begin{flushright}
\tiny{Aranov et al. (J. Neurophys. 2004)}
\end{flushright}
\end{frame}


\begin{frame}{Multi-neuron Victor-Purpura II.}
\color{reddish}
\include{VP-multiunit}
\color{dark}
\[
d=3+q(\delta t_1+\delta t_2)+k
\]
\end{frame}



\begin{frame}{Labelled-line and population codes}
\begin{itemize}
\item If \color{dark}$k=0$\color{black}{} it does not matter what neuron a spike came from, this is \textsl{a population code}.
\begin{itemize}
\item It's like superimposing all the spike trains in the population and then working out the distance.
\end{itemize}
\item If \color{dark}$k=2$\color{black}{} it is never worth changing the label, this is \textsl{a labelled-line code}.
\begin{itemize}
\item The distance between the two population responses is just the sum of the distances for the individual pairs of responses for each neuron.
\end{itemize}
\end{itemize}
\btVFill
\begin{flushright}
\tiny{Aranov et al. (J. Neurophys. 2004)}
\end{flushright}
\end{frame}

\begin{frame}{Problems}
\begin{itemize}
\item Multi-unit VP metric computationally expensive.
\item How do we calculate \color{dark}$q$\color{black}{} and \color{dark}$k$\color{black}{}?
\end{itemize}
\end{frame}


\begin{frame}{SPIKE metric.}
\color{reddish}
\include{SPIKE-diagram}
\color{dark}
$$
d=(\delta t_1+\delta t_2+\delta t_3)+(\delta t_1+\delta t_3)=2\delta t_1+\delta t_2+2\delta t_3
$$
\btVFill
\begin{flushright}
\tiny{Kreuz, Chicharro, Houghton, Andrzeja and Mormann (J. Neurophys. 2013)}
\end{flushright}
\end{frame}


\begin{frame}{Multi-neuron SPIKE metric.}
Extending SPIKE to multiple neurons is easy, just add a \lq{}distance\rq{} between neurons.
\color{reddish}
\include{Multi-SPIKE-diagram}
\color{dark}
$$
d==2\delta t_1+2\delta t_2+\delta t_3+k
$$
\end{frame}

\begin{frame}{Multi-neuron SPIKE metric - problem.}
Still stuck with \color{dark}$k$\color{black}.
\end{frame}

\section{The end}

\begin{frame}{Conclusions}
\begin{itemize}
\item Distance-based measures of mutual information.
\item Multi-neuron distance functions.
\end{itemize}
\end{frame}

\begin{frame}{Acknowledgements}
\begin{itemize}
\item Collaborators.
\begin{itemize}
\item Thomas Kreuz / Josh Tobin / James Gillespie / Cathal Cooney.
\end{itemize}
\item Funding: the James S McDonnell Foundation.
\end{itemize}
\end{frame}


\end{document}

