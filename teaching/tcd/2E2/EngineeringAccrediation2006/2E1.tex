%SourceDoc ../../../main.tex
\subsection{Engineering Mathematics III (2E1)}
\label{Course2E1}
\paragraph{Lecturer}
Dmitri Zaitsev (CV on page \pageref{Dmitri_Zaitsev}).
\paragraph{Course Organisation:}

The course runs for all three terms. It comprises two lectures and one tutorial per week.

\begin{center}
\begin{tabular}{|c|c|c|c|}
\hline
\multicolumn{2}{|c|}{Lectures}&\multicolumn{2}{|c|}{Tutorials}\\
\cline{1-4}
Per Week&Total&Per Week&Total\\
\hline
2&48&1&24\\
\hline
\multicolumn{4}{|c|}{Total Contact Hours: 72}\\
\hline
\end{tabular}
\end{center}

\paragraph{Course Description, Aims and Contribution to Programme:}

This one year long course is taken by Senior Freshman students.
It is a natural continuation of the Junior Freshman courses 1E1
and 1E2 and introduces a student into further fundamental ideas and methods
of mathematics for Engineering, covering the areas of Multivariate
Calculus, Integration, Linear Algebra and Fourier Analysis.
The aim of the course is to provide the necessary background
and to teach the students to use it efficiently in applications.


\paragraph{Learning Outcomes:}

\begin{itemize}
\item The student will be able to analyse the behaviour functions of several variables, present the result graphically and efficiently calculate partial derivatives of functions of several variables, also for functions given implicitly.
\item The student will be able to obtain equations for tangent lines to plane curves and tangent planes to space surfaces.
\item The student will be able to apply derivative tests and the method of Lagrange multipliers to find maxima and minima of functions of several variables, local and global.
\item The student will be able to obtain linear and higher order approximations and to estimate the related errors.
\item The student will be able to effectively calculate multiple integrals, in cartesian and polar coordinates, in particular, to find areas, volumes and centers of mass.
\item The student will be able to calculate eigenvalues and eigenvectors of matrices.
\item The student will be able to apply the Gram-Schmidt Process and the Least Square Method.
\item The student will be able to analyse the behaviour of the Fourier series and Fourier transformation.
\end{itemize}

\paragraph{Content of Course:}

\begin{itemize}
\item Review of Derivatives and Integrals in one variable
\item Functions of several variables, their graphs, level curves and level surfaces
\item Limits and continuity
\item Partial and Directional Derivatives, gradients
\item Standard Linear Approximation
\item Chain Rule
\item Standard Linear Approximation
\item Maxima, Minima and Saddle points, first and second derivative tests
\item Lagrange multiplies
\item Taylor's formula
\item Double and Multiple Integrals
\item Areas, Volumes and Center of Mass
\item Integrals in Polar, Cylindrical and Spherical Coordinates
\item Change of Variable in Multiple integrals
\item Review of Vectors and Matrices, Eigenvalues and Eigenvectors
\item Linear Dependence of vectors and Basis
\item Dimension of a Space, Rank of a Matix
\item Gram-Schmidt Process, Least Square Method, Orthogonal Matrices
\item Periodic Functions, Trigonometric Series, Fourier Series
\item Fourier Transform
\end{itemize}

\paragraph{Contribution to Programme:}

The course completes building the knowledge fundament started in 1E1.
Students will grasp the most fundamental methods of Multivariate Calculus, Integration, Linear Algebra
and Fourier Analysis and learn to apply them creatively in new situations.

The course's contribution to the IEI Programme Areas and outcomes are characterised in the following tables (H=high; M=moderate; L=low):

{\footnotesize\textsf{
\begin{center}
\begin{tabular}{|p{.9in}|p{.9in}|p{.9in}|p{.9in}|p{.9in}|p{.9in}|}
\hline
Science and Mathematics&Discipline Specific Technology&Information and Communications Technology&Design and Development&Engineering Practice&Social and Business Context\\
\hline
H&H&L&L&L&-\\
\hline
\end{tabular}
\end{center}
}}
\begin{center}
\textsf{Contribution to IEI Programme Areas:}
\end{center}

The course contributes to the IEI Programme Outcomes as follows:

{\footnotesize\textsf{
\begin{center}
\begin{tabular}{|p{.9in}|p{.9in}|p{.9in}|p{.9in}|p{.9in}|p{.9in}|}
\hline
(a) The ability to derive and apply solutions from a knowledge of sciences, engineering sciences, technology and mathematics&
(b) The ability to identify, formulate, analyse and solve engineering problems&
(c) The ability to design a system, component or process to meet specified needs&
(d) An understanding of the need for high ethical standards in the practice of
engineering, \ldots &
(e) The ability to work effectively as an individual, in teams and in multidisciplinary
settings \ldots &
(f) The ability to communicate effectively with the engineering community
and with society at large\\
\hline
H&H&H&-&L&L\\
\hline
\end{tabular}
\end{center}
}}
\begin{center}
\textsf{Contribution to IEI Programme Outcomes}
\end{center}


\paragraph{Teaching Strategies:}
The teaching strategy is a mixture of lectures, independent and team work in doing homework and tutorials.
The lectures present the material in traditional form, including motivation, theory and applications.
The most critical phenomena and typical mistakes are emphasized.
The exercises are assigned weekly and aimed to stimulate students to
actively use and revise the learned material.
As an important byproduct, the students learn how to 
express their way of solving problems clearly in written form.
This process is controlled by grading the student solutions
and discussing them in the tutorials.

\paragraph{Assessment:}
Assessment is by weekly assignment and final exam.
The assignments are marked and contribute towards the final grade.
The 3 hour exam is on the material explained in the lectures
and trained in the homework.
The course mark is the maximum of 100\% of the exam or
90\% of the exam and 10\% of the assignments.

\paragraph{Recommended Texts:}

\emph{Calculus}, Thomas \& Finney (10th ed.), Chapters 11-12.
\emph{Elementary Linear Algebra (with applications)}, Anton \& Rorres, Chapters 4-11.
\emph{Advanced Engineering Mathematics}, Kreyszig, Chapter 10.

\paragraph{Further Information}~\\
Exam paper: page \pageref{2E1}.\\
ECTS Credits: 8.
