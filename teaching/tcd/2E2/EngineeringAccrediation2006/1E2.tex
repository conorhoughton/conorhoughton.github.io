%SourceDoc ../../../main.tex
\subsection{Engineering Mathematics II (1E2)}
\label{Course1E2}
\paragraph{Lecturer} Donal O'Donovan (CV on page \pageref{Donal_ODonovan}).

\paragraph{Course Organisation}
The course runs for all three terms.  It consists of two lectures in a class of 225, and one tutorial, in a class of 40, per week.

\begin{center}
\begin{tabular}{|c|c|c|c|}
\hline
\multicolumn{2}{|c|}{Lectures}&\multicolumn{2}{c|}{Tutorials}\\
\hline
Per Week&Total&Per Week&Total\\
\hline
2&48&1&24\\
\hline
\multicolumn{4}{|c|}{Total Contact Hours: 72}\\
\hline
\end{tabular}
\end{center}

\paragraph{Course Description, Aims and Contribution to Programme}
This is a year long course designed to introduce the student to some of the important mathematical ideas for Engineering, mainly in discrete or finite mathematics.  It covers an introduction to Matrices and to Eigenvalues and Eigenvectors with applications.

It also gives an introduction to Logic and Set Theory.  It begins the students study of Probability.  Finally, it begins the study of Differential Equations and having dealt with Linear Differential Equations with constant coefficients it teaches them the Theory of Linear Difference Equations with constant coefficients.

The aim of the course is that students learn to apply mathematical techniques and that they understand the need to analyse these techniques both for their power and their limitations.  This necessitates moving their thinking beyond what was required of them at school, into the region where they can adapt their knowledge to new situations.
\paragraph{Learning Outcomes}
\begin{itemize}
\item The student will be able to calculate solutions to systems of Linear Equations in different ways and to understand why some methods are more efficient than others.
\item The student will be able to examine the eigenvalues of a Linear model and determine its stability or otherwise.
\item The student will understand the underlying principals behind the least squares method of finding the line of best fit.
\item The student will understand how to analyse and synthesise logic circuits.
\item The student will be capable of determining simple probabilities and of deciding which distribution to use in particular cases.
\item The student will learn to solve simple differential equations and to apply similar techniques to difference equations.
\end{itemize}
\paragraph{Content of Course}
\begin{itemize}
\item Systems of Equations, Gaussian Elimination. 
\item Matrices, Invertibility, Linear independence, LU decomposition. 
\item Determinants. Adjoint Matrix and Cramer's Rule.
\item Eigenvalues, Eigenvectors and application to Predator/Prey models, Markov Processes and Linear systems of differential equations.
\item Vectors in R2 and R3. 
\item Perpendicular projections and the Least Squares Method.
\item Propositional Calculus.
\item Logic Circuits, Analysis and Synthesis.
\item Predicate Calculus, Quantifiers.
\item Set Theory.
\item Probability Theory, Bayes' Theorem.
\item Binomial, Poisson, Normal Distributions.
\item First Order Differential Equations, Separable, Exact, Linear.
\item Exponential Models.
\item Complex Numbers.
\item Linear Differential Equations with constant coefficients � using the D operator.
\item Linear Difference Equations with constant coefficients � using the E operator.
\end{itemize}
\paragraph{Contribution to Programme}
The all-important first steps are taken in Linear Algebra, Discrete Mathematics and Differential and Difference Equations.  A start is made in replacing the mainly passive learning of school with the active learning of College.
{\footnotesize\textsf{
\begin{center}
\begin{tabular}{|p{.9in}|p{.9in}|p{.9in}|p{.9in}|p{.9in}|p{.9in}|}
\hline
Science and Mathematics&Discipline Specific Technology&Information and Communications Technology&Design and Development&Engineering Practice&Social and Business Context\\
\hline
H&H&L&L&L&--\\
\hline
\end{tabular}
\end{center}
}}
\begin{center}
\textsf{Contribution to IEI Programme Areas}
\end{center}

The course contributes to the IEI Programme Outcomes as follows:

{\footnotesize\textsf{
\begin{center}
\begin{tabular}{|p{.9in}|p{.9in}|p{.9in}|p{.9in}|p{.9in}|p{.9in}|}
\hline
(a) The ability to derive and apply solutions from a knowledge of sciences, engineering sciences, technology and mathematics&
(b) The ability to identify, formulate, analyse and solve engineering problems&
(c) The ability to design a system, component or process to meet specified needs&
(d) An understanding of the need for high ethical standards in the practice of
engineering, \ldots &
(e) The ability to work effectively as an individual, in teams and in multidisciplinary
settings \ldots &
(f) The ability to communicate effectively with the engineering community
and with society at large\\
\hline
H&H&H&--&L&L\\
\hline
\end{tabular}
\end{center}
}}
\begin{center}
\textsf{Contribution to IEI Programme Outcomes}
\end{center}


\paragraph{Teaching Strategies}
The main goal of lectures and tutorials is to engage the students in thinking about the material. So they are continually asked to reflect and comment on what they have done.  The conclusion might be the statement of a theorem or a conjecture.  The aim is to shift them away from the rote learning that many of them were used to in school.  If the first year courses manage to effect this, the outcome is much superior to simply adding more material to courses, so assignments are meant to solidly what has been taught but also to lead into new material.  This part they find new and difficult, but it is really beneficial.
\paragraph{Assessment}
The students complete 21 assignments.  These are all corrected but only 10 are marked.  They sit $1\frac{1}{2}$ hour exams at the end of the first and second terms.  These exams are purely on the material for that term.  Then they sit a 3 hour exam, on all the material for the year, in June.  The course mark is the maximum of 100\% June exam or 50\% June exam, 20\% each for the other two exams and 10\% assignments.  The questions are made up of ``calculate'' parts and ``explain'' parts to emphasise the necessity to understand and be able to analyse what they do.  Copies of past exams are given to the students on day one of the course to stress this emphasis.
\paragraph{Recommended Texts}
\begin{itemize}
\item[]\emph{Elementary Linear Algebra} --- Applications Version, Anton Rorres, Ninth Edition, Wiley
\item[]Any text on Discrete Mathematics
\item[] \emph{Thomas Calculus}, published by Addison Wesley
\end{itemize}

\paragraph{Further Information}~\\
Exam paper: page \pageref{1E2}.\\
ECTS Credits: 8.


