%SourceDoc ../../../main.tex
\subsection{Engineering Mathematics V (3E1)}
\label{Course3E1}
\paragraph{Lecturer}
John Stalker (CV on page \pageref{John_Stalker}).
\paragraph{Course Organization}
The course runs the full academic year with two lectures and one tutorial
per week.  The total number of contact hours is seventy two.
\begin{center}
\begin{tabular}{|c|c|c|c|c|c|}
\hline
Engineering
&
Start
&
\multicolumn{2}{|c|}{Lectures}
&
\multicolumn{2}{|c|}{Tutorials}
\cr
\cline{3-6}
Semester or Term
&
Week
&
Per Week
&
Total
&
Per Week
&
Total
\cr
\hline
1, 2 and 3
&
1 (each term)
&
2
&
48
&
1
&
24
\cr
\hline
\multicolumn{6}{|c|}{Total Contact Hours: 72}
\cr
\hline
\end{tabular}
\end{center}
\paragraph{Course Description, Aims, and Contribution to Programme}
\textit{Engineering Mathematics V} is a full year course for all
engineering streams.  It continues and extends the material from
1E1, 1E2, 2E1 and 2E2.  The emphasis is primarily on
analytical techniques, the corresponding numerical methods being
taught in 3E2.
\paragraph{Learning Outcomes}
\begin{itemize}
\item The student will be able to calculate Fourier transforms,
discrete or continuous, for a variety of simple functions.
Students will then be able to use these to compute convolutions
or deconvolutions in simple cases.
\item The student will be able to solve the Laplace, Heat and
Wave Equations for a variety of boundary conditions in domains
of simple geometry and
with simple boundary conditions.  For the Wave and Heat Equations
the techniques available will include convolution with the
fundamental solution and separation of variables, including
Fourier expansion.  For the Laplace Equation the techniques will
include those above, and also basic methods from complex variables.
\item The student will be able to solve unconstrained quadratic
optimization problems or problems with linear objective and
linear constraints.  For more complicated constrained optimization
problems the student will be able to formulate the first order
minimization/maximization conditions (Kuhn-Tucker).
\item The student will be able to state the various equivalent
definitions of analyticity--power series expansion, the Cauchy-Riemann
equations, Liouville's and Morera's theorems--and use them verify that
a function is analytic.  The student be able to compute sums, products,
quotients, compositions, derivatives and integrals of analytic functions
given by power series.  The student will be able to explain the relation
between harmonic and analytic functions as use this to solve the Laplace
equation in simple cases.
\end{itemize}
\paragraph{Content of Course}
\begin{itemize}
  \item Review of Fourier Methods
  \begin{itemize}
    \item Sampling, Aliasing, etc.
    \item Definition of Fourier Series, Transform, etc.
    \item Fast Fourier Transform
    \item Gibbs Phenomenon
    \item Regularity and Decay
    \item Filtering and Other Applications
  \end{itemize}
  \item Partial Differential Equations
  \begin{itemize}
    \item Laplace's Equation
    \item The Heat Equation
    \item The Wave Equation
    \item Fundamental Solutions
    \item Separation of Variables
    \item Finite Differences/Finite Elements
  \end{itemize}
  \item Optimization
  \begin{itemize}
    \item Linear Programming
    \item Kuhn-Tucker
    \item Duality
    \item Graph Theory
  \end{itemize}
  \item Complex Analysis
  \begin{itemize}
    \item Power Series
    \item The Cauchy-Riemann Equations
    \item Familiar Functions Extended to Complex Domain
    \item Complex Integrals
  \end{itemize}
\end{itemize}
\paragraph{Contribution to Programme}
This course and 3E2 mark the end of most students' mathematical
preparation.  By the end they should have available the basic
mathematical ``tool kit'' of a working engineer and be equipped
to learn more advanced, discipline-specific techniques independently.

The course's contribution to the IEI Programme Areas is summarized
in the following table.
\begin{center}
\scriptsize
\begin{tabular}{|c|c|c|c|c|c|}
\hline
\parbox[t]{0.135\textwidth}{\vspace{1mm}Science and Mathematics}
&
\parbox[t]{0.135\textwidth}{\vspace{1mm}Discipline Specific Technology}
&
\parbox[t]{0.135\textwidth}{\vspace{1mm}
Information and Communication Technology\\
}
&
\parbox[t]{0.135\textwidth}{\vspace{1mm}Design and Development}
&
\parbox[t]{0.135\textwidth}{\vspace{1mm}Engineering Practice}
&
\parbox[t]{0.135\textwidth}{\vspace{1mm}Social and Business Context}
\cr
\hline
High
&
Low
&
Moderate
&
Low
&
Low
&
Low
\cr
\hline
\end{tabular}
\\
\normalsize
\vspace{2mm}
Contribution to IEI Programme Areas
\end{center}

\vspace{0.5cm}
The course's contribution to the IEI Programme Outcomes is summarized
in the following table.
\begin{center}
\tiny
\begin{tabular}{|c|c|c|c|c|c|}
\hline
\parbox[t]{0.145\textwidth}{\sloppy \vspace{1mm}
(a) The ability to derive and apply solutions from a knowledge of the
sciences, engineering sciences, technology and mathematics\\
}
&
\parbox[t]{0.145\textwidth}{\sloppy \vspace{1mm}
(b) The ability to formulate, analyse and solve engineering problems
}
&
\parbox[t]{0.145\textwidth}{\sloppy \vspace{1mm}
(c) The ability to design a system component or process to meet specified
needs
}
&
\parbox[t]{0.145\textwidth}{\sloppy \vspace{1mm}
(d) An understanding of the need for high ethical standards in the practice
of engineering
}
&
\parbox[t]{0.145\textwidth}{\sloppy \vspace{1mm}
(e) The ability to work effectively as an individual, in teams and in
multidisciplinary settings
}
&
\parbox[t]{0.145\textwidth}{\sloppy \vspace{1mm}
(f) The ability to communicate effectively with the engineering community
and society at large
}
\cr
\hline
High
&
Moderate
&
Low
&
Low
&
Moderate
&
Low
\cr
\hline
\end{tabular}
\\
\vspace{2mm}
\normalsize
Contribution to IEI Programme Outcomes
\end{center}
\paragraph{Teaching Strategies}
The format of the lectures is conventional.  Lecture time is divided
about equally between general theory and specific examples.  The latter
are chosen to be similar to, but slightly harder than, those which appear
in tutorial and on the examination.  Where appropriate, computer
demonstrations are used to illustrate phenomena, e.g Gibbs' overshoot,
which are easier to absorb visually.
The tutorials, held in groups of
much smaller size than the lectures, are devoted entirely
to problem solving.
\vspace{0.5cm}
\\
\textbf{Assessment}
Assessment is primarily by the three hour
written examination at the end of the course.
A small portion of the grade, typically 10-15\%
is determined by the students' solutions
to the tutorial problems.
\vspace{0.5cm}
\\
\textbf{Recommended Texts}
Any of several books with titles like ``Advanced Engineering
Mathematics.''  Chapters 10-14 of Kreyszig cover most of the course
material, though in a slightly different way.
\vspace{0.5cm}
\\
\textbf{Further Information}~\\
The web site \myhref{http://www.maths.tcd.ie/~stalker/3e1/} contains
course materials from the current year's course.\\
Exam paper: page \pageref{3E1}.\\
ECTS Credits: 10.
