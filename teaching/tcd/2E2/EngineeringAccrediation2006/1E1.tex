%SourceDoc ../../../main.tex
\subsection{Engineering Mathematics I (1E1)}
\label{Course1E1}
\paragraph{Lecturer} Dr.~Martin Kurth (CV on page \pageref{Martin_Kurth}).

\paragraph{Course Organisation} The course runs for all 24 weeks of the Academic year and comprises
two lectures and one tutorial per week. Total contact time is 48 hours lectures and 24 hours tutorials.

\begin{center}
\begin{tabular}{|c|c|c|c|c|c|c|}
\hline
  Term & Start & End & \multicolumn{2}{|c|}{Lectures} & \multicolumn{2}{|c|}{Tutorials} \\
   &Week&Week& Per Week & Total & Per Week & Total \\
\hline
  1 & 1 & 9 & 2 & 18 & 1 & 9 \\
  2 & 1 & 9 & 2 & 18 & 1 & 9 \\
  3 & 1 & 6 & 2 & 12 & 1 & 6 \\
  \hline
  \multicolumn{7}{|c|}{Total Contact Hours: 72}\\
  \hline
\end{tabular}
\end{center}

\paragraph{Course Description, Aims and Contribution to Programme} Engineering Mathematics I is a one-year
course taken by all Junior Freshman Engineering students. It covers calculus of functions of one
real variable, formalising and and building on Leaving Certificate mathematics. The course emphasizes
both theoretical foundations of calculus and application of mathematical methods.

The course is intended to enable students to recognise mathematical structures in practical problems, to
translate problems into mathematical language and to apply differentiation and integration to solve them.
The course provides the basic foundations for the Senior Freshman and Junior Sophister Engineering
Mathematics courses.

\paragraph{Learning Outcomes}

\begin{itemize}
\item The student will be able to recognise mathematical structures in practical problems, translate
problems into mathematical language, and analyse problems using methods from one-dimensional calculus.
\item The student will understand the theoretical foundations of calculus.
\item The student will be able to apply differentiation to find minima and maxima of a wide range of
functions of one real variable.
\item The student will be able to apply integration techniques to a wide range of functions.
\item The student will be able to calculate areas between graphs and volumes of solids of revolution
using integrals.
\end{itemize}

\paragraph{Content of Course}
\begin{itemize}
\item Equations of lines
\item Functions: Polynomials, rational functions, roots, exponential functions, logarithms, trigonometric
functions, hyperbolic functions
\item Limits, one-sided limits
\item Methods to find limits, Sandwich Theorem
\item Continuity of functions
\item Definition of derivatives
\item Derivatives of basic functions
\item Differentiation rules: Sum Rule, Constant Multiple Rule, Product Rule, Quotient Rule, Chain Rule
\item L'Hopital's Rule
\item Implicit differentiation
\item Logarithmic differentiation
\item Numerical application: The Newton-Raphson method
\item Intermediate Value Theorems, Rolle's Theorem
\item The shape of graphs, local and global maxima and minima, points of inflection
\item Integrals as areas
\item Mean Value Theorem
\item Fundamental Theorem of Calculus
\item Integrals of basic functions
\item Integration Methods: Integration by parts, substitution, trigonometric substitution, partial fractions
\item Improper integrals
\item Solids of revolution: disc formula and shell formula to find volumes.
\item Numerical application: Simpson's Rule
\item The length of curves
\item Infinite sequences
\item Infinite series, geometric and harmonic series
\item Convergence tests for series
\item Absoute and conditional convergence of series
\item Taylor series and Taylor polynomials
\end{itemize}

\paragraph{Contribution to Programme}
The course marks the start of the mathematics programme for Engineering students. It formalises Leaving Certificate
mathematics, and provides the foundation for the Senior Freshman mathematics courses. The course also provides
most of the mathematical methods needed in Junior Freshman physics and engineering science. There is some overlap
with contents of mathematics course 1E2, especially concerning differential equations, where 1E1 concentrates
on theoretical concepts while 1E2 focuses on application. Timetabling of these overlap topics is coordinated between
the two mathematics courses.

The course's contribution to the IEI Programme Areas and Outcomes are characterised in the following tables (H=high; M=moderate; L=low):

{\footnotesize\textsf{
\begin{center}
\begin{tabular}{|p{.9in}|p{.9in}|p{.9in}|p{.9in}|p{.9in}|p{.9in}|}
\hline
Science and Mathematics&Discipline Specific Technology&Information and Communications Technology&Design and Development&Engineering Practice&Social and Business Context\\
\hline
H & H & & L & & \\
\hline
\end{tabular}
\end{center}
}}
\begin{center}
\textsf{Contribution to IEI Programme Areas}
\end{center}

Contribution to IEI Programme Outcomes:

{\footnotesize\textsf{
\begin{center}
\begin{tabular}{|p{.9in}|p{.9in}|p{.9in}|p{.9in}|p{.9in}|p{.9in}|}
\hline
(a) The ability to derive and apply solutions from a knowledge of sciences, engineering sciences, technology and mathematics&
(b) The ability to identify, formulate, analyse and solve engineering problems&
(c) The ability to design a system, component or process to meet specified needs&
(d) An understanding of the need for high ethical standards in the practice of
engineering, \ldots &
(e) The ability to work effectively as an individual, in teams and in multidisciplinary
settings \ldots &
(f) The ability to communicate effectively with the engineering community
and with society at large\\
\hline
H & H & M & & M & L\\
\hline
\end{tabular}
\end{center}
}}
\begin{center}
\textsf{Contribution to IEI Programme Outcomes}
\end{center}

\paragraph{Teaching Strategies}
The teaching strategy is a mixture of lectures and problem-solving tutorials. The format of lectures is conventional;
however, the atmosphere is informal, and some interaction and discussion is normal. Students are encouraged to ask
questions in the lectures. In the tutorials, the students work on problems to practise and apply the methods introduced
in the lectures. Discussion of problems in small groups is encouraged.

\paragraph{Assessment} Assessment is by examinations and assignments. There are examinations at the end of
Michaelmas Term and Hilary Term (both joint 1E1/1E2 examinations), and an annual examination. The final result is
calculated as: Assignments 10\%, Michaelmas Term examination 20\%, Hilary Term examination 20\% and the
annual examination 50\%, or annual examination 100\%, whichever gives the higher mark for the individual student.

The assignments are marked by the tutorial demonstrators, who also discuss them in the tutorials, and provide feedback
for the students.

The Michaelmas and Hilary Term examinations are three hours long, and contain four questions from the 1E1 course and
four questions from the 1E2 course. The students are asked to answer all questions. The annual examination is three
hours long and contains 8 questions, all from 1E1 (there is a separate 1E2 examination). The students are required
to answer all questions.

\paragraph{Recommended Texts} Thomas' Calculus, 10th Edition, chapters 1-8


\paragraph{Further Information}~\\
Web site: \myhref{http://www.tcd.ie/Engineering/Courses/BAI/JF\_Subjects/1E1/}.\\
Exam paper: page \pageref{1E1}.\\
ECTS Credits: 8.
