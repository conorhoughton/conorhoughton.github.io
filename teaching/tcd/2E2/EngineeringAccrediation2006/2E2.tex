\documentclass[12pt,a4wide]{article}
\begin{document}
\subsection*{Engineering Mathematics IV (2E2)}
\subsection*{Lecturer: Conor Houghton}
\subsection*{ECTS Credits: 8}

\subsection*{Aims and Objectives}

Mathematical skills are useful though-out the engineering and
management sciences. This course teaches mathematical skills providing
training in techniques for solving differential equations. In this
way, it provides students with essential background knowledge with
specific engineering applications while also developing general skills
of use in their further training.

\begin{itemize}
\item[(a)] The ability to derive and apply solutions from a knowledge of sciences, engineering sciences, technology and mathematics.
\item[(b)] The ability to identify, formulate and solve engineering problems. 
\item[(c)] The ability to design a system, component or process to meet specified needs.
\item[(d)] An understanding of the need for high ethical standards in the practice of engineering.
\item[(e)] The ability to work effectively as an individual, in teams and in multidisciplinary settings.
\item[(f)] The ability to communicate effectively with the engineering community and with society at large.
\end{itemize}

\subsection*{Syllabus}

{\sl 2E2 Engineering Mathematics IV} is a year long course taken by
Senior Freshman Engineers and MSISS students. It covers calculational
techniques for solving ordinary differential equations and it
introduces vector calculus, the integral theorems and partial
differential equations. Through computational examples the course
builds up a powerful abstract description of the underlying structure
of dynamical equations and the method used to solve them. This enables
the students to apply what they have learned to new situations and to
integrate it with other techniques. The course also promotes
mathematical confidence and a mathematical sensibility, these are a
great asset in technical work.


 A lecture summary for 2004-05. This varies slightly from year to
 year, but the topics covered do not.
\begin{itemize}
\item Week 1: Introduction. Laplace transform. The Laplace transform of constants and exponentials. Linearity, examples. Laplace transform of powers of t. The first shift theorem. The Laplace transform f'.
\item Week 2: The Laplace transform of the f''. Solving differential equations. Lots of examples of solving differential equations using Laplace transforms. The partial fraction expansion when there is a repeated root.
\item Week 3: Another example with a repeated root. Examples with complex roots. Talk about the complex example. Examples with non-zero initial conditions. General discussion of differential equations and some terminology used.
\item Week 4: The second shift theorem, examples. Another Heaviside example. The Dirac delta functions.
\item Week 5: The convolution theorem, some convolutions. Revision of convolutions. Laplace transforms of periodic functions.
\item Week 6: Review of sequences and series. The Z-transforms. The Z-transform of a geometric sequence, differentiating this formula. The Z-transforms and the Z-transform of a geometric sequence again. The Z-transform of the unit pulse (1,0,0,...). Also, properties of the Z-transform. The shift theorems, delaying and advancing.
\item Week 7: Using the Z-transform to solve difference equations, some examples. Using the Z-transform to solve difference equations, examples.
\item Week 8: Beginning systems of differential equations. The mixing problem. Finishing the mixing problem, another mixing problem.
\item Week 9: Linear homogeneous first order equations, definitions and discussion, some examples. Christmas Quiz, an ungraded and entertaining revision quiz.
\item Week 10: Linear homogeneous first order equations, drawing phase portraits. More on phase portraits, improper nodes and saddle-points.
\item Week 11: More on improper nodes and saddle-points. Start of circles.
\item Week 12: More circles and start of spirals. Spirals. Examples with only one eigenvector.
\item Week 13: Examples with only one eigenvector, start of inhomogeneous equations: revise ordinary differential equations case. System of inhomogeneous equations.
\item Week 14: More in inhomogeneous equations, start of pendulum. The pendulum and linearization.
\item Week 15: More on linearization. Linearization, wind resistance and spirals 
\item Week 16: Series solutions, general explanation and simple example.
\item Week 17: Series solutions, more examples Series solutions, more examples, method of Frobenius.
\item Week 18: Series solutions, method of Frobenius. Bessel's eqn. The Legendre equation, Legendre polynomials.
\item Week 19: More on Legendre equation. Revision of vectors, dot products, cross products. Curves in space, length of a curve.
\item Week 20: Only one lecture because of Good Friday. Gradient of a scalar field, the directional derivative, normal to a surface.
\item Week 21: More on the normal, start of div. More on div, continuity equation. Curl.
\item Week 22: Start of the Gauss theorem. More on Gauss theorem, derivation of the heat equation.
\item Week 23: The Heat equation, solving it: separation of variables. Solving the heat equation, continued.
\item Week 24: Different boundary conditions. Revision of heat equation. Concluding remarks.
\end{itemize}


\subsection*{Recommended Texts}
\begin{itemize}
\item Erwin Kreyszig, {\sl Advanced Engineering Mathematics.} 
\item Glyn James, {\sl Advanced Modern Engineering Mathematics.}
\end{itemize}
As described under Teaching Strategies, the course web-site contains a considerable quantity of material.\\
Web site: {\tt http://www.maths.tcd.ie/~houghton/2E2/}

\subsection*{Learner Outcomes}
Students will able to describe, analyze and visualize the behaviour of dynamical equations. In particular the students will be able to 
\begin{itemize}
\item solve differential and difference equations by applying Laplace and Z transforms.
\item solve linear systems of differential equations and analyze and categorize the behaviour of nonlinear systems using phase plane methods. 
\item calculate the series solution and connect the series behaviour to the dynamics of the equation. 
\item sketch the role of special functions.
\item adapt their understanding of ordinary differential equations to new situations and to more complex scenarios.
\item formulate vector calculus descriptions and apply the integral theorems, they will begin to synthesize these ideas.
\item employ separation of variable techniques, solve the heat equations in two-dimensions and outline the further development of the theory of partial differential equation theory.
\item communicate more effectively mathematically and to collaborate mathematically.
\end{itemize}


\subsection*{Teaching Strategies} 

The course runs though-out the year with two lectures every week and
one tutorial. There is no tutorial on the first week, so, with 26
weeks of term there are 48 lectures and 25 tutorials. The students are
split into groups of about thirty for tutorials, they are presented
with a previously unseen problem sheet on the previous weeks material,
they work on this problem sheet for an hour, often collaborating with
each other. A tutor assists and answers questions. At the end of the
hour the answer sheets are collected, graded and returned the next
week.

There are three things that must be communicated to a student in a
mathematics course: technique, theory and joy.  I try to communicate
these three by placing enactment at the center of my teaching
method. After careful preparation, I lecture in an apparently casual
fashion , without notes, on a blackboard, going through calculational
example after calculational example, while always commentating on the
larger, more impressive, principles behind the examples. I express my
own joy in mathematics and my feeling that a sense of the mathematical
is current in the beauty of life, by linking what I am doing on the
board, sometimes directly, sometimes tangentially, to anecdotes,
humorous comments and wry stories.

I believe a humorous and slightly unpredictable narrative is useful in
mathematics lecturing. It helps create a sense of intellectual
excitement and it helps involve the class in a direct way in the
calculations you are demonstrating. Making a class laugh transforms
them from a collections of individuals into a single cohesive partner
in communication. Of course, an unpredictable narrative is only
possible in the context of a carefully structured presentation, I try
to make sure that my blackboard notes give a cogent account of the
lecture and that the course follows a careful and logical progression.

In addition I have developed a large amount of supporting web-based
material, as well as the tutorial problem sheets and solutions I have
written, my teaching web-sites contain lecture lists, book references,
notes, example calculations and solutions for past exams.  There is
also an anonymous feedback and Questions and Answers facility. By
going to the web-site students can contact me anonymously, either to
make comments, positive or negative, or to ask technical questions
related to the material. In either case, I write an answer and add
both the comment and my reaction, or the question and my answer to the
Questions and Answers page for all to see.

My anonymous feedback facility is part of a general teaching policy I
think of as exposure.  Along with enactment, exposure is a basic
element of my teaching philosophy: I try not to place barriers between
myself and the students, I accept criticism, I welcome questions and
discussion, I avoid inappropriate teaching technology and I try not to
evoke any authority outside of my own willingness to teach. I try to
open myself up to the teaching process.

\subsection*{Assessment} The cumulative tutorial mark constitutes
10\% of the course mark for the year, the course is examined by a
three hour end of year exam and this constitutes the remaining
90\%. In the exam the students are asked to do six questions out of
nine, each question is, formally, or informally divided into three
parts, the first part requiring a definition or statement reproducing
some information given in class, the second part involves a
calculation modeled on the examples demonstrated in class and
rehearsed in tutorials and the third part, again calculational,
requires the student demonstrate an understanding of why the
calculational method works.

\end{document}