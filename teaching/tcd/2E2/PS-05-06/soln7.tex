\documentclass[12pt]{article}
\usepackage{a4wide, amsfonts, epsfig}
\newcommand\soln{\noindent\textit{Solution:} }


%skyline stuff
\font\upright=cmu10 scaled\magstep1
\setlength{\unitlength}{0.012500in}
\begingroup\makeatletter\ifx\SetFigFont\undefined
\def\x#1#2#3#4#5#6#7\relax{\def\x{#1#2#3#4#5#6}}%
\expandafter\x\fmtname xxxxxx\relax \def\y{splain}%
\ifx\x\y   % LaTeX or SliTeX?
\gdef\SetFigFont#1#2#3{%
  \ifnum #1<17\tiny\else \ifnum #1<20\small\else
  \ifnum #1<24\normalsize\else \ifnum #1<29\large\else
  \ifnum #1<34\Large\else \ifnum #1<41\LARGE\else
     \huge\fi\fi\fi\fi\fi\fi
  \csname #3\endcsname}%
\else
\gdef\SetFigFont#1#2#3{\begingroup
  \count@#1\relax \ifnum 25<\count@\count@25\fi
  \def\x{\endgroup\@setsize\SetFigFont{#2pt}}%
  \expandafter\x
    \csname \romannumeral\the\count@ pt\expandafter\endcsname
    \csname @\romannumeral\the\count@ pt\endcsname
  \csname #3\endcsname}%
\fi
\fi\endgroup

\begin{document}
\begin{center}
{\bf 2E2 Tutorial Sheet 7 Solutions\footnote{Conor Houghton, {\tt houghton@maths.tcd.ie}, see also {\tt http://www.maths.tcd.ie/\char126 houghton/2E2.html}}}\\[1cm] 27 November 2005
\end{center}
{
{\bf Questions}
\begin{enumerate}
\item (2) Find the Z-transform of the sequence $(2,0,1,0,-3,0,0,\ldots)$  where all the other entries are zero. 
\vskip 1cm
\soln So, use the formula
\begin{equation}
{\cal Z}[(x_n)]=\sum_{n=0}^\infty{\frac{x_n}{z^n}}
\end{equation}
to get
\begin{equation}
{\cal Z}[(2,0,1,0,-3,0,0,\ldots)]=2+\frac{1}{z^2}-\frac{3}{z^4}
\end{equation}
\vskip 1cm
\item (2) Find the Z-transform of the geometric sequence $(3,15,75,375,\ldots)$. 
\vskip 1cm
\soln This sequence has the form $3\times 5^n$ so we can use the
formula for the geometric sequence to get
\begin{equation}
{\cal Z}[(3\times 5^n)]=3{\cal Z}[(5^n)]=\frac{3z}{z-5}
\end{equation}
\vskip 1cm
\item (2) Find the Z-transform of the sequence $(6,12,24,\ldots)$ both
by considering it the advance of the sequence $(3,6,12,24,\ldots)$ and
by applying the formula for geometrical sequences directly. Do you get
the same answer?

\soln Now this example is advanced by one step, so we use the formula
\begin{equation}
{\cal Z}[(x_{k+1})]=zX(z)-zx_0
\end{equation}
where $X(z)={\cal Z}[(x_{n})]$. In this case we have
\begin{equation}
{\cal Z}[(3,6,12,24,\ldots)]=\frac{3z}{z-2}
\end{equation}
so
\begin{equation}
{\cal Z}[(6,12,24,\ldots)]=z\frac{3z}{z-2}-3z=\frac{3z^2}{z-2}-3z=\frac{6z}{z-2}\end{equation}
Working directly, this is the sequence $6\times 2^n$ and so the Z-transform is
\begin{equation}
{\cal Z}[(6\times 2^n)]=6{\cal Z}[(2^n)]=\frac{6}{z-2}
\end{equation}
which is, of course, the same answer.



\item (2) For the difference equation
\begin{equation}
x_{k+1}=-3x_k
\end{equation}
with $x_0=1$ work out $X(z)={\cal Z}[(x_n)]$ by taking the Z-transform of both sides of the equation. Use this to solve the equation.
\vskip 1cm
\soln So taking the Z-transform of both sides we get
\begin{equation}
zX-z=-3X
\end{equation}
So 
\begin{equation}
X=\frac{z}{z+3}
\end{equation}
Now
\begin{equation}
x_k=(-3)^k
\end{equation}
\end{enumerate}
}
\end{document}
