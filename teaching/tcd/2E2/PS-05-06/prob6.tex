 \documentclass[12pt]{article}
\usepackage{a4wide, amsfonts, epsfig}
\newcommand{\soln}{\noindent\textit{Solution:}}

%skyline stuff
\font\upright=cmu10 scaled\magstep1
\setlength{\unitlength}{0.012500in}
\begingroup\makeatletter\ifx\SetFigFont\undefined
\def\x#1#2#3#4#5#6#7\relax{\def\x{#1#2#3#4#5#6}}%
\expandafter\x\fmtname xxxxxx\relax \def\y{splain}%
\ifx\x\y   % LaTeX or SliTeX?
\gdef\SetFigFont#1#2#3{%
  \ifnum #1<17\tiny\else \ifnum #1<20\small\else
  \ifnum #1<24\normalsize\else \ifnum #1<29\large\else
  \ifnum #1<34\Large\else \ifnum #1<41\LARGE\else
     \huge\fi\fi\fi\fi\fi\fi
  \csname #3\endcsname}%
\else
\gdef\SetFigFont#1#2#3{\begingroup
  \count@#1\relax \ifnum 25<\count@\count@25\fi
  \def\x{\endgroup\@setsize\SetFigFont{#2pt}}%
  \expandafter\x
    \csname \romannumeral\the\count@ pt\expandafter\endcsname
    \csname @\romannumeral\the\count@ pt\endcsname
  \csname #3\endcsname}%
\fi
\fi\endgroup

\begin{document}
\begin{center}

{\bf 2E2 Tutorial Sheet 6 First Term\footnote{Conor Houghton, {\tt houghton@maths.tcd.ie}, see also {\tt http://www.maths.tcd.ie/\char126 houghton/2E2.html}}\\[1cm] 20 November 2005}
\end{center}


\renewcommand{\labelenumi}{\arabic{enumi}.}
\noindent {\bf Useful facts:}

\begin{itemize}

\item Laplace transform of a periodic function with
period $c$:
\begin{equation}
{\cal L}(f)=\frac{1}{1-e^{-cs}}\int_0^c f(t)e^{-st}dt
\end{equation}

\item Integration by parts:
\begin{equation}
\int_a^b udv=\left. uv\right]_a^b-\int_a^b vdu
\end{equation}

\item The convolution:
\begin{equation}
f*g(t)=\int_0^t f(\tau)g(t-\tau)d\tau
\end{equation}


\item The convolution theorem, for two functions $f(t)$ and $g(t)$
\begin{equation}
{\cal L}(f*g)={\cal L}(f){\cal L}(g)
\end{equation}

\end{itemize}

\vskip 1cm

\noindent {\bf Questions:}

\begin{enumerate}

\item (2) Verify the formula ${\cal L}(f*g)={\cal L}(f){\cal L}(g)$ in the case where $f=\exp{(2t)}$ and $g=\exp{(2t)}$. 
\vskip 1cm
\item (3) Find the convolution $(f*g)(t)$ when $f(t) = t$, $g(t) = e^{2t}$ ($t
\geq 0$).
\vskip 1cm
\item (3) Use the convolution theorem to find the function $f(t)$ with
\begin{equation}
\mathcal{L}(f) = \frac{1}{s^2(s-4)}.
\end{equation}

\end{enumerate}

\vfill


\end{document}
