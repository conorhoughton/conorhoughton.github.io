\documentclass[12pt]{article}
\usepackage{a4wide, amsfonts, epsfig}
\newcommand\soln{\noindent\textit{Solution:} }


%skyline stuff
\font\upright=cmu10 scaled\magstep1
\setlength{\unitlength}{0.012500in}
\begingroup\makeatletter\ifx\SetFigFont\undefined
\def\x#1#2#3#4#5#6#7\relax{\def\x{#1#2#3#4#5#6}}%
\expandafter\x\fmtname xxxxxx\relax \def\y{splain}%
\ifx\x\y   % LaTeX or SliTeX?
\gdef\SetFigFont#1#2#3{%
  \ifnum #1<17\tiny\else \ifnum #1<20\small\else
  \ifnum #1<24\normalsize\else \ifnum #1<29\large\else
  \ifnum #1<34\Large\else \ifnum #1<41\LARGE\else
     \huge\fi\fi\fi\fi\fi\fi
  \csname #3\endcsname}%
\else
\gdef\SetFigFont#1#2#3{\begingroup
  \count@#1\relax \ifnum 25<\count@\count@25\fi
  \def\x{\endgroup\@setsize\SetFigFont{#2pt}}%
  \expandafter\x
    \csname \romannumeral\the\count@ pt\expandafter\endcsname
    \csname @\romannumeral\the\count@ pt\endcsname
  \csname #3\endcsname}%
\fi
\fi\endgroup

\begin{document}
\begin{center}
{\bf 2E2 Tutorial Sheet 16 Second Term}\footnote{Conor
Houghton, {\tt houghton@maths.tcd.ie} and {\tt
http://www.maths.tcd.ie/\char126 houghton/ 2E2.html}}
\\[1cm]
 26 February 2006
\end{center}
{
\noindent{\bf Useful facts:}\vskip .5cm
\begin{itemize}
\item Convert into $y_1$, $y_2$ form.
\item Find the stationary points, these are where $y_1'=y_2'=0$.
\item Approximate near the stationary points, drop squares and cubes, use $\sin{\theta}\approx \theta$ for small $\theta$.
\item The arrows will go left to right above the $y_1$-axis and right to left below it.
\item For the trigonometry tables
\begin{eqnarray}
\cos{\left(\frac{\pi}{2}+\theta\right)}&=&-\sin{\theta}\cr
\cos{\left(3\frac{\pi}{2}+\theta\right)}&=&\sin{\theta}
\end{eqnarray}
\end{itemize}
\vskip .5cm
\noindent{\bf Questions:}
\begin{enumerate}
\item (4) By linearizing around the critical points, draw the phase
plane portrait of
\begin{equation}
y''+y-y^3=0
\end{equation}
\vskip .5cm
\item (4) By linearizing around the critical points, draw the phase
plane portrait of
\begin{equation}
y''=\cos{2y}
\end{equation}
\end{enumerate}
}
\end{document}



