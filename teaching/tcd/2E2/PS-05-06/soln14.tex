\documentclass[12pt]{article}
\usepackage{a4wide, amsfonts, epsfig}
\newcommand\soln{\noindent\textit{Solution:} }


%skyline stuff
\font\upright=cmu10 scaled\magstep1
\setlength{\unitlength}{0.012500in}
\begingroup\makeatletter\ifx\SetFigFont\undefined
\def\x#1#2#3#4#5#6#7\relax{\def\x{#1#2#3#4#5#6}}%
\expandafter\x\fmtname xxxxxx\relax \def\y{splain}%
\ifx\x\y   % LaTeX or SliTeX?
\gdef\SetFigFont#1#2#3{%
  \ifnum #1<17\tiny\else \ifnum #1<20\small\else
  \ifnum #1<24\normalsize\else \ifnum #1<29\large\else
  \ifnum #1<34\Large\else \ifnum #1<41\LARGE\else
     \huge\fi\fi\fi\fi\fi\fi
  \csname #3\endcsname}%
\else
\gdef\SetFigFont#1#2#3{\begingroup
  \count@#1\relax \ifnum 25<\count@\count@25\fi
  \def\x{\endgroup\@setsize\SetFigFont{#2pt}}%
  \expandafter\x
    \csname \romannumeral\the\count@ pt\expandafter\endcsname
    \csname @\romannumeral\the\count@ pt\endcsname
  \csname #3\endcsname}%
\fi
\fi\endgroup

\begin{document}
\begin{center}
{\bf 2E2 Tutorial Sheet 14 Second Term}\footnote{Conor
Houghton, {\tt houghton@maths.tcd.ie} and {\tt
http://www.maths.tcd.ie/\char126 houghton/ 2E2.html}}
\\[.5cm]
 10 February 2006
\end{center}
{
\begin{enumerate}
\item (2) Find the general solution to 
\begin{equation}
y'-2y=-t
\end{equation}
\vskip .5cm
\soln This follows from the general solution to
\begin{equation}
y'=ry+f(t)
\end{equation}
which is
\begin{equation}\label{gensol}
y=Ce^{rt}+e^{rt}\int_0^t e^{-r\tau}f(\tau) d\tau
\end{equation}
so here $r=2$ and $f(t)=-t$ so, using integration by parts
\begin{eqnarray}
y&=&Ce^{2t}-e^{2t}\int_0^t \tau e^{-2\tau}d\tau\nonumber\\
&=&Ce^{2t}-e^{2t}\left\{-\frac{1}{2}te^{-2t}+\frac{1}{2}\int e^{-2\tau}d\tau\right\}\cr
&=&Ce^{2t}-e^{2t}\left\{-\frac{1}{2}te^{-2t}-\frac{1}{4}(e^{-2t})\right\}\cr
&=&Ce^{2t}+\frac{t}{2}+\frac{1}{4}
\end{eqnarray}
where $\exp{2t}$ terms have been absorbed in the $C\exp{2t}$.
\vskip .5cm
\item (3) Find the general solution to
\begin{eqnarray}
y_1'&=&5y_2-23\nonumber\\ y_2'&=&5y_1+15.
\end{eqnarray}
with $y_1(0)=-3$ and $y_2(0)=5$.
\vskip .5cm
\soln First of all rewrite the equation in matrix form
\begin{equation}
{\bf y}'=\left(\begin{array}{cc}0&5\\5&0\end{array}\right){\bf y}+\left(\begin{array}{c}-23\\15\end{array}\right).
\end{equation}
Now, the matrix
\begin{equation}
A=\left(\begin{array}{cc}0&5\\5&0\end{array}\right)
\end{equation}
has eigenvalue $\lambda_1=5$ with eigenvector  
\begin{equation}
{\bf x}_1=\left(\begin{array}{c}1\\1\end{array}\right)
\end{equation}
and eigenvalue $\lambda_1=-5$ with eigenvector  
\begin{equation}
{\bf x}_2=\left(\begin{array}{c}1\\-1\end{array}\right)
\end{equation}
so if we write 
\begin{equation}\label{yansatz}
{\bf y}=f_1{\bf x}_1+f_2{\bf x}_2
\end{equation}
and subsituting this into the differential equation gives
\begin{equation}
(f_1'-5f_1){\bf x}_1+(f_2'+5f_2){\bf
x}_2=\left(\begin{array}{c}-23\\15\end{array}\right).
\end{equation}
\vskip .5cm
Now to seperate the equation lets decompose the inhomogeneous part, sometimes called the forcing term, over the two eigenvectors:
\begin{equation}
\left(\begin{array}{c}-23\\15\end{array}\right)=h_1{\bf x}_1+h_2{\bf x}_2
\end{equation}
or, writing it out,
\begin{equation}
\left(\begin{array}{c}-23\\15\end{array}\right)=\left(\begin{array}{c}h_1+h_2\\h_1-h_2\end{array}\right)
\end{equation}
and, hence, $h_1=-4$ and $h_2=-19$. Putting this back into the equation leads to
\begin{equation}
(f_1'-5f_1){\bf x}_1+(f_2'+5f_2){\bf
x}_2=-4{\bf x}_1-19{\bf x}_2
\end{equation}
Hence
\begin{equation}
f_1'-5f_1=-4.
\end{equation}
Thus, this is of the form $y'=ry+f$ with $r=5$, $f(t)=-4$ and so
\begin{equation}
f_1=C_1e^{5t}-4e^{5t}\int^t{e^{-5t}dt}
\end{equation}
and so 
\begin{equation}
f_1=C_1e^{5t}+\frac{4}{5}
\end{equation}
where I have committed the common notational laziness of using $t$ inside the integration sign as well as outside, people do this a lot, because you can think of the integral as being closed off from the rest of the equation, but if it confuses you, keep using $\tau$ inside the integral. 
Similarly,
\begin{equation}
f_2'+5f_2=-19
\end{equation}
Thus, $r=-5$, $f(t)=19$ and using integrating gives above 
\begin{equation}
f_2=C_2e^{-5t}-\frac{19}{5}.
\end{equation}
The general solution is therefore
\begin{equation}
{\bf y}=\left(C_1e^{5t}+\frac{4}{5}\right)\left(\begin{array}{c}1\\1\end{array}\right)+\left(C_2e^{-5t}-\frac{19}{5}\right)\left(\begin{array}{c}1\\-1\end{array}\right).
\end{equation}
\vskip .5cm
If $y_1(0)=-3$ and $y_2(0)=5$ then we get
\begin{equation}
\left(\begin{array}{c}-3\\5\end{array}\right)
=\left(C_1+\frac{4}{5}\right)\left(\begin{array}{c}1\\1\end{array}\right)+\left(C_2-\frac{19}{5}\right)\left(\begin{array}{c}1\\-1\end{array}\right).
\end{equation}
and hence
\begin{eqnarray}
-3&=&C_1+C_2-3\cr
5&=&C_1-C_2+\frac{23}{5}
\end{eqnarray}
so $C_1=-C_2=1/5$ and
\begin{equation}
{\bf y}=\left(\frac{1}{5}e^{5t}+\frac{4}{5}\right)\left(\begin{array}{c}1\\1\end{array}\right)+\left(-\frac{1}{5}e^{-5t}-\frac{19}{5}\right)\left(\begin{array}{c}1\\-1\end{array}\right).
\end{equation}
\vskip .5cm
\item (2) Find the solution to
\begin{eqnarray}
y_1'&=&y_1+3y_2+e^t\nonumber\\
y_2'&=&3y_1+y_2
\end{eqnarray}
\vskip .5cm
\soln Here we have
\begin{equation}
{\bf y}=A{\bf y}+\left(\begin{array}{c}e^t\\0\end{array}\right)
\end{equation}
where 
\begin{equation}
A=\left(\begin{array}{cc}1&3\\3&1\end{array}\right),
\end{equation}
this has eigenvalue $\lambda_1=4$ with eigenvector  
\begin{equation}
{\bf x}_1=\left(\begin{array}{c}1\\1\end{array}\right)
\end{equation}
and eigenvalue $\lambda_1=-2$ with eigenvector  
\begin{equation}
{\bf x}_2=\left(\begin{array}{c}1\\-1\end{array}\right).
\end{equation}
\vskip .5cm
Once again, we split the forcing term over the two eigenvectors:
\begin{equation}
\left(\begin{array}{c}e^t\\0\end{array}\right)=\frac{e^t}{2}{\bf x}_1+\frac{e^t}{2}{\bf x}_2
\end{equation}
We get
\begin{equation}
f_1-4f_1=\frac{1}{2}e^t
\end{equation}
so 
\begin{equation}
f_1=C_1e^{4t}+\frac{1}{2}e^{4t}\int^t e^{-3t}dt.
\end{equation}
and so, 
\begin{equation}
f_1=C_1e^{4t}-\frac{1}{6}e^t
\end{equation}
In the same way
\begin{equation}
f_2+2f_2=\frac{1}{2}e^t
\end{equation}
and so
\begin{equation}
f_2=C_2e^{-2t}+\frac{1}{2}e^{-2t}\int^t e^{3t}dt.
\end{equation}
Integrating gives
\begin{equation}
f_2=C_2e^{-t}+\frac{1}{6}e^t
\end{equation}
This means 
\begin{equation}
{\bf y}=\left(C_1e^{4t}-\frac{1}{6}e^t\right)\left(\begin{array}{c}1\\1\end{array}\right)+\left(C_2e^{-t}+\frac{1}{6}e^t\right)\left(\begin{array}{c}1\\-1\end{array}\right)
\end{equation} 
\vskip .5cm
\item (1) Rewrite $y''+4y'-3y=0$ as a system of two first order differential equations.
\vskip .5cm
\soln So $y_1=y$, $y_2=y_1'$ hence $y_2'=y''=3y-4y'=3y_1-4y_2$ giving
\begin{eqnarray}
y_1'&=&y_2\cr
y_2'&=&3y_1-4y_2
\end{eqnarray}
\end{enumerate} 
}
\end{document}
