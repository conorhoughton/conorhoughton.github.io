
\documentclass[12pt]{article}
\usepackage{a4wide, amsfonts, epsfig}
\newcommand\soln{\noindent\textit{Solution:} }


%skyline stuff
\font\upright=cmu10 scaled\magstep1
\setlength{\unitlength}{0.012500in}
\begingroup\makeatletter\ifx\SetFigFont\undefined
\def\x#1#2#3#4#5#6#7\relax{\def\x{#1#2#3#4#5#6}}%
\expandafter\x\fmtname xxxxxx\relax \def\y{splain}%
\ifx\x\y   % LaTeX or SliTeX?
\gdef\SetFigFont#1#2#3{%
  \ifnum #1<17\tiny\else \ifnum #1<20\small\else
  \ifnum #1<24\normalsize\else \ifnum #1<29\large\else
  \ifnum #1<34\Large\else \ifnum #1<41\LARGE\else
     \huge\fi\fi\fi\fi\fi\fi
  \csname #3\endcsname}%
\else
\gdef\SetFigFont#1#2#3{\begingroup
  \count@#1\relax \ifnum 25<\count@\count@25\fi
  \def\x{\endgroup\@setsize\SetFigFont{#2pt}}%
  \expandafter\x
    \csname \romannumeral\the\count@ pt\expandafter\endcsname
    \csname @\romannumeral\the\count@ pt\endcsname
  \csname #3\endcsname}%
\fi
\fi\endgroup

\begin{document}
\begin{center}
{\bf 2E2 Tutorial Sheet 24, Solutions\footnote{Conor Houghton, {\tt
houghton@maths.tcd.ie} and {\tt http://www.maths.tcd.ie/\char126
houghton/2E2.html} }}\\[.5cm] 18 May 2006\\[1cm]
\end{center}
{
\noindent This problem sheet relates to solving the Dirichlet problem
for the heat equation.. Dirichlet boundary conditions fix the values
at the end.  Consider an iron bar on which heat obeys the heat
equation:
\begin{equation}
\frac{\partial^2 u}{\partial x^2}=\frac{\partial u}{\partial t}
\end{equation}
where  $u(x,t)$ and the boundary conditions are
\begin{equation}
 u(0,t)= u(\pi,t)=0
\end{equation}
\begin{enumerate}
\item (2) Writing $u(x,t)=T(t)X(x)$ show this equation is equivalent to the equations
\begin{eqnarray}
\frac{d^2 X}{d x^2}&=&EX\nonumber\\
\frac{d T}{d t}&=&ET
\end{eqnarray}
where $E$ is some constant.
\vskip .25cm
\soln The main trick for multi-variable problems like this is to try and seperate the
equation for $u$ into a function of $x$ and $t$ into two equations, one for
a function of $x$ and the other for a function of $t$. This way, we
are able to use the ordinary differential equation methods we have
already learned to solve the more complicated example where there is
more than one variables. There are a number of trick for doing this
depending on the equation, but the one that works for the heat
equation, along with a number of other important examples, is the
seperation of variables method. According to this method, you assume
the function $u$ can be written as the multiple of two function, one,
which we will call $T$ is just a function of $t$ and the other, which
we call $X$ is just a function of $x$. Obviously, not every function
of $x$ and $t$ can be split in this way, we are just imagining that
$u$ can be. Although we won't make much of it here, in fact most functions can be written as the sum of
functions that are of this form and, for a linear equation, that's
good enough.

So, we make the assumption and substitute $u=TX$ into the equation. The $x$ differenciation only acts on the $X$ and, in turn, the $X$ only depends on $x$. In the same way, the $t$ differentiation only acts on $T$ and $T$ in turn only depends on $t$. Thus, the equation becomes
\begin{equation}
T\frac{d^2 X}{d x^2}=\frac{d T}{dt}X
\end{equation}
Now divide across by $TX$
\begin{equation}
\frac{1}{X}\frac{d^2 X}{d x^2}=\frac{1}{T}\frac{d T}{dt}
\end{equation}
We still have only one equation, but it has a very suprising form, all
the $x$ stuff is on one side and all the $t$ stuff is on the
other. $x$ and $t$ are supposed to be independent variables: $x$
doesn't depend on $t$ and $t$ doesn't depend on $x$. Hence, we should
be able to hold $x$ fixed and change $t$ and still have the equation
stay true. But holding $x$ fixed and changing $t$ can only change the
right hand side of the equation, not the left, and that would be a
contradition, the equation wouldn't hold, unless it was a fact that
the right hand side didn't change when $t$ was changed, in other
words, unless the right hand side of the equation is a
constant. Basically, by the same arguement the left hand side must be
a constant too, so the heat equation, using seperation of variables
comes down to an equation which only makes sense if
\begin{equation}
\frac{1}{T}\frac{d T}{dt}
\end{equation}
is a constant
\begin{equation}
\frac{1}{T}\frac{d T}{dt}
\end{equation}
is a constant. Usually, this constant is given the name $E$ and hence,
by this suprising trick, the seperation of variables assumption
$u=X(x)T(t)$ has turned one equation with two variables into two
equations with one variable:
\begin{eqnarray}
\frac{1}{X}\frac{d^2 X}{d x^2}&=&E\nonumber\\
\frac{1}{T}\frac{d T}{d t}&=&E
\end{eqnarray}
The only thing is we don't know the value of this new constant $E$, in
fact, if you solve these two equations to get $X$ and $T$ and multiply
them together to get $u$, you have a solution to the heat equation, no
matter what value $E$ has, but we will see that only special values of
$E$ give solutions that satisfy the boundary conditions.

Multiply the first equation by $X$ and the second by $T$ to get the
answer.
\begin{eqnarray}
\frac{d^2 X}{d x^2}&=&EX\nonumber\\
\frac{d T}{d t}&=&ET
\end{eqnarray}
\vskip .25cm
\item (2) Solve these equations and argue from the boundary conditions that $E$ must be negative. Writing $E=-k^2$ calculate what values of $k$ satisfy the boundary conditions.
\vskip .25cm
\soln Now the $T$ equation is easy to solve, either by guessing the answer and substituting it in, or by integration or indeed by using the Laplace transform, we get
\begin{equation}
u=Ae^{Et}
\end{equation}
For the second $X$ there are two possibile types of solution depending
on the sign of $E$, basically, we will see, if $E$ is positive, then
the $X$ equation is solved by real exponentials, but if it is
negative, it is solved by trignometric functions. What we find is that
the corresponding $u$ can only solve the boundary condition if we have
the trignometric solution, if $E$ is negative. 

So, to go through that, if $E$ is positive, we make that clear by
writing it as the square of something, the square of a real number is
positive, so we write $E=k^2$ where $k$ denotes a real number, the
square root of $E$. Now, it is easy to check, for example, by
substituting back in, or solving by converting to first order and
using matrices or even by using Laplace that the solution is
\begin{equation}
X=C_1e^{kx}+C_2e^{-kx}
\end{equation}
In other words, there are lots of ways to solve the equation, the
simplest is probably to kind of remember the solution and check you
guess by substitution:
\begin{eqnarray}
X'&=&C_1ke^{kx}-kC_2e^{-kx}\cr
X''&=&C_1k^2e^{kx}+k^2C_2e^{-kx}=EX\cr
\end{eqnarray}

On the other hand, if $E$ is negative, we can make it clear by
writting it is minus the square of something, so we write $E=-k^2$, in which case the solution is
\begin{equation}
X=C_1\cos{kx}+C_2\sin{kx}
\end{equation}
Again, there are a number of ways of seeing these are the solutions,
the first is to know they are the solutions and quickly check that
differenciating $X$ twive gives the right thing, another is to split
the equation into two first order equations and the write in matrix
form, as we did after Christmas and the third is to use the Laplace
transform.
\vskip .25cm
Now, the $u$ that comes from the positive case can never satisfy the
boundary conditions,
\begin{equation}
u(0,t)=u(\pi,t)=0
\end{equation}
because we have
\begin{equation}
u=TX=\left(C_1e^{kx}+C_2e^{-kx}\right)e^{k^2t}
\end{equation}
where the $A$ has been absorbed into the other two arbitrary constants.
and so, if 
\begin{equation}
 u(0,t)=0
\end{equation}
then 
\begin{equation}
\left(C_1+C_2\right)e^{k^2t}
\end{equation}
so that $C_1=-C_2$. Now 
\begin{equation}
 u(\pi,t)=C_1\left(e^{k\pi}-e^{-k\pi}\right)e^{k^2t}
\end{equation}
which is only zero if 
\begin{equation}
e^{k\pi}-e^{-k\pi}=0
\end{equation}
which never happens for non-zero $\pi$ and $k$. 
\vskip .25cm
Choosing the negative solutions, this gives 
\begin{equation}
u=TX=\left[C_1\cos{(kx)}+C_2\sin{(kx)}\right]e^{-k^2t}
\end{equation}
so
\begin{equation}
 u(t,0)=C_1e^{-k^2t}
\end{equation}
and only gives zero when $C_1=0$, hence
\begin{equation}
 u(\pi,t)= C_2\sin{(k\pi)}e^{-k^2t}
\end{equation}
and this is zero if $k=n$ where $n$ is any integer. Hence, if $C_1=0$ and $E$ is fixed to
be
\begin{equation}
E=-n^2
\end{equation}
where $n$ is an integer we get a $u$ that solves the heat equation and
satisfies the boundary conditions. Hence, solutions are of the form
\begin{equation}
u=A_n\sin{nx}e^{-n^2 t}
\end{equation}
where, for convenience, we are using the name $A_n$ for the arbitrary
constant when $E=-n^2$.

In fact, this is a linear equation so the sum of solutions is a solution and so
\begin{equation}
u=\sum_{n=0}^\infty A_n\sin{nx}e^{-n^2 t}
\end{equation}
is a solution. This is the clever thing, we start off looking for one
solution, but end up with a whole family of solutions, one for each
$n$ and, since the equation is linear we can just add them all
together. There is still the initial condition to deal with, in
general, matching the initial condition comes down to Fourier series,
but we have an easier example in the next question.
\vskip .25cm
\item (2) Say the initial condition is 
\begin{equation}
u(x,0)=f(x)=\sin{x}-\sin{4x}
\end{equation}
what is $u(x,t)$.
\vskip .25cm
\soln So the general solution is
\begin{equation}
u=\sum_{n=0}^\infty A_n\sin{nx}e^{-n^2 t}
\end{equation}
and at $t=0$ this gives
\begin{equation}
u(x,0)=\sum_{n=0}^\infty A_n\sin{nx}
\end{equation}
which matches up with the initial condition if $A_1=1$, $A_4=-1$ and
all the others are zero. Hence putting these $A$s back in
\begin{equation}
u(x,t)=\sin{x}e^{-t}-\sin{4x}e^{-16t}
\end{equation}
\end{enumerate}
}
\end{document}



