\documentclass[12pt]{article}
\usepackage{a4wide, amsfonts, epsfig}
\newcommand\soln{\noindent\textit{Solution:} }


%skyline stuff
\font\upright=cmu10 scaled\magstep1
\setlength{\unitlength}{0.012500in}
\begingroup\makeatletter\ifx\SetFigFont\undefined
\def\x#1#2#3#4#5#6#7\relax{\def\x{#1#2#3#4#5#6}}%
\expandafter\x\fmtname xxxxxx\relax \def\y{splain}%
\ifx\x\y   % LaTeX or SliTeX?
\gdef\SetFigFont#1#2#3{%
  \ifnum #1<17\tiny\else \ifnum #1<20\small\else
  \ifnum #1<24\normalsize\else \ifnum #1<29\large\else
  \ifnum #1<34\Large\else \ifnum #1<41\LARGE\else
     \huge\fi\fi\fi\fi\fi\fi
  \csname #3\endcsname}%
\else
\gdef\SetFigFont#1#2#3{\begingroup
  \count@#1\relax \ifnum 25<\count@\count@25\fi
  \def\x{\endgroup\@setsize\SetFigFont{#2pt}}%
  \expandafter\x
    \csname \romannumeral\the\count@ pt\expandafter\endcsname
    \csname @\romannumeral\the\count@ pt\endcsname
  \csname #3\endcsname}%
\fi
\fi\endgroup

\begin{document}
\begin{center}
{\bf 2E2 Tutorial Sheet 11}\footnote{Conor
Houghton, {\tt houghton@maths.tcd.ie} and {\tt
http://www.maths.tcd.ie/\char126 houghton/2E2.html}}
\\[1cm]
 22 January 2006
\end{center}
{
\noindent {\bf Useful facts:}\\ 
\begin{itemize}
\item The stationary point, or critical point, is the
point where $y_1'=y_2'=0$, in linear examples where ${\bf y}'=A{\bf
y}$ this happens only at $y_1=y_2=0$. Because ${\bf y}'$ is determined
by ${\bf y}$ the stationary point is the only place lines can
cross. 
\item If you are asked to name the stationary point, you are asked to
say what the behaviour around it it, we have seen so far improper and
proper inward and outward nodes and saddlepoints, next week we will
see spiral and circle nodes. 
\item When drawing the phase diagram, put the eigenvectors in first,
if the eigenvector has a negative eigenvalues it goes inwards, if it
has positive, it goes out. When the eigenvectors are done, it is
simple to add the other trajectories.
\item One eigenvalue positive one negative gives a saddlepoint. Two unequal positive eigenvalues gives an outward improper node, if they are both negative the node is inward, if they are equal the node is proper.
\end{itemize}
\newpage
\noindent{\bf Questions}
\begin{enumerate}
\item (2) For the system
\begin{eqnarray*}
\frac{dy_1}{dt}&=&-3y_1+2y_2\\
\frac{dy_2}{dt}&=&-2y_1+2y_2
\end{eqnarray*}
The solution is
\begin{equation}
{\bf y}=c_1\left(\begin{array}{c}1\\2\end{array}\right)e^t+c_2\left(\begin{array}{c}2\\1\end{array}\right)e^{-2t}
\end{equation}
Sketch the phase diagram and and describe the stationary point.
\vskip .5cm
\item (2) For the system
\begin{eqnarray}
\frac{dy_1}{dt}&=&3y_1+y_2\\
\frac{dy_2}{dt}&=&y_1+3y_2
\end{eqnarray}
The solution is 
\begin{equation}{\bf y}=\left(\begin{array}{cc}y_1\\y_2\end{array}\right)=c_1\left(\begin{array}{cc}1\\1\end{array}\right)e^{4t}+c_2\left(\begin{array}{cc}-1\\1\end{array}\right)e^{2t}.\end{equation}
Sketch the phase diagram and and describe the stationary point.
\vskip .5cm
\item (4) Find the general solutions for the system
\begin{eqnarray}
\frac{dy_1}{dt}&=&2y_1-y_2\\
\frac{dy_2}{dt}&=&-4y_2
\end{eqnarray}
Sketch the phase diagram and and describe the stationary point.
\end{enumerate}
}
\end{document}



