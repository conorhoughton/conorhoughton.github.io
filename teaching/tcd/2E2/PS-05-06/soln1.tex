\documentclass[12pt]{article}
\usepackage{a4wide, amsfonts, epsfig}


%skyline stuff
\font\upright=cmu10 scaled\magstep1
\setlength{\unitlength}{0.012500in}
\begingroup\makeatletter\ifx\SetFigFont\undefined
\def\x#1#2#3#4#5#6#7\relax{\def\x{#1#2#3#4#5#6}}%
\expandafter\x\fmtname xxxxxx\relax \def\y{splain}%
\ifx\x\y   % LaTeX or SliTeX?
\gdef\SetFigFont#1#2#3{%
  \ifnum #1<17\tiny\else \ifnum #1<20\small\else
  \ifnum #1<24\normalsize\else \ifnum #1<29\large\else
  \ifnum #1<34\Large\else \ifnum #1<41\LARGE\else
     \huge\fi\fi\fi\fi\fi\fi
  \csname #3\endcsname}%
\else
\gdef\SetFigFont#1#2#3{\begingroup
  \count@#1\relax \ifnum 25<\count@\count@25\fi
  \def\x{\endgroup\@setsize\SetFigFont{#2pt}}%
  \expandafter\x
    \csname \romannumeral\the\count@ pt\expandafter\endcsname
    \csname @\romannumeral\the\count@ pt\endcsname
  \csname #3\endcsname}%
\fi
\fi\endgroup

\begin{document}
\begin{center}
{\bf 2E2 Tutorial Sheet 1, Solutions\footnote{Conor Houghton, {\tt houghton@maths.tcd.ie}, see also {\tt http://www.maths.tcd.ie/\char126 houghton/2E2.html}}}\\[1cm]{} 16 October 2005
\end{center}


\renewcommand{\labelenumi}{\arabic{enumi}.}
\begin{enumerate}


\item (1) Using the linearity of the Laplace transform, calculate the
Laplace transform of
\begin{equation}
f(t)=2-\frac{t}{2}
\end{equation}

\noindent\textit{Solution:} So, split it up using linearity
\begin{eqnarray}
{\cal L}\left(2-\frac{t}{2}\right)&=&2{\cal L}(1)-\frac{1}{2}{\cal L}(t)\cr
&=&\frac{2}{s}-\frac{1}{2s^2}
\end{eqnarray}

\item (1) Using the linearity of the Laplace transform, calculate the
Laplace transform of
\begin{equation}
f(t)=2e^{2t}+3t+4e^{-4t}
\end{equation}

\noindent\textit{Solution:} So, split it up using linearity
\begin{eqnarray}
{\cal L}\left(2e^{2t}+3t+4e^{-4t}\right)&=&2{\cal L}\left(e^{2t}\right)+3{\cal L}(t)+4{\cal L}\left(e^{-4t}\right)\cr
&=&\frac{2}{s-2}+\frac{3}{s^2}+\frac{4}{s+4}
\end{eqnarray}

\item (2)
The hyperbolic sine is defined as
\begin{equation}
\sinh{x}=\frac{e^x-e^{-x}}{2}
\end{equation}
using the linearity of the Laplace transform, show that
\begin{equation}
{\cal L}(\sinh{at})=\frac{a}{s^2-a^2}
\end{equation}

\noindent\textit{Solution:} Well, just write it out
\begin{eqnarray}
{\cal L}(\sinh(at))&=&{\cal L}\left(\frac{e^{at}-e^{-at}}{2}   \right)=\frac{1}{2}{\cal L}(e^{at})-\frac{1}{2}{\cal L}(e^{-at})\cr
&=&\frac{1}{2}\frac{1}{s-a}-\frac{1}{2}\frac{1}{s+a}\cr
&=&\frac{1}{2}\frac{s+a-(s-a)}{s^2-a^2}=\frac{a}{s^2-a^2}
\end{eqnarray}

\item (2)
Using the shift theorem find the Laplace transform of
\[
f(t)=e^{2t}t^2
\]

\noindent\textit{Solution:}  Recall the first shift theorem says
\begin{equation}
{\cal L}\left(e^{-at}f(t)\right)=F(s-a)
\end{equation}
where ${\cal L}(f)=F(s)$. Now, we know that
\begin{equation}
{\cal L}\left(t^2\right)=\frac{2!}{s^3}=\frac{2}{s^3}
\end{equation}
so, by the shift theorem
\begin{equation}
{\cal L}\left(e^{2t}t^2\right)=\frac{2}{(s-2)^3}
\end{equation}

\item (2)
Using the formula for the Laplace transform of the differential find ${\cal L}(f')$ where $f=t^2$, check your answer by differentiating $f$ directly and then working out its Laplace transform.

\noindent\textit{Solution:} Well 
\begin{equation}
{\cal L}(f)={\cal L}(t^2)=\frac{2}{s^3}
\end{equation}
and $f(0)=0$ so
\begin{equation}
{\cal L}(f')=s{\cal L}(f)-f(0)=\frac{2}{s^2}
\end{equation}
Doing it by differenciating first, we have $f'=2t$ so
\begin{equation}
{\cal L}(f')={\cal L}(2t)=\frac{2}{s^2}
\end{equation}
As expected, this is same answer.



\end{enumerate}

\vfill

\noindent

\end{document}
