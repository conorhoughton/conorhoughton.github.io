 \documentclass[12pt]{article}
\usepackage{a4wide, amsfonts, epsfig}
\newcommand{\soln}{\noindent\textit{Solution:}}

%skyline stuff
\font\upright=cmu10 scaled\magstep1
\setlength{\unitlength}{0.012500in}
\begingroup\makeatletter\ifx\SetFigFont\undefined
\def\x#1#2#3#4#5#6#7\relax{\def\x{#1#2#3#4#5#6}}%
\expandafter\x\fmtname xxxxxx\relax \def\y{splain}%
\ifx\x\y   % LaTeX or SliTeX?
\gdef\SetFigFont#1#2#3{%
  \ifnum #1<17\tiny\else \ifnum #1<20\small\else
  \ifnum #1<24\normalsize\else \ifnum #1<29\large\else
  \ifnum #1<34\Large\else \ifnum #1<41\LARGE\else
     \huge\fi\fi\fi\fi\fi\fi
  \csname #3\endcsname}%
\else
\gdef\SetFigFont#1#2#3{\begingroup
  \count@#1\relax \ifnum 25<\count@\count@25\fi
  \def\x{\endgroup\@setsize\SetFigFont{#2pt}}%
  \expandafter\x
    \csname \romannumeral\the\count@ pt\expandafter\endcsname
    \csname @\romannumeral\the\count@ pt\endcsname
  \csname #3\endcsname}%
\fi
\fi\endgroup

\begin{document}
\begin{center}

{\bf 2E2 Tutorial Sheet 6 Solutions\footnote{Conor Houghton, {\tt houghton@maths.tcd.ie}, see also {\tt http://www.maths.tcd.ie/\char126 houghton/2E2.html}}\\[1cm] 20 November 2005}
\end{center}


\renewcommand{\labelenumi}{\arabic{enumi}.}

\noindent {\bf Questions:}

\begin{enumerate}

\item (2) Verify the formula ${\cal L}(f*g)={\cal L}(f){\cal L}(g)$ in the case where $f=\exp{(2t)}$ and $g=\exp{(2t)}$. 

\soln So, 
\begin{eqnarray}
e^{2t}*e^{-2t}
&=&\int_0^t e^{2\tau}*e^{2(t-\tau)}d\tau=\int_0^t e^{2t}d\tau\cr
&=&e^{2t}\int_0^td\tau=te^{2t}
\end{eqnarray}
Now, this means
\begin{equation}
{\cal L}\left(e^{2t}*e^{2t}\right)={\cal L}\left(te^{2t}\right)=\frac{1}{(s-2)^2}
\end{equation}
by the shift theorem. Doing it from the formula for the Laplace transform gives
\begin{equation}
{\cal L}\left(e^{2t}*e^{2t}\right)=\left[{\cal L}\left(e^{2t}\right)\right]^2=\frac{1}{(s-2)^2}
\end{equation}


\item (3) Find the convolution $(f*g)(t)$ when $f(t) = t$, $g(t) = e^{2t}$ ($t
\geq 0$).

\soln From the definition of convolutions
\begin{eqnarray*}
(f*g)(t) &=& \int_0^t f(\tau ) g(t-\tau ) \, d\tau 
= \int_0^t \tau  e^{2(t-\tau )} \, d\tau\\ 
&=& \int_0^t \tau  e^{2t} e^{-2\tau } \, d\tau 
= e^{2t} \int_0^t \tau  e^{-2\tau } \, d\tau \\
\mbox{Use integration by parts with}
&& u = \tau , \quad dv = e^{-2\tau } \, d\tau \\
&& du = d\tau , \quad v = -\frac{1}{2} e^{-2\tau }\\
&=& e^{2t} \int_0^t u \, dv
= e^{2t} \left( \left[ uv \right]_0^t - \int_0^t v \, du \right)\\
&=& e^{2t} \left( \left[ - \frac{\tau }{2}e^{-2\tau } \right]_0^t - \int_0^t
-\frac{1}{2} e^{-2\tau } \, d\tau  \right)\\
&=& e^{2t} \left( - \frac{t}{2} e^{-2t} + 0 + \frac{1}{2} \int_0^t
e^{-2\tau } \, d\tau  \right)\\
&=& - \frac{t}{2} + \frac{e^{2t}}{2} \left[ -\frac{1}{2} e^{-2\tau }
\right]_0^t\\
&=& - \frac{t}{2} + \frac{e^{2t}}{2} \left( -\frac{1}{2} e^{-2t} +
\frac{1}{2} \right)\\
&=& - \frac{t}{2} - \frac{1}{4} + \frac{1}{4} e^{2t}
\end{eqnarray*}

\item (3) Use the convolution theorem to find the function $f(t)$ with
\begin{equation}
\mathcal{L}(f) = \frac{1}{s^2(s-4)}.
\end{equation}

\soln We know $\mathcal{L}(t)) = \frac{1}{s^2}$ and
$\mathcal{L}(e^{4t}) = \frac{1}{s-4}$. From the convolution
theorem, we see
\[
\mathcal{L}(f) = \frac{1}{s^2(s-4)} = \mathcal{L}(t)
\mathcal{L}(e^{4t}) = \mathcal{L}(t * e^{4t}) 
\]
so that $f(t)$ is the convolution $t * e^{4t}$.
\begin{eqnarray*}
f(t) 
&=& \int_0^t \tau  e^{4(t-\tau )} \, d\tau \\
&=& \int_0^t \tau  e^{4t} e^{-4\tau } \, d\tau 
= e^{4t} \int_0^t \tau  e^{-4\tau } \, d\tau \\
\mbox{Use integration by parts with}
&& U = \tau , \quad dV = e^{-4\tau } \, d\tau \\
&& dU = d\tau , \quad V = -\frac{1}{4} e^{-4\tau }\\
&=& e^{4t} \int_0^t U \, dV
= e^{4t} \left( \left[ UV \right]_0^t - \int_0^t V \, dU \right)\\
&=& e^{4t} \left( \left[ - \frac{\tau }{4}e^{-4\tau } \right]_0^t - \int_0^t
-\frac{1}{4} e^{-4\tau } \, d\tau  \right)\\
&=& e^{4t} \left( - \frac{t}{4} e^{-4t} + 0 + \frac{1}{4} \int_0^t
e^{-4\tau } \, d\tau  \right)\\
&=& - \frac{t}{4} + \frac{e^{4t}}{4} \left[ -\frac{1}{2} e^{-4\tau }
\right]_0^t\\
&=& - \frac{t}{4} + \frac{e^{4t}}{4} \left( -\frac{1}{4} e^{-4t} +
\frac{1}{4} \right)\\
&=& - \frac{t}{4} - \frac{1}{16} + \frac{1}{16} e^{4t}
\end{eqnarray*}

\end{enumerate}

\vfill


\end{document}
