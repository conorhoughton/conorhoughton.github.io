 \documentclass[12pt]{article}
\usepackage{a4wide, amsfonts, epsfig}


%skyline stuff
\font\upright=cmu10 scaled\magstep1
\setlength{\unitlength}{0.012500in}
\begingroup\makeatletter\ifx\SetFigFont\undefined
\def\x#1#2#3#4#5#6#7\relax{\def\x{#1#2#3#4#5#6}}%
\expandafter\x\fmtname xxxxxx\relax \def\y{splain}%
\ifx\x\y   % LaTeX or SliTeX?
\gdef\SetFigFont#1#2#3{%
  \ifnum #1<17\tiny\else \ifnum #1<20\small\else
  \ifnum #1<24\normalsize\else \ifnum #1<29\large\else
  \ifnum #1<34\Large\else \ifnum #1<41\LARGE\else
     \huge\fi\fi\fi\fi\fi\fi
  \csname #3\endcsname}%
\else
\gdef\SetFigFont#1#2#3{\begingroup
  \count@#1\relax \ifnum 25<\count@\count@25\fi
  \def\x{\endgroup\@setsize\SetFigFont{#2pt}}%
  \expandafter\x
    \csname \romannumeral\the\count@ pt\expandafter\endcsname
    \csname @\romannumeral\the\count@ pt\endcsname
  \csname #3\endcsname}%
\fi
\fi\endgroup

\begin{document}
\begin{center}

{\bf 2E2 Tutorial Sheet 3 First Term\footnote{Conor Houghton, {\tt houghton@maths.tcd.ie}, see also {\tt http://www.maths.tcd.ie/\char126 houghton/2E2.html}}\\[1cm] 1 November 2005}
\end{center}


\renewcommand{\labelenumi}{\arabic{enumi}.}
\noindent {\bf Useful facts:}
\begin{itemize}
\item Laplace transform of differenciated functions: if ${\cal L}[f(t)]=F(s)$ then
\begin{equation}
{\cal L}(f')=sF-f(0)
\end{equation}
and
\begin{equation}
{\cal L}(f'')=s^2F-sf(0)-f'(0)
\end{equation}                                

\item If there a repeated factor in the fraction the partial fraction expansion
looks like:
\begin{equation}
\frac{1}{(s-a)^2(s-b)}=\frac{A}{s-a}+\frac{B}{(s-a)^2}+\frac{C}{s-b}
\end{equation}

\item ${\cal L}(f)=F(s)$ then  ${\cal L}(e^{at}f)=F(s-a)$, in particular 
\begin{equation}
{\cal L}(e^{at}t)=\frac{1}{(s-a)^2},
\end{equation}

\item The Heaviside function $H_a(t)$ is zero for $t<a$ and one for $t\ge a$, it Laplace transform is given by
\begin{equation}
{\cal L}[H_a(t)]=\frac{e^{-as}}{s}
\end{equation}

\item The shift theorem: if ${\cal L}(f)=F(s)$ then
\begin{equation}
{\cal L}[H_a(t)f(t-a)]=e^{-as}F(s)
\end{equation}

\end{itemize}

\newpage

\noindent {\bf Questions}
%made 1 easier by setting a=1 and 4 easier by giving the rhs in terms of heaviside.

\begin{enumerate}

\item (2)
Using the Laplace transform solve the differential equation
\begin{equation}
f''-2f'+f=0
\end{equation}
with boundary conditions $f'(0)=1$ and $f(0)=0$.

\item (2)
Using the Laplace transform solve the differential equation
\begin{equation}
f''+f'-6f=e^{-3t}
\end{equation}
with boundary conditions $f(0)=f'(0)=0$.


\item (2) Use Laplace transform methods to solve the differential equation
\begin{equation}
f'' + 2 f' - 3 f =
\left\{ \begin{array}{ll}
1, & 0 \leq t < c\\
0, & t \geq c
\end{array}\right.
\end{equation}
subject to the initial conditions $f(0) =f'(0) = 0$. Notice that the right hand side is $1-H_1(t)$.
\vskip 1cm

\item (2) Use Laplace transform methods to solve the differential equation
\begin{equation}
f'' + 2 f' - 3 f =
\left\{ \begin{array}{ll}
0, & 0 \le t < 1\\
1, & 1\le t < 2\\
0, & t\ge 2
\end{array}\right.
\end{equation}
subject to the initial conditions $f(0) =f'(0) = 0$. You should begin by rewriting the right-hand side in terms of the Heaviside function:
\begin{equation}
H_1(t)-H_2(t)=
\left\{ \begin{array}{ll}
0, & 0 \le t < 1\\
1, & 1\le t < 2\\
0, & t\ge 2
\end{array}\right.
\end{equation}
\end{enumerate}

\vfill

\noindent

\end{document}
