 \documentclass[12pt]{article}
\usepackage{a4wide, amsfonts, epsfig}


%skyline stuff
\font\upright=cmu10 scaled\magstep1
\setlength{\unitlength}{0.012500in}
\begingroup\makeatletter\ifx\SetFigFont\undefined
\def\x#1#2#3#4#5#6#7\relax{\def\x{#1#2#3#4#5#6}}%
\expandafter\x\fmtname xxxxxx\relax \def\y{splain}%
\ifx\x\y   % LaTeX or SliTeX?
\gdef\SetFigFont#1#2#3{%
  \ifnum #1<17\tiny\else \ifnum #1<20\small\else
  \ifnum #1<24\normalsize\else \ifnum #1<29\large\else
  \ifnum #1<34\Large\else \ifnum #1<41\LARGE\else
     \huge\fi\fi\fi\fi\fi\fi
  \csname #3\endcsname}%
\else
\gdef\SetFigFont#1#2#3{\begingroup
  \count@#1\relax \ifnum 25<\count@\count@25\fi
  \def\x{\endgroup\@setsize\SetFigFont{#2pt}}%
  \expandafter\x
    \csname \romannumeral\the\count@ pt\expandafter\endcsname
    \csname @\romannumeral\the\count@ pt\endcsname
  \csname #3\endcsname}%
\fi
\fi\endgroup

\begin{document}
\begin{center}

{\bf 2E2 Tutorial Sheet 4 First Term\footnote{Conor Houghton, {\tt houghton@maths.tcd.ie}, see also {\tt http://www.maths.tcd.ie/\char126 houghton/2E2.html}}\\[1cm] 6 November 2005}
\end{center}


\renewcommand{\labelenumi}{\arabic{enumi}.}
\noindent {\bf Useful facts:}
\begin{itemize}

\item The Heaviside function $H_a(t)$ is zero for $t<a$ and one for $t\ge a$, it Laplace transform is given by
\begin{equation}
{\cal L}[H_a(t)]=\frac{e^{-as}}{s}
\end{equation}

\item The Dirac delta function $\delta(t-a)$ is zero everywhere except
$t=a$ where it is infinite. It can be thought of as the $b$ goes to
zero limit of 
\begin{equation}
\delta_b(t-a)=\frac{1}{b}\left(H_a(t)-H_{a+b}(t)\right)
\end{equation}

\item A delta function evaluates an integral:
\begin{equation}
\int_0^\infty \delta(t-a)f(t)dt=f(a)
\end{equation}

\item The Laplace transform: ${\cal L}(\delta(t-a))=e^{-as}$.

\item The shift theorem: if ${\cal L}(f)=F(s)$ then
\begin{equation}
{\cal L}[H_a(t)f(t-a)]=e^{-as}F(s)
\end{equation}

\item The formula for complex exponentials:
\begin{eqnarray}
e^{i\theta}&=&\cos{\theta}+i\sin{\theta}\cr
e^{-i\theta}&=&\cos{\theta}-i\sin{\theta}
\end{eqnarray}

\item Remember $e^{a+b}=e^a e^b$ so, 
\begin{eqnarray}
e^{a+ib}&=&(\cos{b}+i\sin{b})e^{a}\cr
e^{a-ib}&=&(\cos{b}-i\sin{b})e^{a}
\end{eqnarray}

\end{itemize}

\newpage

\noindent {\bf Questions}

\begin{enumerate}

\item (4) Use Laplace transform methods to solve the differential equation
\begin{equation}
f'' + 2 f' - 3 f = \delta(t-1)
\end{equation}
subject to the initial conditions $f(0) = 0$, $f'(0) = 1$. 

\item (4)
Using the Laplace transform solve the differential equation
\begin{equation}
f''+6f'+13f=0
\end{equation}
with boundary conditions $f(0)=0$ and $f'(0)=1$ and get your answer into a real form. 


\end{enumerate}

\vfill


\end{document}
