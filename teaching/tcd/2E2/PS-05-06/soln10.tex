\documentclass[12pt]{article}
\usepackage{a4wide, amsfonts, epsfig}
\newcommand\soln{\noindent\textit{Solution:} }


%skyline stuff
\font\upright=cmu10 scaled\magstep1
\setlength{\unitlength}{0.012500in}
\begingroup\makeatletter\ifx\SetFigFont\undefined
\def\x#1#2#3#4#5#6#7\relax{\def\x{#1#2#3#4#5#6}}%
\expandafter\x\fmtname xxxxxx\relax \def\y{splain}%
\ifx\x\y   % LaTeX or SliTeX?
\gdef\SetFigFont#1#2#3{%
  \ifnum #1<17\tiny\else \ifnum #1<20\small\else
  \ifnum #1<24\normalsize\else \ifnum #1<29\large\else
  \ifnum #1<34\Large\else \ifnum #1<41\LARGE\else
     \huge\fi\fi\fi\fi\fi\fi
  \csname #3\endcsname}%
\else
\gdef\SetFigFont#1#2#3{\begingroup
  \count@#1\relax \ifnum 25<\count@\count@25\fi
  \def\x{\endgroup\@setsize\SetFigFont{#2pt}}%
  \expandafter\x
    \csname \romannumeral\the\count@ pt\expandafter\endcsname
    \csname @\romannumeral\the\count@ pt\endcsname
  \csname #3\endcsname}%
\fi
\fi\endgroup

\begin{document}
\begin{center}
{\bf 2E2 Tutorial Sheet 10 Solutions\footnote{Conor Houghton, {\tt houghton@maths.tcd.ie}, see also {\tt http://www.maths.tcd.ie/\char126 houghton/2E2.html}}}\\[1cm] 15 January 2006
\end{center}
{
{\bf Questions}
\begin{enumerate}
\item (2) Find the general solution for the system
\begin{eqnarray}
\frac{dy_1}{dt}&=&3y_1+y_2\\
\frac{dy_2}{dt}&=&y_1+3y_2
\end{eqnarray}
\vskip .25cm
\soln
The eigenvectors and eigenvalues of
\begin{equation}A=\left(\begin{array}{cc}3&1\\1&3\end{array}\right)\end{equation}
are $\lambda_1=4$ with
\begin{equation}{\bf x}_1=\left(\begin{array}{cc}1\\1\end{array}\right)\end{equation}
and $\lambda_2=2$ with 
\begin{equation}{\bf x}_2=\left(\begin{array}{cc}-1\\1\end{array}\right)\end{equation}
so the general soln is
\begin{equation}{\bf y}=\left(\begin{array}{cc}y_1\\y_2\end{array}\right)=c_1\left(\begin{array}{cc}1\\1\end{array}\right)e^{4t}+c_2\left(\begin{array}{cc}-1\\1\end{array}\right)e^{2t}.\end{equation}
\vskip .25cm
\item (3) Find the solution of the system
\begin{eqnarray}
\frac{dy_1}{dt}&=&3y_1+4y_2\\
\frac{dy_2}{dt}&=&4y_1-3y_2
\end{eqnarray}
with $y_1(0)=2$ and $y_2(0)=-1$.
\vskip .25cm
he characteristic equation is
\begin{equation}
\left|\begin{array}{cc}3-\lambda&4\\4&-3-\lambda\end{array}\right|=0
\end{equation}
so
\begin{equation}
(3-\lambda)(-3-\lambda)-16=0
\end{equation}
or
\begin{equation}
\lambda^2+2\lambda-48-25=0
\end{equation}
Solve this gives us $\lambda=\pm 5$. Taking the $\lambda=5$ first
\begin{equation}
\left(\begin{array}{cc}3&4\\4&-3\end{array}\right)\left(\begin{array}{c}a\\b\end{array}\right)=5\left(\begin{array}{c}a\\b\end{array}\right)
\end{equation}
so the first equation is $3a+4b=5a$ or $a=2b$, the other equation is $4a-3b=5b$ which is also $a=2b$. Taking $a=2$ an eigenvalue $5$ eigenvector is,
\begin{equation}
{\bf x}=\left(\begin{array}{c}2\\1\end{array}\right)
\end{equation}
Taking $\lambda=-5$ next
\begin{equation}
\left(\begin{array}{cc}3&4\\4&-3\end{array}\right)\left(\begin{array}{c}a\\b\end{array}\right)=-5\left(\begin{array}{c}a\\b\end{array}\right)
\end{equation}
so the first equation is $3a+4b=-5a$ or $2a=-b$, the other equation is $4a-3b=-5b$ which is also $2a=-b$. Taking $a=1$ an eigenvalue $-5$ eigenvector is,
\begin{equation}
{\bf x}=\left(\begin{array}{c}1\\-2\end{array}\right)
\end{equation}
\vskip .25cm
Hence, the solution is 
\begin{equation}
{\bf y}=C_1\left(\begin{array}{c}2\\1\end{array}\right)e^{5t}+C_2\left(\begin{array}{c}1\\-2\end{array}\right)e^{-5t}
\end{equation}
or
\begin{eqnarray}
y_1&=&2C_1e^{5t}+C_2e^{-5t}\\
y_2&=&C_1e^{5t}-2C_2e^{-5t}
\end{eqnarray}
So, for $t=0$ we have
\begin{eqnarray}
y_1(0)=2&=&2C_1+C_2\\
y_2(0)=-1&=&C_1-2C_2
\end{eqnarray}
giving $C_1=3/5$ and $C_2=4/5$ so
\begin{eqnarray}
y_1&=&\frac{6}{5}e^{5t}+\frac{4}{5}e^{-5t}\\
y_2&=&\frac{3}{5}e^{5t}-\frac{8}{5}e^{-5t}
\end{eqnarray}
\vskip .25cm
\begin{figure}[bht]
\begin{center}
$$
\begin{array}{c}
\begin{picture}(205,100)(55,615)
\thinlines
\put(  55,685){\makebox(0,0)[lb]{\smash{\SetFigFont{12}{14.4}{rm}Tank 1}}}
\put( 205,685){\makebox(0,0)[lb]{\smash{\SetFigFont{12}{14.4}{rm}Tank 2}}}
\put( 120,706){\makebox(0,0)[lb]{\smash{\SetFigFont{12}{14.4}{rm}$\frac{1}{2}m^3s^{-1}$}}}
\put( 118,547){\makebox(0,0)[lb]{\smash{\SetFigFont{12}{14.4}{rm}$\frac{1}{2}m^3s^{-1}$}}}
\put( 47,627){\makebox(0,0)[lb]{\smash{\SetFigFont{12}{14.4}{rm}$V_1=5m^3$}
}}
\put( 200,627){\makebox(0,0)[lb]{\smash{\SetFigFont{12}{14.4}{rm}$V_2=7m^3$}}}
\end{picture}\end{array}
$$
\vskip -3.5cm
\leavevmode
\epsfxsize=8cm\epsffile{tanks.eps}
\end{center}
\caption{Two containers with flow between them.}
\label{twosphtri}
\end{figure}
\vskip .25cm                  
\item (3) As illustrated in Fig. 1, two large containers are connected
and American style sandwich spead is pumped between them at a rate of
$1/2m^3s^{-1}$. One container has volume $5m^3$, the other
$7m^3$. Both are full of spread. Initially the smaller container
contains pure jam, the second container has $5m^3$ of jam and $2m^3$
of peanut butter. Assume perfect mixing and so on.  \\ (i) Write down
the differential equation for $y_1(t)$ and $y_2(t)$, the amount of
peanut butter in the first and second container.\\ (ii) Solve it to
find $y_1(t)$ and $y_2(t)$ explicitly. \\(iii) Use the initial data to find the values of the constants in the solution.
\vskip .25cm
\soln Well if there is
$y_1$ peanut butter in the small container then the concentration of
the spead in the small container is $y_1/5$ and so $y_1/10$
is flowing out per second. In the same way $y_2/7$ is the
concentration of peanut butter in the second tank and so $y_2/14$ per
second is going from the large tank to the small one. This means the
equations are
\begin{eqnarray}
y_1'&=&-\frac{1}{10}y_1+\frac{1}{14}y_2\\
y_2'&=&\frac{1}{10}y_1-\frac{1}{14}y_2
\end{eqnarray}
This equation can be rewritten
\begin{equation}
{\bf y}'=\left(\begin{array}{cc}-\frac{1}{10}&\frac{1}{14}\\
\frac{1}{10}&-\frac{1}{14}\end{array}\right){\bf y}
\end{equation}
We work out the eigenvalues
\begin{eqnarray}
\left|\begin{array}{cc}-\frac{1}{10}-\lambda&\frac{1}{14}\\
\frac{1}{10}&-\frac{1}{14}-\lambda\end{array}\right|&=&\left(\frac{1}{10}+\lambda\right)\left(\frac{1}{14}+\lambda\right)-\frac{1}{140}\\
&=&\lambda^2+\frac{6}{35}\lambda=0
\end{eqnarray}
This means that there are two eigenvalues, $\lambda_1=0$ and
$\lambda_2=-6/35$. The corresponding eigenvectors are given by
\begin{equation}
\left(\begin{array}{cc}-\frac{1}{10}&\frac{1}{14}\\
\frac{1}{10}&-\frac{1}{14}\end{array}\right)\left(\begin{array}{c}a\\b\end{array}\right)=0
\end{equation}
which has solutions of the form $a=10$ and $b=14$ for $\lambda_1$ and 
\begin{equation}
\left(\begin{array}{cc}-\frac{1}{10}&\frac{1}{14}\\
\frac{1}{10}&-\frac{1}{14}\end{array}\right)\left(\begin{array}{c}a\\b\end{array}\right)=-\frac{6}{35}\left(\begin{array}{c}a\\b\end{array}\right)
\end{equation}
for $\lambda_2$. This has solution $a=-1$ and $b=1$. Thus, the general
solution is
\begin{equation}
\left(\begin{array}{c}y_1\\y_2\end{array}\right)=c_1\left(\begin{array}{c}10\\14\end{array}\right)+c_2\left(\begin{array}{c}-1\\1\end{array}\right)e^{-\frac{6}{35}t}
\end{equation}
\vskip .25cm
For part (iii), matching with $y_1(0)=0$ and $y_2(0)=2$, this gives $c_1=1/12$ and $c_2=5/6$ and hence
\begin{equation}
\left(\begin{array}{c}y_1\\y_2\end{array}\right)=\frac{1}{12}\left(\begin{array}{c}10\\14\end{array}\right)+\frac{5}{6}\left(\begin{array}{c}-1\\1\end{array}\right)e^{-\frac{6}{35}t}
\end{equation}
\vskip .25cm
By the way, clearly the exponetially decaying part
goes away with time so that
\begin{equation}
\lim_{t\rightarrow \infty}{{\bf y}}=\left(\begin{array}{c}\frac{5}{6}\\ \frac{7}{6}\end{array}\right)
\end{equation}
\end{enumerate}
}
\end{document}
