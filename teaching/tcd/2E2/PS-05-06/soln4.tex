 \documentclass[12pt]{article}
\usepackage{a4wide, amsfonts, epsfig}
\newcommand{\soln}{\noindent\textit{Solution:}}


%skyline stuff
\font\upright=cmu10 scaled\magstep1
\setlength{\unitlength}{0.012500in}
\begingroup\makeatletter\ifx\SetFigFont\undefined
\def\x#1#2#3#4#5#6#7\relax{\def\x{#1#2#3#4#5#6}}%
\expandafter\x\fmtname xxxxxx\relax \def\y{splain}%
\ifx\x\y   % LaTeX or SliTeX?
\gdef\SetFigFont#1#2#3{%
  \ifnum #1<17\tiny\else \ifnum #1<20\small\else
  \ifnum #1<24\normalsize\else \ifnum #1<29\large\else
  \ifnum #1<34\Large\else \ifnum #1<41\LARGE\else
     \huge\fi\fi\fi\fi\fi\fi
  \csname #3\endcsname}%
\else
\gdef\SetFigFont#1#2#3{\begingroup
  \count@#1\relax \ifnum 25<\count@\count@25\fi
  \def\x{\endgroup\@setsize\SetFigFont{#2pt}}%
  \expandafter\x
    \csname \romannumeral\the\count@ pt\expandafter\endcsname
    \csname @\romannumeral\the\count@ pt\endcsname
  \csname #3\endcsname}%
\fi
\fi\endgroup

\begin{document}
\begin{center}

{\bf 2E2 Tutorial Sheet 4 Solutions\footnote{Conor Houghton, {\tt houghton@maths.tcd.ie}, see also {\tt http://www.maths.tcd.ie/\char126 houghton/2E2.html}}\\[1cm] 6 November 2005}
\end{center}

\noindent {\bf Questions}

\begin{enumerate}

\item (4) Use Laplace transform methods to solve the differential equation
\begin{equation}
f'' + 2 f' - 3 f = \delta(t-1)
\end{equation}
subject to the initial conditions $f(0) = 0$, $f'(0) = 1$. 

\soln soln The only thing that is unusual is that there is a delta function. We take the Laplace transform using
\begin{equation}
{\cal L}(\delta(t-a))=e^{-as}
\end{equation}
hence
\begin{equation}
(s^2+2s-3)F-1=e^{-s}
\end{equation}
Now, if we do partial fractions on $1/(s^2+2s-3)$ we get
\begin{equation}
\frac{1}{s^2+2s-3}=-\frac{1}{4(s+3)}+\frac{1}{4(s-1)}
\end{equation}
Hence
\begin{equation}
F=\left(-\frac{1}{4(s+3)}+\frac{1}{4(s-1)}\right)\left(1+e^{-s}\right)
\end{equation}
Since 
\begin{equation}
{\cal L}\left(-\frac{1}{4}e^{-3t}+\frac{1}{4}e^t\right)=-\frac{1}{4(s+3)}+\frac{1}{4(s-1)}
\end{equation}
then, by the third shift theorem we have
\begin{equation}
f=\left(-\frac{1}{4}e^{-3t}+\frac{1}{4}e^{t}\right)+H_1(t)\left(-\frac{1}{4}e^{-3t+3}+\frac{1}{4}e^{t-1}\right)
\end{equation}

\item (4)
Using the Laplace transform solve the differential equation
\begin{equation}
f''+6f'+13f=0
\end{equation}
with boundary conditions $f(0)=0$ and $f'(0)=1$ and get your answer into a real form. 

\soln
So, taking the Laplace transform of the
equation we get,
\begin{equation}
s^2F-1+6sF+13F=0
\end{equation}
and, hence,
\begin{equation}
F=\frac{1}{s^2+6s+13}.
\end{equation}
Now, using minus b plus or minus the square root of b squared minus four a c all over two a, we get
\begin{equation}
s^2+6s+13=0
\end{equation}
if 
\begin{equation}
s=\frac{-6\pm\sqrt{36-52}}{2}=-3\pm 2i
\end{equation}
which means
\begin{equation}
s^2+6s+13=(s+3-2i)(s+3+2i)
\end{equation}
Next, we do the partial fraction expansion,
\begin{equation}
\frac{1}{s^2+6s+13}=\frac{A}{s+3-2i}+\frac{B}{s+3+2i}
\end{equation}
and multiplying across we get
\begin{equation}
1=A(s+3+2i)+B(s+3-2i)
\end{equation}
therefore we choose $s=-3+2i$ to get
\begin{equation}
A=\frac{1}{4i}=-\frac{i}{4}
\end{equation}
and $s=-3-2i$ to get
\begin{equation}
B=-\frac{1}{4i}=\frac{i}{4}
\end{equation} 
and so
\begin{equation}
F=-\frac{i}{4}\frac{1}{s+3-2i}+\frac{i}{4}\frac{1}{s+3+2i}.
\end{equation}
If we take the inverse transform
\begin{eqnarray}
f&=&-\frac{i}{4}e^{-(3-2i)t}+\frac{i}{4}e^{-(3+2i)t}\nonumber\\
 &=&\frac{i}{4}e^{-3t}(e^{-2it}-e^{2it})\nonumber\\
 &=&\frac{1}{2}e^{-3t}\sin{2t}
\end{eqnarray}

\end{enumerate}

\vfill


\end{document}
