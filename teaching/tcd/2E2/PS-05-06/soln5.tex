 \documentclass[12pt]{article}
\usepackage{a4wide, amsfonts, epsfig}
\newcommand{\soln}{\noindent\textit{Solution:}}


%skyline stuff
\font\upright=cmu10 scaled\magstep1
\setlength{\unitlength}{0.012500in}
\begingroup\makeatletter\ifx\SetFigFont\undefined
\def\x#1#2#3#4#5#6#7\relax{\def\x{#1#2#3#4#5#6}}%
\expandafter\x\fmtname xxxxxx\relax \def\y{splain}%
\ifx\x\y   % LaTeX or SliTeX?
\gdef\SetFigFont#1#2#3{%
  \ifnum #1<17\tiny\else \ifnum #1<20\small\else
  \ifnum #1<24\normalsize\else \ifnum #1<29\large\else
  \ifnum #1<34\Large\else \ifnum #1<41\LARGE\else
     \huge\fi\fi\fi\fi\fi\fi
  \csname #3\endcsname}%
\else
\gdef\SetFigFont#1#2#3{\begingroup
  \count@#1\relax \ifnum 25<\count@\count@25\fi
  \def\x{\endgroup\@setsize\SetFigFont{#2pt}}%
  \expandafter\x
    \csname \romannumeral\the\count@ pt\expandafter\endcsname
    \csname @\romannumeral\the\count@ pt\endcsname
  \csname #3\endcsname}%
\fi
\fi\endgroup

\begin{document}
\begin{center}

{\bf 2E2 Tutorial Sheet 5 Solutions\footnote{Conor Houghton, {\tt houghton@maths.tcd.ie}, see also {\tt http://www.maths.tcd.ie/\char126 houghton/2E2.html}}\\[1cm] 12 November 2005}
\end{center}


\renewcommand{\labelenumi}{\arabic{enumi}.}


\noindent {\bf Questions}

\begin{enumerate}

\item (2) Solve, using Laplace transforms, 
\begin{equation}
f''+4f=1
\end{equation}
with $f(0)=f'(0)=0$.

\soln First take the Laplace transfrom of each side:
\begin{equation}
s^2F+4F=\frac{1}{s}
\end{equation}
and so 
\begin{equation}
F=\frac{1}{s(s+2i)(s-2i)}
\end{equation}
Now lets do partial fractions
\begin{equation}
\frac{1}{s(s+2i)(s-2i)}=\frac{A}{s}+\frac{B}{s+2i}+\frac{C}{s-2i}
\end{equation}
giving
\begin{equation}
1=A(s-2i)(s+2i)+Bs(s-2i)+Cs(s+2i)
\end{equation}
hence, choosing $s=0$ gives $A=1/4$, $s=-2i$ gives
\begin{equation}
1=B(-2i)(-4i)=-8B
\end{equation}
hence $B=-1/8$. $s=2i$ gives $C=-1/8$ also. Now
\begin{equation}
F=\frac{1}{4}\frac{1}{s}-\frac{1}{8}\frac{1}{s+2i}-\frac{1}{8}\frac{1}{s-2i}
\end{equation}
so 
\begin{equation}
f=\frac{1}{4}-\frac{1}{8}\left(e^{-2it}+e^{2it}\right)
\end{equation}
then, using 
\begin{equation}
\cos{2t}=\frac{e^{2it}+e^{-2it}}{2}
\end{equation}
we conclude
\begin{equation}
f=\frac{1}{4}-\frac{1}{4}\cos{2t}
\end{equation}

\item (3)
Using the Laplace transform solve the differential equation
\begin{equation}
f''+6f'+13f=e^t
\end{equation}
with boundary conditions $f(0)=0$ and $f'(0)=0$.

\soln
Taking the Laplace transform of the equation gives
\begin{equation}
s^2F+6sF+13F=\frac{1}{s-1}
\end{equation}
so that
\begin{equation}
F=\frac{1}{(s-1)(s+3+2i)(s+3-2i)}.
\end{equation}
We write
\begin{equation}
\frac{1}{(s-1)(s+3+2i)(s+3-2i)}=\frac{A}{s+3-2i}+\frac{B}{s+3+2i}+\frac{C}{s-1}
\end{equation}
giving
\begin{equation}
1=A(s-1)(s+3+2i)+B(s-1)(s+3-2i)+C(s+3-2i)(s+3+2i).
\end{equation}
$s=-3+2i$ gives
\begin{equation}
1=A(-4+2i)(4i)=A(-8-16i)
\end{equation}
so
\begin{equation}
A=-\frac{1}{8+16i}=-\frac{1}{8+16i}\frac{8-16i}{8-16i}=-\frac{1-2i}{40}
\end{equation}
In the same way, $s=-3-2i$ leads to 
\begin{equation}
B=-\frac{1+2i}{40}
\end{equation}
and, finally, $s=1$ gives
\begin{equation}
C=\frac{1}{20}.
\end{equation}
\vskip 1cm
Putting all this together we get
\begin{equation}
F=-\frac{1-2i}{40}\frac{1}{s+3-2i}-\frac{1+2i}{40}\frac{1}{s+3+2i}+\frac{1}{20}\frac{1}{s-1}
\end{equation}
and so 
\begin{eqnarray}
f&=&-\frac{1-2i}{40}e^{-(3-2i)t}-\frac{1+2i}{40}e^{-(3+2i)t}+\frac{1}{20}e^t\nonumber\\
&=&-\frac{1}{40}e^{-3t}\left[(1-2i)e^{2it}+(1+2i)e^{-2it}\right]+\frac{1}{20}e^t
\end{eqnarray}
We then substitute in 
\begin{eqnarray}
e^{2it}&=&\cos{2t}+i\sin{2t}\nonumber\\
e^{-2it}&=&\cos{2t}-i\sin{2t}
\end{eqnarray}
to end up with 
\begin{equation}
f=-\frac{1}{20}e^{-3t}[2\sin{2t}+\cos{2t}]+\frac{1}{20}e^t
\end{equation}

\item (3) Use the formula for the Laplace transform of a periodic function
to find the Laplace transform of  a half-rectified wave
\begin{equation}
f(t)=\left\{\begin{array}{ll}\sin{t}&\sin{t}>0\\0&\sin{t}\le 0\end{array}\right.\end{equation}
\begin{center}
\epsfig{file=rectifiedwave.epsi,width=15cm}
\end{center}
This is the form a AC current has after going through a diode and is a periodic function with period $2\pi$.
\soln
So we substitute this into the formula
\begin{equation}
{\cal L}(f)=\frac{1}{1-e^{-2\pi s}}\int_0^{2\pi} f(t)e^{-st}dt=\frac{1}{1-e^{-2\pi s}}\int_0^\pi \sin{t}e^{-st}dt
\end{equation}
We need to do the integral. There are two obvious ways, the first is
to split the sine into exponentials
\begin{eqnarray}
\int_0^\pi \sin{t}e^{-st}dt&=&\frac{1}{2i}\left(\int_0^\pi e^{(i-s)t}dt-\int_0^\pi e^{-(i+s)t}dt\right)\nonumber\\
&=&
\frac{1}{2i}\left[\frac{1}{i-s}\left( e^{(i-s)\pi}-1\right)+\frac{1}{i+s}\left( e^{-(i+s)\pi}-1\right)\right]
\end{eqnarray}
Now, we use 
\begin{equation}
e^{i\pi}=e^{-i\pi}=-1
\end{equation}
and
\begin{eqnarray}
\frac{1}{i-s}&=&\frac{1}{i-s}\frac{-i-s}{-i-s}=-\frac{s+i}{s^2+1}\nonumber\\
\frac{1}{i+s}&=&\frac{1}{i+s}\frac{-i+s}{-i+s}=\frac{s-i}{s^2+1}
\end{eqnarray}
to get
\begin{equation}\label{ans}
\int_0^\pi \sin{t}e^{-st}dt=\frac{1+e^{-s\pi}}{1+s^2}
\end{equation}
or 
\begin{equation}
{\cal L}(f)=\frac{1}{s^2+1}\frac{1+e^{-s\pi}}{1-e^{-2s\pi}}=\frac{1}{s^2+1}\frac{1}{1-e^{-s\pi}}
\end{equation}
where the final equality uses 
\begin{equation}
1-e^{-2s\pi}=\left(1-e^{-s\pi}\right)\left(1+e^{-s\pi}\right)
\end{equation}
\vskip 1cm
The other way to do the integral is to integrate by parts. Briefly, write
\begin{eqnarray}
I=\int_0^\pi \sin{t}e^{-st}dt&=&-\frac{1}{s}\int_0^\pi\cos{t}e^{-st}dt\nonumber\\
&=&-\frac{1}{s}\left[-\frac{1}{s}\left(e^{-\pi s}+1\right)+\frac{1}{s}I\right]
\end{eqnarray}
and solve for $I$ to get the answer given at (\ref{ans}) above.


\end{enumerate}

\vfill


\end{document}
