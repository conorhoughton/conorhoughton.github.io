\documentclass[12pt]{article}
\usepackage{a4wide, amsfonts, epsfig}


%skyline stuff
\font\upright=cmu10 scaled\magstep1
\setlength{\unitlength}{0.012500in}
\begingroup\makeatletter\ifx\SetFigFont\undefined
\def\x#1#2#3#4#5#6#7\relax{\def\x{#1#2#3#4#5#6}}%
\expandafter\x\fmtname xxxxxx\relax \def\y{splain}%
\ifx\x\y   % LaTeX or SliTeX?
\gdef\SetFigFont#1#2#3{%
  \ifnum #1<17\tiny\else \ifnum #1<20\small\else
  \ifnum #1<24\normalsize\else \ifnum #1<29\large\else
  \ifnum #1<34\Large\else \ifnum #1<41\LARGE\else
     \huge\fi\fi\fi\fi\fi\fi
  \csname #3\endcsname}%
\else
\gdef\SetFigFont#1#2#3{\begingroup
  \count@#1\relax \ifnum 25<\count@\count@25\fi
  \def\x{\endgroup\@setsize\SetFigFont{#2pt}}%
  \expandafter\x
    \csname \romannumeral\the\count@ pt\expandafter\endcsname
    \csname @\romannumeral\the\count@ pt\endcsname
  \csname #3\endcsname}%
\fi
\fi\endgroup

\begin{document}
\begin{center}
{\bf 2E2 Tutorial Sheet 10\footnote{Conor Houghton, {\tt houghton@maths.tcd.ie}, see also {\tt http://www.maths.tcd.ie/\char126 houghton/2E2.html}}}\\[1cm] 15 January 2006
\end{center}

\noindent {\bf Useful facts:}\\
\begin{itemize}
\item Solving a linear differential equation: for $A$ a $2\times 2$ matrix with eigenvalues $\lambda_1$ and $\lambda_1$ and corresponding eigenvectors ${\bf x}_1$ and ${\bf x}_2$ then if
\begin{equation}
{\bf y}'=A{\bf y}
\end{equation}
the solution is
\begin{equation}
{\bf y}=C_1{\bf x}_1e^{\lambda_1t}+C_2{\bf x}_2e^{\lambda_2t}
\end{equation}
where $C_1$ and $C_2$ are arbitrary constants.
\item If initial condition are given, just set $t=0$ to find $y_1(0)$ and $y_2(0)$ and this should give simultaneous equations for $C_1$ and $C_2$.
\end{itemize}
\newpage
{\bf Questions}
\begin{enumerate}
\item (2) Find the general solution for the system
\begin{eqnarray}
\frac{dy_1}{dt}&=&3y_1+y_2\\
\frac{dy_2}{dt}&=&y_1+3y_2
\end{eqnarray}
\vskip .25cm
\item (3) Find the solution of the system
\begin{eqnarray}
\frac{dy_1}{dt}&=&3y_1+4y_2\\
\frac{dy_2}{dt}&=&4y_1-3y_2
\end{eqnarray}
with $y_1(0)=2$ and $y_2(0)=-1$.
\vskip .25cm
\begin{figure}[bh]
\begin{center}
$$
\begin{array}{c}
\begin{picture}(205,100)(55,615)
\thinlines
\put(  55,685){\makebox(0,0)[lb]{\smash{\SetFigFont{12}{14.4}{rm}Tank 1}}}
\put( 205,685){\makebox(0,0)[lb]{\smash{\SetFigFont{12}{14.4}{rm}Tank 2}}}
\put( 120,706){\makebox(0,0)[lb]{\smash{\SetFigFont{12}{14.4}{rm}$\frac{1}{2}m^3s^{-1}$}}}
\put( 118,547){\makebox(0,0)[lb]{\smash{\SetFigFont{12}{14.4}{rm}$\frac{1}{2}m^3s^{-1}$}}}
\put( 47,627){\makebox(0,0)[lb]{\smash{\SetFigFont{12}{14.4}{rm}$V_1=5m^3$}
}}
\put( 200,627){\makebox(0,0)[lb]{\smash{\SetFigFont{12}{14.4}{rm}$V_2=7m^3$}}}
\end{picture}\end{array}
$$
\vskip -3.5cm
\leavevmode
\epsfxsize=8cm\epsffile{tanks.eps}
\end{center}
\caption{Two containers with flow between them.}
\label{twosphtri}
\end{figure}
\vskip .25cm                  
\item (3) As illustrated in Fig. 1, two large containers are connected
and American style sandwich spead is pumped between them at a rate of
$1/2m^3s^{-1}$. One container has volume $5m^3$, the other
$7m^3$. Both are full of spread. Initially the smaller container
contains pure jam, the second container has $5m^3$ of jam and $2m^3$
of peanut butter. Assume perfect mixing and so on.  \\ (i) Write down
the differential equation for $y_1(t)$ and $y_2(t)$, the amount of
peanut butter in the first and second container.\\ (ii) Solve it to
find $y_1(t)$ and $y_2(t)$ explicitly. \\(iii) Use the initial data to find the values of the constants in the solution.
\end{enumerate}

\end{document}
