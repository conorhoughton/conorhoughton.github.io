\documentclass[12pt]{article}
\usepackage{a4wide, amsfonts, epsfig}
\newcommand\soln{\noindent\textit{Solution:} }


%skyline stuff
\font\upright=cmu10 scaled\magstep1
\setlength{\unitlength}{0.012500in}
\begingroup\makeatletter\ifx\SetFigFont\undefined
\def\x#1#2#3#4#5#6#7\relax{\def\x{#1#2#3#4#5#6}}%
\expandafter\x\fmtname xxxxxx\relax \def\y{splain}%
\ifx\x\y   % LaTeX or SliTeX?
\gdef\SetFigFont#1#2#3{%
  \ifnum #1<17\tiny\else \ifnum #1<20\small\else
  \ifnum #1<24\normalsize\else \ifnum #1<29\large\else
  \ifnum #1<34\Large\else \ifnum #1<41\LARGE\else
     \huge\fi\fi\fi\fi\fi\fi
  \csname #3\endcsname}%
\else
\gdef\SetFigFont#1#2#3{\begingroup
  \count@#1\relax \ifnum 25<\count@\count@25\fi
  \def\x{\endgroup\@setsize\SetFigFont{#2pt}}%
  \expandafter\x
    \csname \romannumeral\the\count@ pt\expandafter\endcsname
    \csname @\romannumeral\the\count@ pt\endcsname
  \csname #3\endcsname}%
\fi
\fi\endgroup

\begin{document}
\begin{center}
{\bf 2E2 Tutorial Sheet 17 Solutions}\footnote{Conor
Houghton, {\tt houghton@maths.tcd.ie} and {\tt
http://www.maths.tcd.ie/\char126 houghton/ 2E2.html}}
\\[1cm]
 5 March 2006
\end{center}
{
\noindent{\bf Questions:}
\begin{enumerate}
\item (2) Assuming the solution of 
\begin{equation}
(1-t)y'+y=0
\end{equation}
 has a series expansion about
$t=0$ work out the recursion relation. Write out the first few terms
and show that the series $a_2=0$ so the series actually terminates to give $y=A(1-t)$ for arbitrary $A$.
\vskip .25cm
\soln  So we begin by writing 
\begin{equation}
y=\sum_{n=0}^{\infty}a_nt^n
\end{equation}
and so by differentiation we get
\begin{equation}\label{yp}
y'=\sum_{n=0}^{\infty}a_nnt^{n-1}
\end{equation}
and hence
\begin{equation}
ty'=\sum_{n=0}^{\infty}a_nnt^n.
\end{equation}
Thus, substituting the differential equation we get
\begin{equation}
\sum_{n=0}^{\infty}a_nnt^{n-1}-\sum_{n=0}^{\infty}a_nnt^n+\sum_{n=0}^{\infty}a_nt^n=0
\end{equation}
In order to make progress we need to rewrite the first of these three series so 
that it is in the form 
\begin{equation}
\sum_{n=0}^{\infty}\mbox{stuff}_nt^n
\end{equation}
so that all three bits in the equation match. Well, let $m=n-1$ in the expression for $y'$, (\ref{yp}), to get
\begin{equation}\label{yp2}
y'=\sum_{m=0}^{\infty}a_{m+1}(m+1)t^m.
\end{equation}
In fact, this looks at first like it gives 
\begin{equation}\label{yp3}
y'=\sum_{m=-1}^{\infty}a_{m+1}(m+1)t^m
\end{equation}
but the $m=-1$ term is zero, so that's fine. Now $m$ is just an index so we 
can rename it $n$, don't get confused, this isn't the original $n$, we just 
want all parts of the equation to look the same. 
\vskip .25cm
In fact, we now have
\begin{equation}
\sum_{n=0}^{\infty}a_{n+1}(n+1)t^n-\sum_{n=0}^{\infty}a_nnt^n+\sum_{n=0}^{\infty}a_nt^n=0
\end{equation}
and we can group this all together to give
\begin{equation}
\sum_{n=0}^{\infty}[a_{n+1}(n+1)+(1-n)a_n]t^n=0.
\end{equation}
The recursion relation is 
\begin{equation}
a_{n+1}=-\left(\frac{1-n}{1+n}\right)a_n
\end{equation}
and this applies to $n$ from zero upwards since that is what appears
in the sum sign.
\vskip .25cm
Starting at $n=0$ we have
\begin{equation}
a_1=-a_0.
\end{equation}
For $n=1$ we get
\begin{equation}
a_2=0
\end{equation}
and the series terminates here because every term is something
multiplied by the one before and so if $a_2$ is zero the rest of the
series is zero. Thus $y=a_0(1-t)$ for arbitrary $a_0$.
\vskip .5cm
\item (2) Assuming the solution of
\begin{equation}
(1-t^2)y'-2ty=0
\end{equation}
has a series expansion about $t=0$, work out the recursion relation.
\vskip .25cm
\soln Once again let 
\begin{equation}
y=\sum_{n=0}^{\infty}a_nt^n
\end{equation}
and, as before, 
\begin{equation}
y'=\sum_{n=0}^{\infty}a_nnt^{n-1}
\end{equation}
so 
\begin{equation}
t^2y'=\sum_{n=0}^{\infty}a_nnt^{n+1}
\end{equation}
and finally
\begin{equation}
ty=\sum_{n=0}^{\infty}a_nt^{n+1}.
\end{equation}
The equation then reads
\begin{equation}
\sum_{n=0}^{\infty}a_nnt^{n-1}-\sum_{n=0}^{\infty}a_nnt^{n+1}-2\sum_{n=0}^{\infty}a_nt^{n+1}.
\end{equation}
\vskip .25cm
Once again, the first term is a problem because it doesn't have the same form 
as the other two. So, take
\begin{equation}
\sum_{n=0}^{\infty}a_nnt^{n-1}
\end{equation}
and put $n-1=m+1$ and, hence, $n=m+2$. When $n=0$, $m=-2$ and when
$n=1$, $m=-1$. Thus
\begin{equation}
\sum_{n=0}^{\infty}a_nnt^{n-1}=\sum_{m=-2}^{\infty}a_{m+2}(m+2)t^{m+1}
\end{equation}
and, once again renaming $m$ as $n$ we get
\begin{equation}
\sum_{n=-2}^{\infty}(n+2)a_{n+2}t^{n+1}-\sum_{n=0}^{\infty}na_nt^{n+1}-2\sum_{n=0}^{\infty}a_nt^{n+1}=0.
\end{equation}
The problem now is with the range that the first sum runs over. The $n=-2$ 
term  is no problem, it is zero, but the $n=-1$ term is $a_1$. Thus, we write
\begin{equation}
\sum_{n=-2}^{\infty}(n+2)a_{n+2}t^{n+1}=a_1+\sum_{n=0}^{\infty}(n+2)a_{n+2}t^{n+1}
\end{equation}
and the equation becomes
\begin{equation}
a_1+\sum_{n=0}^{\infty}(n+2)a_{n+2}t^{n+1}-\sum_{n=0}^{\infty}a_nnt^{n+1}-2\sum_{n=0}^{\infty}a_nt^{n+1}=0.
\end{equation}
Thus
\begin{equation}
a_1+\sum_{n=0}^{\infty}[(n+2)a_{n+2}-na_n-2a_n]t^{n+1}=0.
\end{equation}
Notice that the summand starts with the $t$ term. The recursion relation is 
therefore
\begin{equation}
a_{n+2}=a_n
\end{equation}
with the additional conditions $a_1=0$. Hence, $a_6=a_4=a_2=a_0$, 
$a_5=a_3=a_1=0$ and so on. The first four nonzero terms of the expansion gives
\begin{equation}
y=a_0(1+t^2+t^4+t^6+\ldots).
\end{equation}
\vskip .25cm
\item (2) Assuming the solution of
\begin{equation}
y''-3y'+2y=0
\end{equation}
has a series expansion about
$t=0$, by substitution, work out the recursion relation. If $y(0)=1$ and $y'(0)=
0$ what are the first three non-zero terms
\vskip .25cm
\soln Again
\begin{equation}
y=\sum_{n=0}^{\infty}a_nt^n
\end{equation}
so
\begin{equation}
y'=\sum_{n=0}^{\infty}na_nt^{n-1}
\end{equation}
and
\begin{equation}
y''=\sum_{n=0}^{\infty}n(n-1)a_nt^{n-2}
\end{equation}
Thus,
\begin{equation}
\sum_{n=0}^{\infty}n(n-1)a_nt^{n-2}-3\sum_{n=0}^{\infty}na_nt^{n-1}+2\sum_{n=0}^{\infty}a_nt^n=0
\end{equation}
Again, we want to make each part look the same. As before, changing the index gi
ves
\begin{equation}
y'=\sum_{n=0}^{\infty}na_nt^{n-1}=\sum_{n=0}^{\infty}(n+1)a_{n+1}t^n.
\end{equation}
The same thing can be done with the $y''$: let $m=n-2$ to get
\begin{equation}
\sum_{n=0}^{\infty}n(n-1)a_nt^{n-2}=\sum_{m=-2}^{\infty}(m+1)(m+2)a_{m+2}t^m
\end{equation}
and the $m=-2$ and $m=-1$ terms are both zero, so, renaming the $m$ as
$n$ we get
\begin{equation}
\sum_{n=0}^{\infty}(n+1)(n+2)a_{n+2}t^n-3\sum_{n=0}^{\infty}(n+1)a_{n+1}t^n+2\sum_{n=0}^{\infty}a_nt^n=0
\end{equation}
and this gives
\begin{equation}
\sum_{n=0}^{\infty}[(n+1)(n+2)a_{n+2}-3(n+1)a_{n+1}+2a_n]t^n=0.
\end{equation}
The recursion relation is 
\begin{equation}
(n+1)(n+2)a_{n+2}-3(n+1)a_{n+1}+2a_n=0.
\end{equation}
\vskip .25cm
Now apply the initial conditions, $y(0)=1$ implies that $a_0=1$, $y'(0)=0$ implies $a_1=0$. For $n=0$ the recursion relation gives
\begin{equation}
2a_2-3a_1+2a_0=0
\end{equation}
and so $a_2=-a_0=-1$. Next $n=1$ gives
\begin{equation}
6a_3-6a_2+2a_1=0
\end{equation}
and so $a_3=a_2=-a_0=-1$. Therefore the first three nonzero terms are
\begin{equation}
y=1-t^2-t^3+\ldots.  \end{equation}
\vskip .5cm
\item (2) Assuming the solution of
\begin{equation}
y''-3t^2y=0
\end{equation}
 has a series expansion about $t=0$ work out the recursion relation and write out the first four non-zero terms if $y(0)=1$ and $y'(0)=1$.
\vskip .25cm
\soln
We substitute
\begin{equation}
y=\sum_{n=0}^\infty{a_nt^n}
\end{equation}
into the equation. This gives
\begin{equation}
\sum_{n=0}^\infty{n(n-1)a_nt^{n-2}}-\sum_{n=0}^\infty{3a_nt^{n+2}}=0
\end{equation}
The problem here is with the powers of $t$. The easiest thing is 
to change everything to the highest power, in this case $n+2$. Hence, put $m+2=n
-2$ in the first sum
\begin{equation}
\sum_{n=0}^\infty{n(n-1)a_nt^{n-2}}=\sum_{m=-4}^\infty{(m+4)(m+3)a_{m+4}t^{m+2}}
.
\end{equation}
and substitute that back into the equation, writing $m$ as $n$:
\begin{equation}
\sum_{n=-4}^\infty{(n+4)(n+3)a_{n+4}t^{n+2}} -\sum_{n=0}^\infty{3a_nt^{n+2}}=0
\end{equation}
and so the problem now is that the ranges are different. We need to take out the
 first few term of the first sum, well, the $n=-4$ and $n=-3$ terms are zero and
 so 
\begin{equation}
\sum_{n=-4}^\infty{(n+4)(n+3)a_{n+4}t^{n+2}}=2a_2+6a_3t+\sum_{n=0}^\infty{(n+4)(
n+3)a_{n+4}t^{n+2}}.
\end{equation}
Now the equation reads
\begin{equation}
2a_2+6a_3t+\sum_{n=0}^\infty{(n+4)(n+3)a_{n+4}t^{n+2}} -\sum_{n=0}^\infty{3a_nt^
{n+2}}=0
\end{equation}
or
\begin{equation}
2a_2+6a_3t+\sum_{n=0}^\infty{\left[(n+4)(n+3)a_{n+4}-3a_n\right]t^{n+2}}=0.
\end{equation}
Thus 
\begin{eqnarray}
a_2&=&0\nonumber\\
a_3&=&0\nonumber\\
a_{n+4}&=&\frac{3}{(n+4)(n+3)}a_n
\end{eqnarray}
where the recursion relation applies for $n=0,1,\ldots$.
Now, $y(0)=1$ implies $a_0=1$ and $y'(0)=1$ implies $a_1=1$, next, with $n=0$, t
he recursion gives
\begin{equation}
a_4=\frac{1}{4}a_0=\frac{1}{4}
\end{equation}
and with $n=1$
\begin{equation}
a_5=\frac{3}{20}a_1=\frac{3}{20}.
\end{equation}
Now since $a_2=a_3=0$ the $n=2$ recusion gives $a_6=0$ and the $n=3$ recursion g
ives $a_7=0$. However, $n=4$ gives
\begin{equation}
a_8=\frac{3}{32}a_4=\frac{3}{128}
\end{equation}
and so 
\begin{equation}
y=1+t+\frac{1}{4}t^4+\frac{3}{20}t^5+\frac{3}{128}t^8+\ldots.
\end{equation}
\vskip 1cm 
\noindent {\bf Aside}. In the above we made all the powers the same as the
highest power, this is usually the easiest thing, but it is just a
matter of convenience. If we had decided to make them equal the smallest power i
nstead, we would have substituted $n+2=m-2$ in the second sum to get
\begin{equation}
\sum_{n=0}^\infty{n(n-1)a_nt^{n-2}}-\sum_{n=4}^\infty{3a_{n-4}t^{n-2}}=0
\end{equation}
and we would then remove the first four term from the first sum to get
\begin{equation}
2a_2+6a_3t+\sum_{n=4}^\infty{\left[n(n-1)a_nt^{n-2}-3a_{n-4}\right]t^{n-2}}=0
\end{equation}
and so 
\begin{eqnarray}
a_2&=&0\nonumber\\
a_3&=&0\nonumber\\
a_n&=&\frac{3}{n(n-1)}a_{n-4}
\end{eqnarray}
where now the recursion relation applies to $n=4,5,\ldots$ because that
is what is in the sum. Another way of proceeding is to define $a_{-4}=a_{-3}=a_{
-2}=a_{-1}=0$ and then rewrite the equation as
\begin{equation}
\sum_{n=0}^\infty{n(n-1)a_nt^{n-2}}-\sum_{n=0}^\infty{3a_{n-4}t^{n-2}}=0
\end{equation}
and carry on from there.
\end{enumerate}
}
\end{document}



