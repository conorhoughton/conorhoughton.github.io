\documentclass[12pt]{article}
\usepackage{a4wide, amsfonts, epsfig}
\newcommand\soln{\noindent\textit{Solution:} }


%skyline stuff
\font\upright=cmu10 scaled\magstep1
\setlength{\unitlength}{0.012500in}
\begingroup\makeatletter\ifx\SetFigFont\undefined
\def\x#1#2#3#4#5#6#7\relax{\def\x{#1#2#3#4#5#6}}%
\expandafter\x\fmtname xxxxxx\relax \def\y{splain}%
\ifx\x\y   % LaTeX or SliTeX?
\gdef\SetFigFont#1#2#3{%
  \ifnum #1<17\tiny\else \ifnum #1<20\small\else
  \ifnum #1<24\normalsize\else \ifnum #1<29\large\else
  \ifnum #1<34\Large\else \ifnum #1<41\LARGE\else
     \huge\fi\fi\fi\fi\fi\fi
  \csname #3\endcsname}%
\else
\gdef\SetFigFont#1#2#3{\begingroup
  \count@#1\relax \ifnum 25<\count@\count@25\fi
  \def\x{\endgroup\@setsize\SetFigFont{#2pt}}%
  \expandafter\x
    \csname \romannumeral\the\count@ pt\expandafter\endcsname
    \csname @\romannumeral\the\count@ pt\endcsname
  \csname #3\endcsname}%
\fi
\fi\endgroup

\begin{document}
\begin{center}
{\bf 2E2 Tutorial Sheet 17 Second Term}\footnote{Conor
Houghton, {\tt houghton@maths.tcd.ie} and {\tt
http://www.maths.tcd.ie/\char126 houghton/ 2E2.html}}
\\[1cm]
 5 March 2006
\end{center}
{
\noindent{\bf Useful facts:}\vskip .5cm
\begin{itemize}
\item Substitute 
\begin{equation}
y=\sum_{n=0}^{\infty}a_nt^n
\end{equation}
and work out each term in the equation, for example, by differenciating
\begin{equation}
y'=\sum_{n=0}^{\infty}a_nnt^{n-1}
\end{equation}
or by multiplying
\begin{equation}
ty=\sum_{n=0}^{\infty}a_nt^{n+1}
\end{equation}
\item Do a change of index so all the terms have the same power as the term with the highest power, remove terms from the sums to get the same summation range.
\item All coefficients are zero, this gives the recursion relation and, often, some other, conditions.
\end{itemize}
\vskip .25cm
{\bf Note:} If 
\begin{equation}
y=\sum_{n=0}^{\infty}a_nt^n
\end{equation}
then, by setting $t=0$
\begin{equation}
y(0)=a_0.
\end{equation}
Similarily, 
\begin{equation}
y'=\sum_{n=0}^{\infty}na_nt^{n-1}
\end{equation}
and, by setting $t=0$ 
\begin{equation}
y'(0)=a_1.
\end{equation}
On the other hand if no initial condition is given then for a first
order equation $a_0$ is arbitrary and for a second order equation
$a_0$ and $a_1$ are both arbitrary, so when you write out the non-zero
terms, there are $a_0$'s and $a_1$'s appearing. In Q1 the arbitrary
constant $a_0$ has been renamed $A$, this is just to make the solution
look nicer.
\newpage
\noindent{\bf Questions:}
\begin{enumerate}
\item (2) Assuming the solution of 
\begin{equation}
(1-t)y'+y=0
\end{equation}
 has a series expansion about
$t=0$ work out the recursion relation. Write out the first few terms
and show that the series $a_2=0$ so the series actually terminates to give $y=A(1-t)$ for arbitrary $A$.
\vskip .25cm
\item (2) Assuming the solution of
\begin{equation}
(1-t^2)y'-2ty=0
\end{equation}
has a series expansion about $t=0$, work out the recursion relation.
\vskip .25cm
\item (2) Assuming the solution of
\begin{equation}
y''-3y'+2y=0
\end{equation}
has a series expansion about
$t=0$, by substitution, work out the recursion relation. If $y(0)=1$ and $y'(0)=
0$ what are the first three non-zero terms.
\vskip .25cm
\item (2) Assuming the solution of
\begin{equation}
y''-3t^2y=0
\end{equation}
 has a series expansion about $t=0$ work out the recursion relation and write out the first four non-zero terms if $y(0)=1$ and $y'(0)=1$.
\end{enumerate}
}
\end{document}



