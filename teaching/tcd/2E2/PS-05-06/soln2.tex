 \documentclass[12pt]{article}
\usepackage{a4wide, amsfonts, epsfig}


%skyline stuff
\font\upright=cmu10 scaled\magstep1
\setlength{\unitlength}{0.012500in}
\begingroup\makeatletter\ifx\SetFigFont\undefined
\def\x#1#2#3#4#5#6#7\relax{\def\x{#1#2#3#4#5#6}}%
\expandafter\x\fmtname xxxxxx\relax \def\y{splain}%
\ifx\x\y   % LaTeX or SliTeX?
\gdef\SetFigFont#1#2#3{%
  \ifnum #1<17\tiny\else \ifnum #1<20\small\else
  \ifnum #1<24\normalsize\else \ifnum #1<29\large\else
  \ifnum #1<34\Large\else \ifnum #1<41\LARGE\else
     \huge\fi\fi\fi\fi\fi\fi
  \csname #3\endcsname}%
\else
\gdef\SetFigFont#1#2#3{\begingroup
  \count@#1\relax \ifnum 25<\count@\count@25\fi
  \def\x{\endgroup\@setsize\SetFigFont{#2pt}}%
  \expandafter\x
    \csname \romannumeral\the\count@ pt\expandafter\endcsname
    \csname @\romannumeral\the\count@ pt\endcsname
  \csname #3\endcsname}%
\fi
\fi\endgroup

\begin{document}
\begin{center}
\textsf{
{\bf 2E2 Tutorial Sheet 2, Solutions\footnote{Conor Houghton, {\tt houghton@maths.tcd.ie}, see also {\tt http://www.maths.tcd.ie/\char126 houghton/2E2.html}}}\\[1cm] 26 October 2004}
\end{center}


\renewcommand{\labelenumi}{\arabic{enumi}.}

\noindent {\bf Questions}
%the previous year had an extra question on laplace of f' that is left out here
\noindent In the answers here I use the expression {\sl subsidiary equation}, this is a name sometimes given for the Laplace transform of the differential equations, hence, the subsidiary equation is the equation you solve to get F.

\begin{enumerate}
\item (2)
Find the Laplace transform of both sides of the differential equation
\[
 2 \frac{df}{dt} = 1
\]
with initial conditions $f(0) =4$. By solving the resulting equations
find $F(s)$. Based on the Laplace trasforms you know, decide what $f(t)$ is.

\noindent\textit{Solution:} Using linearity of $\mathcal{L}$, plus the
property of Laplace transforms of derivatives, we get
\begin{eqnarray}
\mathcal{L}\left(2 \frac{dx}{dt} \right)
&=& \mathcal{L}(1)\nonumber\\
2\mathcal{L}\left(\frac{df}{dt}\right) &=&\frac{1}{s}\nonumber\\
 2 s F(s)-8
&=& \frac{1}{s}\\
\end{eqnarray}
This means that
\begin{equation}
F(s)=\frac{4}{s}+\frac{1}{2s^2}
\end{equation}
and, since, ${\cal L}(t^n)=n!/s^{n+1}$
\begin{equation}
f=4+\frac{1}{2}t
\end{equation}
To verify that this solves the equation note that $f(0)=4$ as required and $f'=1/2$.




\item (2)
Using the Laplace transform solve the differential equation
\begin{equation}
f''-4f'+3f=1
\end{equation}
with boundary conditions $f(0)=f'(0)=0$. You will need to do partial fractions.

\noindent\textit{Solution:} First, take the Laplace transform of the
equation. Since $f'(0)=f(0)=0$, if ${\cal L}(f)=F(s)$ then ${\cal
L}(f')=sF(s)$ and ${\cal L}(f'')=s^2F(s)$. Thus, the subsidiary
equation is
\begin{equation}
s^2F-4sF+3F=\frac{1}{s}
\end{equation}
and so
\begin{eqnarray}
(s^2-4s+3)F&=&\frac{1}{s}\nonumber\\
F&=&\frac{1}{s}\frac{1}{s^2-4s+3}
\end{eqnarray}
and, since $s^2-4s+3=(s-3)(s-1)$, this gives
\begin{equation}
F=\frac{1}{s(s-3)(s-1)}
\end{equation}
Before we can invert this, we need to do a partial fraction expansion.
\begin{eqnarray}
\frac{1}{s(s-3)(s-1)}&=&\frac{A}{s}+\frac{B}{s-3}+\frac{C}{s-1}\nonumber\\
1&=&A(s-3)(s-1)+Bs(s-1)+Cs(s-3)
\end{eqnarray}
So substituting in $s=0$ we get $A=1/3$, $s=3$ gives $B=1/6$ and $s=1$ gives $C=-1/2$. Hence
\begin{equation}
F=\frac{1}{3s}+\frac{1}{6(s-3)}-\frac{1}{2(s-1)}
\end{equation}
and so
\begin{equation}
f(t)=\frac{1}{3}+\frac{1}{6}e^{3t}-\frac{1}{2}e^t
\end{equation}

\item (2)
Using the Laplace transform solve the differential equation
\begin{equation}
f''-4f'+3f=0
\end{equation}
with boundary conditions $f(0)=1$ and $f'(0)=1$.

\noindent\textit{Solution:}  In this example there are non-zero boundary conditions. Since
\begin{eqnarray}
{\cal L}(f')&=&sF-f(0)\\
{\cal L}(f'')&=&s^2F-sf(0)-f'(0)
\end{eqnarray}
the subsidiary equation in this case is
\begin{equation}
s^2F-s-1-4sF+4+3F=0
\end{equation}
so
\begin{equation}
(s^2-4s+3)F=s-3.
\end{equation}
Hence
\begin{equation}
F=\frac{1}{s-1}
\end{equation}
and
\begin{equation}
f(t)=e^t
\end{equation}



\item (2)
Using the Laplace transform solve the differential equation
\begin{equation}
f''-4f'+3f=0
\end{equation}
with boundary conditions $f(0)=1$ and $f'(0)=0$.

\noindent\textit{Solution:} In this example there are also non-zero
boundary conditions. Since
\begin{eqnarray}
{\cal L}(f')&=&sF-f(0)\\
{\cal L}(f'')&=&s^2F-sf(0)-f'(0)
\end{eqnarray}
the subsidiary equation in this case is
\begin{equation}
s^2F-s-4sF+4+3F=0
\end{equation}
so
\begin{equation}
(s^2-4s+3)F=s-4.
\end{equation}
or
\begin{equation}
F=\frac{s-4}{(s-3)(s-1)}
\end{equation}
We do partial fractions
\begin{equation}
\frac{s-4}{(s-3)(s-1)}=\frac{A}{s-3}+\frac{B}{s-1}
\end{equation}
implying
\begin{equation}
s-4=A(s-1)+B(s-3)
\end{equation}
Choosing $s=3$ give $A=-1/2$ and chosing $s=1$ gives $B=3/2$ so
\begin{equation}
F=-\frac{1}{2}\frac{1}{s-3}+\frac{3}{2}\frac{1}{s-1}
\end{equation}
and
\begin{equation}
f=-\frac{1}{2}e^{3t}+\frac{3}{2}e^t
\end{equation}






\end{enumerate}

\vfill

\noindent

\end{document}
