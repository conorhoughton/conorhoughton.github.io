 \documentclass[12pt]{article}
\usepackage{a4wide, amsfonts, epsfig}
\newcommand{\soln}{\noindent\textit{Solution:}}

%skyline stuff
\font\upright=cmu10 scaled\magstep1
\setlength{\unitlength}{0.012500in}
\begingroup\makeatletter\ifx\SetFigFont\undefined
\def\x#1#2#3#4#5#6#7\relax{\def\x{#1#2#3#4#5#6}}%
\expandafter\x\fmtname xxxxxx\relax \def\y{splain}%
\ifx\x\y   % LaTeX or SliTeX?
\gdef\SetFigFont#1#2#3{%
  \ifnum #1<17\tiny\else \ifnum #1<20\small\else
  \ifnum #1<24\normalsize\else \ifnum #1<29\large\else
  \ifnum #1<34\Large\else \ifnum #1<41\LARGE\else
     \huge\fi\fi\fi\fi\fi\fi
  \csname #3\endcsname}%
\else
\gdef\SetFigFont#1#2#3{\begingroup
  \count@#1\relax \ifnum 25<\count@\count@25\fi
  \def\x{\endgroup\@setsize\SetFigFont{#2pt}}%
  \expandafter\x
    \csname \romannumeral\the\count@ pt\expandafter\endcsname
    \csname @\romannumeral\the\count@ pt\endcsname
  \csname #3\endcsname}%
\fi
\fi\endgroup

\begin{document}
\begin{center}

{\bf 2E2 Tutorial Sheet 5 First Term\footnote{Conor Houghton, {\tt houghton@maths.tcd.ie}, see also {\tt http://www.maths.tcd.ie/\char126 houghton/2E2.html}}\\[1cm] 12 November 2005}
\end{center}


\renewcommand{\labelenumi}{\arabic{enumi}.}
\noindent {\bf Useful facts:}
\begin{itemize}



\item The formula for complex exponentials:
\begin{eqnarray}
e^{i\theta}&=&\cos{\theta}+i\sin{\theta}\cr
e^{-i\theta}&=&\cos{\theta}-i\sin{\theta}
\end{eqnarray}

\item Remember $e^{a+b}=e^a e^b$ so, 
\begin{eqnarray}
e^{a+ib}&=&(\cos{b}+i\sin{b})e^{a}\cr
e^{a-ib}&=&(\cos{b}-i\sin{b})e^{a}
\end{eqnarray}

\item Laplace transform of a periodic function with
period $c$:
\begin{equation}
{\cal L}(f)=\frac{1}{1-e^{-cs}}\int_0^c f(t)e^{-st}dt
\end{equation}
\item Integration by parts:
\begin{equation}
\int_a^b udv=\left. uv\right]_a^b-\int_a^b vdu
\end{equation}
\end{itemize}

\newpage

\noindent {\bf Questions}

\begin{enumerate}

\item (2) Solve, using Laplace transforms, 
\begin{equation}
f''+4f=1
\end{equation}
with $f(0)=f'(0)=0$.

\item (3)
Using the Laplace transform solve the differential equation
\begin{equation}
f''+6f'+13f=e^t
\end{equation}
with boundary conditions $f(0)=0$ and $f'(0)=0$.

\item (3) Use the formula for the Laplace transform of a periodic function
to find the Laplace transform of  a half-rectified wave
\begin{equation}
f(t)=\left\{\begin{array}{ll}\sin{t}&\sin{t}>0\\0&\sin{t}\le 0\end{array}\right.\end{equation}
\begin{center}
\epsfig{file=rectifiedwave.epsi,width=15cm}
\end{center}
This is the form a AC current has after going through a diode and is a periodic function with period $2\pi$. To do the integral, the easiest way is probably to split the sine up using
\begin{equation}
\sin{\theta}=\frac{e^{i\theta}-e^{-i\theta}}{2i}
\end{equation}
and then use the usual formula from the log tables for the integral of an exponetential:
\begin{equation}
\int e^{at}dt=\frac{1}{a}e^{at}
\end{equation}
This works for complex $a$. Try to get a real answer.
\end{enumerate}


\vfill


\end{document}
