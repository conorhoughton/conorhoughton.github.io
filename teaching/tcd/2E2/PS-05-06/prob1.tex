\documentclass[12pt]{article}
\usepackage{a4wide, amsfonts, epsfig}


%skyline stuff
\font\upright=cmu10 scaled\magstep1
\setlength{\unitlength}{0.012500in}
\begingroup\makeatletter\ifx\SetFigFont\undefined
\def\x#1#2#3#4#5#6#7\relax{\def\x{#1#2#3#4#5#6}}%
\expandafter\x\fmtname xxxxxx\relax \def\y{splain}%
\ifx\x\y   % LaTeX or SliTeX?
\gdef\SetFigFont#1#2#3{%
  \ifnum #1<17\tiny\else \ifnum #1<20\small\else
  \ifnum #1<24\normalsize\else \ifnum #1<29\large\else
  \ifnum #1<34\Large\else \ifnum #1<41\LARGE\else
     \huge\fi\fi\fi\fi\fi\fi
  \csname #3\endcsname}%
\else
\gdef\SetFigFont#1#2#3{\begingroup
  \count@#1\relax \ifnum 25<\count@\count@25\fi
  \def\x{\endgroup\@setsize\SetFigFont{#2pt}}%
  \expandafter\x
    \csname \romannumeral\the\count@ pt\expandafter\endcsname
    \csname @\romannumeral\the\count@ pt\endcsname
  \csname #3\endcsname}%
\fi
\fi\endgroup

\begin{document}
\begin{center}
{\bf 2E2 Tutorial Sheet 1\footnote{Conor Houghton, {\tt houghton@maths.tcd.ie}, see also {\tt http://www.maths.tcd.ie/\char126 houghton/2E2.html}}}\\[1cm]{} 16 October 2005
\end{center}

\noindent {\bf Useful formulae:}
\begin{itemize}
\item The Laplace transform of $f(t)$:
\begin{equation}
{\cal L}(f)=\int_0^{\infty}{f(t)e^{-st}dt}
\end{equation}

\item Linearity:
\begin{equation}
{\cal L}(af+bg)=a{\cal L}(f)+b{\cal L}(g)
\end{equation}
where $a$ and $b$ are constants.

\item Integration by parts:
\begin{equation}
\int_a^b udv=\left. uv\right]_a^b- \int_a^b udv
\end{equation}


\item Table of Laplace transforms: ${\cal L}(1)=1/s$, 
\begin{equation}
{\cal L}(t^n)=\frac{n!}{s^{n+1}}
\end{equation}
and
\begin{equation}
{\cal L}\left(e^{at}\right)=\frac{1}{s-a}
\end{equation}


\item The first shift theorem: if ${\cal L}[f(t)]=F(s)$ then
\begin{equation}
{\cal L}\left[f(t)e^{at}\right]=F(s-a)
\end{equation}
\vskip 1cm
\item Laplace transform and differenciation: if ${\cal L}[f(t)]=F(s)$ then
\begin{equation}
{\cal L}(f')=sF-f(0)
\end{equation}

\end{itemize}

\newpage

\noindent {\bf Questions}
\renewcommand{\labelenumi}{\arabic{enumi}.}
\begin{enumerate}

\item (1) Using the linearity of the Laplace transform, calculate the
Laplace transform of
\begin{equation}
f(t)=2-\frac{t}{2}
\end{equation}

\item (1) Using the linearity of the Laplace transform, calculate the
Laplace transform of
\begin{equation}
f(t)=2e^{2t}+3t+4e^{-4t}
\end{equation}


\item (2)
The hyperbolic sine is defined as
\begin{equation}
\sinh{x}=\frac{e^x-e^{-x}}{2}
\end{equation}
using the linearity of the Laplace transform, show that
\begin{equation}
{\cal L}(\sinh{at})=\frac{a}{s^2-a^2}
\end{equation}

\item (2)
Using the shift theorem find the Laplace transform of
\begin{equation}
f(t)=e^{2t}t^2
\end{equation}

\item (2)
Using the formula for the Laplace transform of the differential find ${\cal L}(f')$ where $f=t^2$, check your answer by differentiating $f$ directly and then working out its Laplace transform.


\end{enumerate}

\vfill

\noindent

\end{document}
