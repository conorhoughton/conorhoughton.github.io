 \documentclass[12pt]{article}
\usepackage{a4wide, amsfonts, epsfig}
\newcommand{\soln}{\noindent\textit{Solution:}}

%skyline stuff
\font\upright=cmu10 scaled\magstep1
\setlength{\unitlength}{0.012500in}
\begingroup\makeatletter\ifx\SetFigFont\undefined
\def\x#1#2#3#4#5#6#7\relax{\def\x{#1#2#3#4#5#6}}%
\expandafter\x\fmtname xxxxxx\relax \def\y{splain}%
\ifx\x\y   % LaTeX or SliTeX?
\gdef\SetFigFont#1#2#3{%
  \ifnum #1<17\tiny\else \ifnum #1<20\small\else
  \ifnum #1<24\normalsize\else \ifnum #1<29\large\else
  \ifnum #1<34\Large\else \ifnum #1<41\LARGE\else
     \huge\fi\fi\fi\fi\fi\fi
  \csname #3\endcsname}%
\else
\gdef\SetFigFont#1#2#3{\begingroup
  \count@#1\relax \ifnum 25<\count@\count@25\fi
  \def\x{\endgroup\@setsize\SetFigFont{#2pt}}%
  \expandafter\x
    \csname \romannumeral\the\count@ pt\expandafter\endcsname
    \csname @\romannumeral\the\count@ pt\endcsname
  \csname #3\endcsname}%
\fi
\fi\endgroup

\begin{document}
\begin{center}

{\bf 2E2 Tutorial Sheet 3 Solutions\footnote{Conor Houghton, {\tt houghton@maths.tcd.ie}, see also {\tt http://www.maths.tcd.ie/\char126 houghton/2E2.html}}\\[1cm] 1 November 2005}
\end{center}


\renewcommand{\labelenumi}{\arabic{enumi}.}

\noindent {\bf Questions}
%made 1 easier by setting a=1 and 4 easier by giving the rhs in terms of heaviside.

\begin{enumerate}

\item (2)
Using the Laplace transform solve the differential equation
\begin{equation}
f''-2f'+f=0
\end{equation}
with boundary conditions $f'(0)=1$ and $f(0)=0$.

\noindent\textit{Solution:} Taking the Laplace transform we get
\begin{equation}
s^2F-1-2F+a^2F=0
\end{equation}
and hence
\begin{equation}
F=\frac{1}{(s-1)^2}
\end{equation}
which means that 
\begin{equation}
f=te^{at}
\end{equation}


\item (2)
Using the Laplace transform solve the differential equation
\begin{equation}
f''+f'-6f=e^{-3t}
\end{equation}
with boundary conditions $f(0)=f'(0)=0$.

\noindent\textit{Solution:} So, as before, the subsidiary equation is
\begin{equation}
s^2F+sF-6F=\frac{1}{s+3}
\end{equation}
or 
\begin{equation}
F=\frac{1}{(s+3)^2(s-2)}
\end{equation}
As before, we do partial fractions
\begin{eqnarray}
\frac{1}{(s+3)^2(s-2)}&=&\frac{A}{s+3}+\frac{B}{(s+3)^2}+\frac{C}{s-2}\nonumber\\
1&=&A(s+3)(s-2)+B(s-2)+C(s+3)^2
\end{eqnarray}
$s=-3$ gives $B=-1/5$ and $s=2$ gives $C=1/25$. Putting in $s=1$ we find
\begin{equation}
1=-4A+\frac{1}{5}+\frac{16}{25}
\end{equation}
and so $A=-1/25$. Putting all this together says that
\begin{equation}
f=-\frac{1}{25}e^{-3t}-\frac{t}{5}e^{-3t}+\frac{1}{25}e^{2t}
\end{equation}


\item (2) Use Laplace transform methods to solve the differential equation
\begin{equation}
f'' + 2 f' - 3 f =
\left\{ \begin{array}{ll}
1, & 0 \leq t < c\\
0, & t \geq c
\end{array}\right.
\end{equation}
subject to the initial conditions $f(0) =f'(0) = 0$. Notice that the right hand side is $1-H_1(t)$.
\vskip 1cm

\noindent\textit{Solution:}
Taking Laplace transforms of both sides and using the tables for
the Laplace transform of the right hand side function, leads to
\begin{eqnarray}
(s^2 + 2 s -3)F &=& \frac{1 - e^{-cs}}{s}\nonumber\\
F &=& \frac{1 - e^{-cs}}{s(s^2 + 2 s -3)}\nonumber\\
&=& (1 - e^{-cs})\frac{1}{s(s-1)(s+3)}\nonumber\\
&=& (1 - e^{-cs})\left(\frac{A}{s} + \frac{B}{s-1} +\frac{C}{s+3}
\right)
\end{eqnarray}
Concentrating on the partial fractions part, we have
\begin{eqnarray*}
\frac{1}{s(s-1)(s+3)} &=& \frac{A}{s} + \frac{B}{s-1} +\frac{C}{s+3}\\
1 &=& A(s-1)(s+3) +Bs(s+3) + Cs(s-1)\\
\underline{s=0:}\\
1 &=& -3A\\
A &=& - \frac{1}{3}\\
\underline{s=1:}\\
1 & = & 0 + 4B + 0\\
B &=& \frac{1}{4}\\
\underline{s=-3:}\\
1 &=& 0 + 0 12C\\
C &=& \frac{1}{12}
\end{eqnarray*}
Hence we have
\begin{equation}
F = (1 - e^{-cs}) \left( - \frac{1}{3} \frac{1}{s} +
\frac{1}{4} \frac{1}{s-1} + \frac{1}{12} \frac{1}{s+3} \right)
\end{equation}

\item (2) Use Laplace transform methods to solve the differential equation
\begin{equation}
f'' + 2 f' - 3 f =
\left\{ \begin{array}{ll}
0, & 0 \le t < 1\\
1, & 1\le t < 2\\
0, & t\ge 2
\end{array}\right.
\end{equation}
subject to the initial conditions $f(0) =f'(0) = 0$. You should begin by rewriting the right-hand side in terms of the Heaviside function:
\begin{equation}
H_1(t)-H_2(t)=
\left\{ \begin{array}{ll}
0, & 0 \le t < 1\\
1, & 1\le t < 2\\
0, & t\ge 2
\end{array}\right.
\end{equation}

\soln
So the thing here is to rewrite the right hand side of the equations in terms of Heaviside functions. Remember the definition of the Heaviside function:
\begin{equation}
H_a(t)=\left\{\begin{array}{ll}0&t<a\\1&t\ge a\end{array}\right.
\end{equation}
so the Heaviside function is zero until $a$ and then it is one. The
right hand side is zero until $t=1$ and then it is one until $t=2$ and
then it is zero again. Consider $H_1(t)-H_2(t)$, this is zero until
you reach $t=1$, then the first Heaviside function switches on, the
other one remains zero. Things stay like this until you reach $t=2$, then the second Heaviside function switches on aswell and you get $1-1=0$. Thus
\begin{equation}
H_1(t)-H_2(t)=\left\{ \begin{array}{ll}
0, & 0 \le t < 1\\
1, & 1\le t < 2\\
0, & t\ge 2
\end{array}\right.
\end{equation}

Now, using
\begin{equation}
{\cal L}(H_a(t))=\frac{e^{-as}}{s}
\end{equation}
we take the Laplace transform of the differential equation:
\begin{equation}
s^2F+2sF-3F=\frac{e^{-s}}{s}-\frac{e^{-2s}}{s}
\end{equation}
This gives
\begin{eqnarray}
(s^2+2s-3)F&=&\frac{1}{s}\left(e^{-s}-e^{-2s}\right)\nonumber\\
          F=\frac{1}{s(s-1)(s+3)}\left(e^{-s}-e^{-2s}\right)
\end{eqnarray}
Now, 
\begin{equation}
\frac{1}{s(s-1)(s+3)}=-\frac{1}{3s}+\frac{1}{4(s-1)}+\frac{1}{12(s+3)}
\end{equation}
and we know that 
\begin{equation}
{\cal L}\left(-\frac{1}{3}+\frac{1}{4}e^t+\frac{1}{12}e^{-3t}\right)
=-\frac{1}{3}+\frac{1}{4(s-1)}+\frac{1}{12(s+3)}
\end{equation}
In other word, if it wasn't for the expontentials we'd know the little $f$. However, we know from the third shift thereom that the affect of the exponential $e^{-as}$ is to change $t$ to $t-a$ and to introduce an overall factor of $H_a(t)$. Thus
\begin{equation}
f=H_1(t)\left(-\frac{1}{3}+\frac{1}{4}e^{t-1}+\frac{1}{12}e^{-3t+3}\right)
-H_2(t)\left(-\frac{1}{3}+\frac{1}{4}e^{t-2}+\frac{1}{12}e^{-3t+6}\right)
\end{equation}
\end{enumerate}

\vfill

\noindent

\end{document}
