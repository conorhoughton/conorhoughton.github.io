 \documentclass[12pt]{article}
\usepackage{a4wide, amsfonts, epsfig}


%skyline stuff
\font\upright=cmu10 scaled\magstep1
\setlength{\unitlength}{0.012500in}
\begingroup\makeatletter\ifx\SetFigFont\undefined
\def\x#1#2#3#4#5#6#7\relax{\def\x{#1#2#3#4#5#6}}%
\expandafter\x\fmtname xxxxxx\relax \def\y{splain}%
\ifx\x\y   % LaTeX or SliTeX?
\gdef\SetFigFont#1#2#3{%
  \ifnum #1<17\tiny\else \ifnum #1<20\small\else
  \ifnum #1<24\normalsize\else \ifnum #1<29\large\else
  \ifnum #1<34\Large\else \ifnum #1<41\LARGE\else
     \huge\fi\fi\fi\fi\fi\fi
  \csname #3\endcsname}%
\else
\gdef\SetFigFont#1#2#3{\begingroup
  \count@#1\relax \ifnum 25<\count@\count@25\fi
  \def\x{\endgroup\@setsize\SetFigFont{#2pt}}%
  \expandafter\x
    \csname \romannumeral\the\count@ pt\expandafter\endcsname
    \csname @\romannumeral\the\count@ pt\endcsname
  \csname #3\endcsname}%
\fi
\fi\endgroup

\begin{document}
\begin{center}
\textsf{
{\bf 2E2 Tutorial Sheet 2 First Term\footnote{Conor Houghton, {\tt houghton@maths.tcd.ie}, see also {\tt http://www.maths.tcd.ie/\char126 houghton/2E2.html}}}\\[1cm] 26 October 2004}
\end{center}


\renewcommand{\labelenumi}{\arabic{enumi}.}
\noindent {\bf Useful facts:}
\begin{itemize}
\item Laplace transform of differenciated functions: if ${\cal L}[f(t)]=F(s)$ then
\begin{equation}
{\cal L}(f')=sF-f(0)
\end{equation}
and
\begin{equation}
{\cal L}(f'')=s^2F-sf(0)-f'(0)
\end{equation}                                

\item Partial fractions: assume
\begin{equation}
\frac{a}{(s-b)(s-c)}=\frac{A}{s-b}+\frac{B}{s-c}
\end{equation}
multiply across by $(s-b)(s-c)$
\begin{equation}
a=A(s-c)+B(s-b)
\end{equation}
and choose $s=c$ and $s=b$ to get $A$ and $B$. 

\item Similarily, 
\begin{equation}
\frac{a}{(s-b)(s-c)(s-d)}=\frac{A}{s-b}+\frac{B}{s-c}+\frac{C}{s-d}
\end{equation}
then multiply across by $(s-b)(s-c)(s-d)$ and choose $s$ equal to $b$, $c$ and $d$ to get $A$, $B$ and $C$.

\item Finally, it doesn't matter if there is a polynomial in $s$ above the line:\begin{equation}
\frac{as+e}{(s-b)(s-c)(s-d)}=\frac{A}{s-b}+\frac{B}{s-c}+\frac{C}{s-d}
\end{equation}
then multiply across by $(s-b)(s-c)(s-d)$ and choose $s$ equal to $b$, $c$ and $d$ to get $A$, $B$ and $C$.


\end{itemize}

\newpage

\noindent {\bf Questions}
%the previous year had an extra question on laplace of f' that is left out here

\begin{enumerate}


\item (2)
Find the Laplace transform of both sides of the differential equation
\[
 2 \frac{df}{dt} = 1
\]
with initial conditions $f(0) =4$. By solving the resulting equations
find $F(s)$. Based on the Laplace trasforms you know, decide what $f(t)$ is.


\item (2)
Using the Laplace transform solve the differential equation
\begin{equation}
f''-4f'+3f=1
\end{equation}
with boundary conditions $f(0)=f'(0)=0$. You will need to do partial fractions.

\item (2)
Using the Laplace transform solve the differential equation
\begin{equation}
f''-4f'+3f=0
\end{equation}
with boundary conditions $f(0)=1$ and $f'(0)=1$.

\item (2)
Using the Laplace transform solve the differential equation
\begin{equation}
f''-4f'+3f=0
\end{equation}
with boundary conditions $f(0)=1$ and $f'(0)=0$.

\end{enumerate}

\vfill

\noindent

\end{document}
