\documentclass[12pt]{article}
\usepackage{a4wide, amsfonts, epsfig}
\usepackage{amssymb}
 
\begin{document}
\begin{center}
{\bf Gated channels\footnote{Conor Houghton, {\tt houghton@maths.tcd.ie},
see also {\tt http://www.maths.tcd.ie/\char126 houghton/231}}}\\[1cm]{} 22 October 2008
\end{center}
So the persistient Potassium channel has gating probability $n^4$, the transient sodium channel, $m^3h$. The $n$, $m$ and $h$ satisfy 
\begin{equation}
\frac{d\ell}{dt}=\alpha_{\ell}(V)(1-\ell)-\beta_{\ell}(V)\ell
\end{equation}
which can be rewritten as
\begin{equation}
\tau_{\ell}\frac{d\ell}{dt}=\ell_{\infty}-\ell
\end{equation}
where
\begin{equation}
\tau_{\ell}=\frac{1}{\alpha_{\ell}+\beta_{\ell}}
\end{equation}
and
\begin{equation}
\ell_{\infty}=\frac{\alpha_{\ell}}{\alpha_{\ell}+\beta_{\ell}}
\end{equation}

A standard example set of functional forms for  the alpha and beta are givin in Dayan and Abbott:
\begin{eqnarray}
\alpha_n&=&\frac{.01(V+55)}{1-\exp(-.1(V+55))}\cr
\beta_n&=&.125\exp(-.0125(V+65))
\end{eqnarray}
for $n$, for $m$
\begin{eqnarray}
\alpha_m&=& \frac{.1(V+40)}{1-\exp(-.1(V+40))}\cr
\beta_m&=&4\exp(-.0556(V+65))
\end{eqnarray}
and for $h$
\begin{eqnarray}
\alpha_h&=&.07\exp(-.05(V+65))\cr
\beta_h&=&\frac{1}{1+\exp(-.1(V+35))}
\end{eqnarray}
\end{document}
