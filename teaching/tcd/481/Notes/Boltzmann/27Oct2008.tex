\documentclass[12pt]{article}
\usepackage{a4wide, amsfonts, epsfig}
 
\begin{document}
\begin{center}
{\bf A brief note on the Maxwell-Boltzmann distribution\footnote{Conor Houghton, {\tt houghton@maths.tcd.ie},
see also {\tt http://www.maths.tcd.ie/\char126 houghton/481}}}\\[1cm]{} 27 October 2008
\end{center}
The {\sl Maxwell-Boltzmann distribution} describes the energy
distribution of molecules in a gas of non-interacting molecules; it is
probably easiest to discuss for this example, though, in our case we
will be applying it to ions dissolved in a fluid. The idea is that
molecules in a gas are moving, they have velocity and kinetic
energy. The average velocity determines the temperature: what we
perceive as heat is a measure of the average kinetic energy of
molecules. This does not tell us the distribution of energies, just
the average.

The Maxwell-Boltzmann distribution gives the approximate distribution of the
velocities if interactions are ignored: in this approximation, the
only interaction is hard collision, the molecules don't attract each
other or repel by any long range force, but when two molecules collide
energy can be transfered from one to the other. Obviously this is not
accurate and there are other distributions that are more
accurate for particular situations, however, the Maxwell-Boltzmann
distribution is often accurate enough to be useful.

It is possible to show that the Maxwell-Boltzmann distribution for the energy ${\cal E}$ of a molecule in a gas at temperature $T$ is
\begin{equation}
p({\cal E})=\frac{1}{Z}e^{-{\cal E}/k_bT}
\end{equation}
where $k_b=1.380\times 10^23 JK^{-1}$ is a constant called the Boltzmann constant. This is a probability distribution, what it means is that
\begin{equation}
\mbox{prob}({\cal E}_1<{\cal E}<{\cal E}_2)=\int_{{\cal E}_1}^{{\cal E}_2}{d{\cal E}p({\cal E})}
\end{equation}
$Z$ is a normalization constant, chosen to make the integral of distribution over all possible energies equal to one, so
\begin{equation}
1=\int_0^\infty{d{\cal E}p({\cal E})}=\frac{1}{Z}\int_0^\infty{d{\cal E}e^{-{\cal E}/k_bT}}=\frac{k_bT}{Z}
\end{equation}
and so $Z=k_bT$. In short, through collisions the energy is spread among the
molecules according to this distribution, a process known as {\sl thermalization}.

It is easy to work out the average energy of particles in the gas:
\begin{equation}
<{\cal E}>=\int_0^\infty {d{\cal E}{\cal E}p({\cal E})}=\frac{1}{k_bT}\int_0^\infty {d{\cal E}{\cal E}e^{-{\cal E}/k_bT}}=k_bT
\end{equation}
by integrating by parts. In our application it is also useful to know what fraction of ions have energy greater than some value ${\cal E_0}$ say,
\begin{equation}
\mbox{prob}({\cal E}>{\cal E}_0)=\int_{{\cal E}_0}^\infty{d{\cal E} p({\cal E})}=\frac{1}{k_bT}\int_{{\cal E}_0}^\infty{d{\cal E} e^{-{\cal E}/k_bT}}=e^{-{\cal E}_0/k_bT}
\end{equation}

\end{document}
