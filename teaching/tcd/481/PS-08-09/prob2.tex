\documentclass[12pt]{article}
\usepackage{a4wide, amsfonts, epsfig}

\begin{document}
\begin{center}
{\bf 481 Tutorial Sheet 2\footnote{Conor Houghton, {\tt houghton@maths.tcd.ie}, see also {\tt http://www.maths.tcd.ie/\char126 houghton/481}}}\\[1cm]{} 7 January 2009
\end{center}
\noindent {\bf Questions}


These questions are all pretty straight-forward if you have the patience to read through their lengthy statement.

\begin{enumerate}
\item The spike triggered average is
\begin{equation}
\bar{A}(t)=\left<\sum_{t_i}{s(t_i-t)}\right>
\end{equation}
where $s(t)$ is the stimulus, the $t_i$ are the spike times and the angle brackets denote the average over trials. For convenience let us write the spike triggered average of a single trial as
\begin{equation}
A(t)=\sum_{t_i}{s(t_i-t)}
\end{equation}
so $\bar{A}=\left<A\right>$. If we represent the spike train as
\begin{equation}
\rho(t)=\sum_{t_i}{\delta(t-t_i)}
\end{equation}
then 
\begin{equation}
\bar{A}(t)=\left<\int{dt' s(t'-t)\rho(t)}\right>=\int{dt' s(t'-t)r(t)}
\end{equation}
where the firing rate is 
\begin{equation}
r(t)=\left<\rho(t)\right>
\end{equation}
If there is only a smooth number of trials, so that the rate is smoothed with a kernel
\begin{equation}
r(t)=\left<\int d\tau\rho(t-\tau)k(\tau)\right>
\end{equation}
what is $\bar{A}(t)$ in terms of $k(\tau)$ and $A(t)$? What about
$Q_{\mbox{\scriptsize{rs}}}(\tau)$ which is used in the linear rate model? These comments
are intended to show that kernel smoothing, something that is always
done, might be useful when presenting a rate function, but is not
necessarily useful in applications of the firing rate where some other
integral might provide some smoothing.

\item For the linear rate model the integral equation for the kernel
  was calculated using functional differentiation, that equation was
  then solved by discretizing time and solving the corresponding
  matrix equation. Show that you get the same answer if you discretize
  earlier; that is, discretize the linear model
\begin{equation}
\tilde{r}=r_0+\int{d\tau h(\tau) s(t-\tau)}
\end{equation}
so, for example $H_i=h(i\delta t)$ and then differentiate the error with respect to $H_i$.

\item The convolution theorem for the Fourier transform states that
\begin{equation}
{\cal F}(f*g)=2\pi{\cal F}(f){\cal F}(g)
\end{equation}
where as usual
\begin{equation}
{\cal F}(f)=\frac{1}{2\pi}\int dt f(t)e^{-ikt}
\end{equation}
and the convolution is given by
\begin{equation}
f*g(t)=\int d\tau f(t)g(t-\tau)
\end{equation}
Hence 
\begin{eqnarray}
{\cal F}(f*g)&=&\frac{1}{2\pi}\int dt \int d\tau f(t)g(t-\tau) e^{-ikt}\cr
             &=&\int dt \int d\tau f(\tau)g(t-\tau) e^{-ik(t-\tau)}e^{-ik\tau}\cr
             &=&\int dt \int dt' f(t)g(t') e^{-ikt'}e^{-ikt}\cr
             &=&2\pi \frac{1}{2\pi}\int dt f(t)e^{-ikt}\frac{1}{2\pi}\int dt g(t)e^{-ikt}
\end{eqnarray}
as required. You should note that in proving the convolution theorem
we have assumed all the integrals run over $t\in
(-\infty,\infty)$. Ignoring the finite integration limits use the
convolution theorem to solve the equation for the kernel
\begin{equation}
\int {d\tau' Q_{\mbox{\scriptsize{ss}}}(\tau-\tau')h(\tau')}=Q_{\mbox{\scriptsize{rs}}}(-\tau).
\end{equation}



\end{enumerate}


\end{document}
