\documentclass[12pt]{article}
\usepackage{a4wide, amsfonts, epsfig}
\newcommand\soln{\noindent\textit{Solution:} }

\begin{document}
\begin{center}
{\bf 481 Tutorial Sheet 2a\footnote{Conor Houghton, {\tt houghton@maths.tcd.ie}, see also {\tt http://www.maths.tcd.ie/\char126 houghton/481}}}\\[1cm]{} 12 May 2009
\end{center}

What is the the Fano factor for inhomogeneous Poisson spiking? Well for inhomogenous Poisson spiking
\begin{equation}
P[t_1,t_2,\ldots,t_n]=\prod_{i=1}^nr(t_i)e^{-\int dt r(t)}
\end{equation}
and the Fano factor is $\sigma_n^2/<n>$, in other words, it relates to
the average and variance for the number of spikes. Since it relates to
the number of spike and not to their distribution you might expect
that it won't change between the homogeneous and inhomogeneous case
and that is what happens. To get $P_n$, the probability of $n$ spikes,
we just have to integrate out the possible spike times and divide by the possible orderings:
\begin{equation}
P_n=\frac{1}{n!}\left(\prod_{i=1}^n\int dt_i\right)\left(\prod_{i=1}^nr(t_i)\right)e^{-\int dt r(t)}=\frac{1}{n!}\prod_{i=1}^n\int dt r(t)e^{-\int dt r(t)}
\end{equation}
which is the same as for the homogeneous case, but with the $rT$ replaced by $\int dt r(t)$. Everything now goes through the same, to save some writing let $\rho=\int dtr(t)$, then
\begin{equation}
<n>=\sum_n nP_n= \sum_n n\frac{1}{n!}\rho^n e^{-\rho}
\end{equation}
and, then use 
\begin{equation}
e^\rho=\sum_n \frac{\rho^n}{n!} 
\end{equation} 
so, differenciating each side by $\rho$ gives
\begin{equation}
e^\rho=\sum_n n\frac{\rho^{n-1}}{n!}
\end{equation}
from which we see
\begin{equation}
<n>=\sum_n nP_n= \sum_n n\frac{1}{n!}\rho^n e^{-\rho}=\rho
\end{equation}
A similar calculation, which involves differentiating the expresion for $\exp{\rho}$ twice gives
\begin{equation}
<n^2>=\rho+\rho^2
\end{equation}
and so $\sigma_n^2=<n^2>-<n>^2=\rho$ and we get a Fano factor of one.

\end{document}
