
\documentclass[12pt]{article}
\usepackage{a4wide, amsfonts, epsfig}
\newcommand\soln{\noindent\textit{Solution:} }


\begin{document}
\begin{center}
{\bf 481 sample paper\footnote{Conor Houghton, {\tt houghton@maths.tc
d.ie}, see also {\tt http://www.maths.tcd.ie/\char126 houghton/1S1}}}\\[1cm]{} 27 March 2009
\end{center}

\section*{2 hour exam, do three questions.}

\begin{enumerate}
\item 
\begin{enumerate}
\item (5) Draw a labelled diagram of a neuron indicating the soma,
  axon and dentrites along with the regions with high concentrations
  of sodium and potassium concentration.
\item (4) The Hodgkin-Huxley equation is
\begin{equation}
c_m\frac{dV}{dt}=-i_m
\end{equation}
where $i_m$ is the membrane current
\begin{equation}
i_m=g_L(V-E_L)+g_K(V-E_K)+g_{Na}(V-E_{Na})+\ldots
\end{equation}
The $E_L$, $E_K$ and $E_{Na}$ are {\sl reversal potentials}, in the
case of $E_K$ and $E_{Na}$ explain what is meant by reversal
potential.
\item (5) The potassium channel is a persistent channel with
\begin{equation}
g_{K}\propto n^4
\end{equation}
Draw a Potassium channel and explain where the exponent comes from. Likewise, the sodium channel is a transient channel with
\begin{equation}
g_{Na}=m^3h
\end{equation}
Draw a transient channel and explain the meaning of $m$ and $h$.
\item (6) What equation is satisfied by $n$, $m$ and $h$? What do the terms in this equation mean? Give a rough account of the dynamics that produces spikes.
\end{enumerate}


\item 
\begin{enumerate} 
\item (6) Write down the equation governing a leaky integrate and fire neuron and describe the model. Is it linear?
\item (8) Calculate the firing rate for a leaky integrate and fire neuron with a constant injected current.
\item (6) What is spike rate adaptation? How is it modelled?
\end{enumerate}

\item
\begin{enumerate}
\item (5) With reference to the famous Hubel and Wesel study of vision describe what is meant by a tuning curve.
\item (5) What is the spike triggered average?
\item (10) Show how the spike triggered average can be related to the kernel in a linear filter model of the firing rate.
\end{enumerate}

\item
\begin{enumerate}
\item (5) Draw a labelled diagram of a synapse and give a brief account of what happens when a spike arrives. What is meant by a {\sl post-synaptic potential}?
\item (10) In a useful model of synpatic conductance, the conductance is $g_s=g_0P_s$ where 
\begin{equation}
\tau_s\frac{d}{dt}P_s=-P_s
\end{equation}
where $\beta_s=1/tau_s$ is the closing rate for the synaptic ionic gates and
\begin{equation}
P_s\rightarrow P_s+P_0(1-P_s)
\end{equation}
whenever there is a presynaptic spike. Show how this model can be justified. 
\item (5) What is {\sl Hebb's rule} and how does it differ from \lq{}Neurons that fire together wire together\rq{}. What is {\sl spike timing dependent plasticity}?
\end{enumerate}
\end{enumerate}
\end{document}
