\documentclass[12pt]{article}
\usepackage{a4wide, amsfonts, epsfig}
\newcommand\soln{\noindent\textit{Solution:} }

\begin{document}
\begin{center}
{\bf 481 Tutorial Sheet 2\footnote{Conor Houghton, {\tt houghton@maths.tcd.ie}, see also {\tt http://www.maths.tcd.ie/\char126 houghton/481}}}\\[1cm]{} 7 January 2009
\end{center}
\noindent {\bf Questions}


These questions are all pretty straight-forward if you have the patience to read through their lengthy statement.

\begin{enumerate}
\item The spike triggered average is
\begin{equation}
\bar{A}(t)=\left<\sum_{t_i}{s(t_i-t)}\right>
\end{equation}
where $s(t)$ is the stimulus, the $t_i$ are the spike times and the angle brackets denote the average over trials. For convenience let us write the spike triggered average of a single trial as
\begin{equation}
A(t)=\sum_{t_i}{s(t_i-t)}
\end{equation}
so $\bar{A}=\left<A\right>$. If we represent the spike train as
\begin{equation}
\rho(t)=\sum_{t_i}{\delta(t-t_i)}
\end{equation}
then 
\begin{equation}
\bar{A}(t)=\left<\int{dt' s(t'-t)\rho(t)}\right>=\int{dt' s(t'-t)r(t)}
\end{equation}
where the firing rate is 
\begin{equation}
r(t)=\left<\rho(t)\right>
\end{equation}
If there is only a smooth number of trials, so that the rate is smoothed with a kernel
\begin{equation}
r(t)=\left<\int d\tau \rho(t-\tau)k(\tau)\right>
\end{equation}
what is $\bar{A}(t)$ in terms of $k(\tau)$ and $A(t)$? What about
$Q_{\mbox{\scriptsize{rs}}}(\tau)$ which is used in the linear rate model? These comments
are intended to show that kernel smoothing, something that is always
done, might be useful when presenting a rate function, but is not
necessarily useful in applications of the firing rate where some other
integral might provide some smoothing.

\soln The point being made here is that the spike rate is a smoothing of the spike train and this isn't always needed, since the spike train is smoothed anyway:
\begin{equation}
Q_{\mbox{\scriptsize{rs}}}(\tau)=\int dt r(\tau')s(\tau-\tau')
\end{equation}
and so
\begin{equation}
Q_{\mbox{\scriptsize{rs}}}(\tau)=\left<\int dt \rho(\tau'-t)k(t)  s(\tau-\tau')\right>=\int dt k(t) \int dt \left<\rho(\tau'-t)\right> s(\tau-\tau')
\end{equation}
so if
\begin{equation}
q_{\mbox{\scriptsize{rs}}}(\tau)=\int dt \left<\rho(\tau')\right>s(\tau-\tau')
\end{equation}
then
\begin{equation}
Q_{\mbox{\scriptsize{rs}}}(\tau)=\int dt k(t) q_{\mbox{\scriptsize{rs}}}(\tau-t)
\end{equation}
and so the original smoothing kernel applied to spike trains acts as a
smoothing kernel on the stimulus-response correllation function, something
that might not need smoothing since the stimulus will smooth it anyway.




\item For the linear rate model the integral equation for the kernel
  was calculated using functional differentiation, that equation was
  then solved by discretizing time and solving the corresponding
  matrix equation. Show that you get the same answer if you discretize
  earlier; that is, discretize the linear model
\begin{equation}
\tilde{r}=r_0+\int{d\tau h(\tau) s(t-\tau)}
\end{equation}
so, for example $H_i=h(i\delta t)$ and then differentiate the error with respect to $H_i$.

\soln So this is just a question of following the instruction to discretize the functional integral:
\begin{eqnarray}
H_i&=&h(i\delta t)\cr
R_i\delta t&=&r(i\delta t)\cr
\tilde{R}_i&=&\tilde{r}(i\delta t)\cr
S_{ij}&=&s(i\delta t-j\delta t)
\end{eqnarray}
$S_{ij}$ has a special structure, from its definition
$S_{i-k,j-k}=S_{ij}$ for many values of $i,j$ and $k$. A matrix like
this is called a {\sl Toeplitz}; suprisingly this property of $S_{ij}$
plays no r\^{o}le in what follows. The factor of $1/\delta t$ is added for convenience 

Discretizing the linear model gives
\begin{equation}
\tilde{R}_i=\bar{R}+\sum_k S_{ik}H_k
\end{equation}
where $\bar{R}=r_0\delta t$, the extra factors of $\delta t$ are for
convenience and mean that $R_i$ is the probability of a spike in the
small interval labelled by $i$. The error we need to minimize is
\begin{equation}
\epsilon=\sum_i(R_i-\tilde{R}_i)^2=\sum_i\left(R_iR_i-2R_i\tilde{R}_i+\tilde{R}_i^2\right)
\end{equation}
To minimize with respect to $H_i$ we take the derivative
\begin{equation}
\frac{\partial\epsilon}{\partial H_j}=\sum_i(\tilde{R}_i-R_i)\frac{\partial\tilde{R}_i}{\partial H_j}
\end{equation}
and
\begin{equation}
\frac{\partial\tilde{R}_i}{\partial H_j}=S_{ij}
\end{equation}
This means the minimum is at
\begin{equation}
\sum_i\left(r_0+\sum_k S_{ik}H_k-R_i\right)S_{ij}=0
\end{equation}
and so
\begin{equation}
\sum_k Q^{\mbox{\scriptsize{ss}}}_{jk} H_k=Q^{\mbox{\scriptsize{rs}}}_j
\end{equation}
where the stimulus-stimulus correllation is
\begin{equation}
Q^{\mbox{\scriptsize{ss}}}_{jk}=\sum_i S_{ij}S_{ik}
\end{equation}
and the stimulus-response correllation is
\begin{equation}
Q^{\mbox{\scriptsize{rs}}}_j=\sum_i (R_i-\bar{R})S_{ij}
\end{equation}


\item The convolution theorem for the Fourier transform states that
\begin{equation}
{\cal F}(f*g)=2\pi{\cal F}(f){\cal F}(g)
\end{equation}
where as usual
\begin{equation}
{\cal F}(f)=\frac{1}{2\pi}\int dt f(t)e^{-ikt}
\end{equation}
and the convolution is given by
\begin{equation}
f*g(t)=\int d\tau f(t)g(t-\tau)
\end{equation}
Hence 
\begin{eqnarray}
{\cal F}(f*g)&=&\frac{1}{2\pi}\int dt \int d\tau f(t)g(t-\tau) e^{-ikt}\cr
             &=&\int dt \int d\tau f(\tau)g(t-\tau) e^{-ik(t-\tau)}e^{-ik\tau}\cr
             &=&\int dt \int dt' f(t)g(t') e^{-ikt'}e^{-ikt}\cr
             &=&2\pi \frac{1}{2\pi}\int dt f(t)e^{-ikt}\frac{1}{2\pi}\int dt g(t)e^{-ikt}
\end{eqnarray}
as required. You should note that in proving the convolution theorem
we have assumed all the integrals run over $t\in
(-\infty,\infty)$. Ignoring the finite integration limits use the
convolution theorem to solve the equation for the kernel
\begin{equation}
\int {d\tau' Q_{\mbox{\scriptsize{ss}}}(\tau-\tau')h(\tau')}=Q_{\mbox{\scriptsize{rs}}}(-\tau).
\end{equation}

\soln Well this follows fairly directly if we write the equation for the kernel as a convolution
\begin{equation}
Q_{\mbox{\scriptsize{ss}}}*h(\tau)==Q_{\mbox{\scriptsize{rs}}}(-\tau).
\end{equation}
Now, take the Fourier transform of both sides
\begin{equation}
2\pi{\cal F}(Q{\mbox{\scriptsize{ss}}})(k){\cal F}(h)(k)={\cal F}(Q_{\mbox{\scriptsize{rs}}})(-k)
\end{equation}
Hence
\begin{equation}
h(\tau)=\frac{1}{2\pi}{\cal F}^{-1}\left(\frac{{\cal F}(Q{\mbox{\scriptsize{ss}}})(k)}{{\cal F}(Q_{\mbox{\scriptsize{rs}}})(-k)}\right)
\end{equation}


\end{enumerate}


\end{document}
