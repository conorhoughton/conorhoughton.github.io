\documentclass[12pt]{article}
\usepackage{a4wide, amsfonts, epsfig}

\begin{document}
\begin{center}
{\bf 481 Tutorial Sheet 1\footnote{Conor Houghton, {\tt houghton@maths.tcd.ie}, see also {\tt http://www.maths.tcd.ie/\char126 houghton/481}}}\\[1cm]{} 14 November 2008
\end{center}
\noindent {\bf Questions}

\noindent Exam question on this part of the course will {\sl mostly} consist of derivations and descriptions of the course material. Questions 2 and 3 are also questions that could be asked, though in an exam they would be phrased in a less open-ended way.

\begin{enumerate}
\item The Nernst equation was derived under the assumption that the
  membrane potential was negative and the ion being considered had
  positive charge. Rederive this result for a negatively charged ion
  and for the case when $E$ is positive to verify that it applies in
  all these cases.

\item Consider the effect of a triangular pulse on the integrate and fire neuron. When does this cause a spike?
\begin{equation}
I_e=\left\{\begin{array}{ll}At&t\in(0,T)\\
                            A(2T-t)&t\in(T,2T)\\
                             0&\mbox{otherwise}
\end{array}\right.
\end{equation}

\item Another model of the synaptic conductance has an auxiliary function $z$ and satisfies
\begin{eqnarray}
\tau_s\dot{P}_s&=&eP_mz-P_s\cr
\tau_z\dot{z}&=&-z
\end{eqnarray}
with the rule that $z$ is set to one whenever a spike arrives. $P_m$ is a constant. Solve this for the response to single spike, both with $\tau_s=\tau_z$ and otherwise. In the $\tau_s=\tau_z$ case consider the maximum value of $P_s$ and how this changes if two spikes arrive one after the other. Speculate on the physiological meaning of $z$.




\end{enumerate}


\end{document}
