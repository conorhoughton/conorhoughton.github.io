\documentclass[12pt]{article}
\usepackage{a4wide, amsfonts, epsfig}
\newcommand\soln{\noindent\textit{Solution:} }

\begin{document}
\begin{center}
{\bf 481 Tutorial Sheet 1, Outline solutions\footnote{Conor Houghton, {\tt houghton@maths.tcd.ie}, see also {\tt http://www.maths.tcd.ie/\char126 houghton/481}}}\\[1cm]{} 14 November 2008
\end{center}
\noindent {\bf Questions}

\begin{enumerate}
\item The Nernst equation was derived under the assumption that the
  membrane potential was negative and the ion being considered had
  positive charge. Rederive this result for a negatively charged ion
  and for the case when $E$ is positive to verify that it applies in
  all these cases.

\soln We will begin by revising the original derivation. There are two key formula, first, if a positive ion has charge $zq$, the idea being that $q$ is the charge of a proton and $z$ is a positive integer, then the energy required to move it from potential zero to potential $V$ is $zqV$. Secondly, at a temperature $T$ the probability a particle has energy greater than ${\cal E}$ is
\begin{equation}
\mbox{prob}(\mbox{energy}>{\cal E})=e^{-{\cal E}/k_bT}
\end{equation}
This is calculated by integrating the Boltzmann probability density, as described, for example, in note 4a. Now, in the original senario the cell is at potential $V<0$ and we are considering a positive ion with charge $zq$ where $z>0$. Now the potential gap does not prevent ions diffusing in to the cell, but it does prevent them from diffusing out; only ions with enough energy to cross the gap can diffuse out. Since $V$ is negative the potential gap is $-zqV>0$ and the proportion of cells with energy greater than $-zqV$ is $\exp{zqV/k_bT}=\exp{zV/V_T}$ where we have used the notation $V_T=k_bT/q$. Since the diffusive flow is proportion to the concentration of ions energetically capable of crossing the gap, the flow in is proportional to $\rho_o$, the density outside and flow out is proportional to $\rho_i\exp{zV/V_T}$. Now, the equilibrium potential $V=E$ is the potential for which the two flows balance and hence
\begin{equation}
\rho_o=\rho_ie^{zE/V_T}
\end{equation}
or
\begin{equation}
E=\frac{V_T}{z}\log{\frac{\rho_o}{\rho_i}}
\end{equation}
which is the Nernst equation.

Now, to actually address the question, say instead the ion was negatively charged, $z<0$, so that there is no potential barrier to it flowing out of the cell, but there is one to it flowing in. The energy barrier is $zqV>0$ and hence the flow in is proportional to $\rho_o\exp{(-zV/V_T)}$ and at equilibrium potential
\begin{equation}
\rho_i=\rho_oe^{-zE/V_T}
\end{equation}
or, as before,
\begin{equation}
E=\frac{V_T}{z}\log{\frac{\rho_o}{\rho_i}}
\end{equation}
If $V$ is positive and the ion is positive then the equation is still
\begin{equation}
\rho_i=\rho_oe^{-zE/V_T}
\end{equation}
since the positive ion has a potential barrier of $zqV$ to entering the cell and, if the potential is positive and the ion negative, the equation is
\begin{equation}
\rho_o=\rho_ie^{zE/V_T}
\end{equation}
Hence, the Nernst equation is the same in each case, even though $z$ and $V$ have different signs.


\item Consider the effect of a triangular pulse on the integrate and fire neuron. When does this cause a spike?
\begin{equation}
I_e=\left\{\begin{array}{ll}At&t\in(0,T)\\
                            A(2T-t)&t\in(T,2T)\\
                             0&\mbox{otherwise}
\end{array}\right.
\end{equation}

\soln This question is just a big long annoyance, the challenge is to solve the integrate and fire equation 
\begin{equation}
\tau_m \dot{V}=E_L-V+R_mI_e
\end{equation}
with a different input current. To save on writing lets absorb $R_m$ into $A$ and consider the situation when $t<T$:
\begin{equation}
\tau_m \dot{V}=E_L-V+At
\end{equation}
and hence
\begin{equation}
\dot{V}+\frac{1}{\tau_m}V=\frac{1}{\tau_m}\left(E_L+At\right)
\end{equation}
or
\begin{equation}
\frac{d}{dt}\left(Ve^{t/\tau_m}\right)=\frac{1}{\tau_m}\left(E_L+At\right)e^{t/\tau_m}
\end{equation}
and so, integrating by parts we get
\begin{equation}
V=E_L+A(t-\tau)+Ce^{-t/\tau_m}
\end{equation}
so, assuming the voltage starts at $E_L$. the resting value, we have 
\begin{equation}
E_L=E_L-A\tau+C
\end{equation}
and hence $C=A\tau$ giving
\begin{equation}
V=E_L+At+A\tau\left(e^{-t/\tau_m}-1\right)
\end{equation}
If $V(T)>V_t$, the threshold value, then it spikes before $t=T$; hence we have a spike if
\begin{equation}
V_t<E_L+AT+A\tau e^{-T/\tau_m}-A\tau
\end{equation}
or
\begin{equation}
A>\frac{V_t-E_L}{\exp{(-T/\tau_m)}+T-\tau}
\end{equation}

On the otherhand, even if it didn't spike as the current increased, it could spike as the current falls. For $T<t<2T$ 
\begin{equation}
\dot{V}+\frac{1}{\tau_m}V=\frac{1}{\tau_m}\left(E_L+2AT-At\right)
\end{equation}
so, by analogy with the previous solution, we have
\begin{equation}
V=E_L+2AT-A(t-\tau)+Ce^{-t/\tau_m}
\end{equation}
and setting, $t=T$ and, using the value of $V(T)$ calculated above, we have 
\begin{equation}
C=A\tau\left(1-2e^{T/tau}\right)
\end{equation}
Substituting this back in we have
\begin{equation}
\tau\dot{V}=-A\tau-A\tau e^{-t/\tau}+2A\tau e^{(T-t)/tau}
\end{equation}
so $\dot{V}=0$ if $t=t_m$ where
\begin{equation}
t_m=\tau\log{\left(2e^{T/\tau}-1\right)}
\end{equation}
and there is a spike if $V(t_m)>V_t$ if $t_m<2T$ or if $V(2T)>V_t$
otherwise. As an exam question, this would be a good enough question
if I had left out the down slope, that is, if the current had been
\begin{equation}
I_e=\left\{\begin{array}{ll}At&t\in(0,T)\\
0&\mbox{otherwise}
\end{array}\right.
\end{equation}
but this way it is too tedious to ask in an exam.

\item Another model of the synaptic conductance has an auxiliary function $z$ and satisfies
\begin{eqnarray}
\tau_s\dot{P}_s&=&eP_mz-P_s\cr
\tau_z\dot{z}&=&-z
\end{eqnarray}
with the rule that $z$ is set to one whenever a spike arrives. $P_m$ is a constant. Solve this for the response to single spike, both with $\tau_s=\tau_z$ and otherwise. In the $\tau_s=\tau_z$ case consider the maximum value of $P_s$ and how this changes if two spikes arrive one after the other. Speculate on the physiological meaning of $z$.

\soln This is a much nicer question: first, the biology, it is easy to surmise that $z$ is the level of neurotransmitter in the cleft so it goes instantaneously to one when the spike arrives. $P_s$ relaxes towards the value of $z$ but $z$ is itself falling because of the re-uptake pumps reabsorb the neurotransmitter. This differs from the model we looked at in the lectures where the neurotransmitter had a block wave form, it rose instantaneously to its maximum value, sustained that value and then disappeared.

Now, $z$ is set to one whenever a spike arrives, so if a spike arrives at time $t=0$ 
\begin{equation}
z=e^{-t/\tau_z}
\end{equation}
Substituting back in to the equation for $P_s$ we have
\begin{equation}
\dot{P}_s+frac{1}{\tau_s}P_s=\frac{e}{\tau_s}P_me^{-t/\tau_z}
\end{equation}
Hence
\begin{equation}
\frac{d}{dt}\left(P_se^{t/\tau_s}\right)=\frac{e}{\tau_s}P_me^{t/\tau_s-t/\tau_z}
\end{equation}
and, if $\tau_s\not=\tau_z$ we integrate to get
\begin{equation}
P_s=-\frac{\tau_z}{\tau_s-\tau_z}P_me^{-t\tau_z}+Ce^{-t\tau_s}
\end{equation}
where I have used
\begin{equation}
\left(\frac{1}{\tau_s}-\frac{1}{\tau_z}\right)^{-1}=-\frac{\tau_s\tau_z}{\tau_s-\tau_z}
\end{equation}
and we have switched the terms in the denominator because $\tau_s$ is typically bigger than $\tau_z$. Finally, if this is the first spike in a long time, $P_s(0)=0$ so 
\begin{equation}
C=\frac{\tau_z}{\tau_s-\tau_z}P_m
\end{equation}
and
\begin{equation}
P_s=\frac{\tau_z}{\tau_s-\tau_z}P_m\left(e^{-t\tau_s}-e^{-t\tau_z}\right).
\end{equation}

If $\tau_s=\tau_z$ then the integration is slightly different, let $\tau=\tau_s=\tau-z$ so
\begin{equation}
\frac{d}{dt}\left(P_s e^{t/\tau}\right)=\frac{e}{\tau}P_m
\end{equation}
and
\begin{equation}
P_s=\frac{e}{\tau}P_mt e^{-t/\tau}+Ce^{-t/\tau}
\end{equation}
and, setting $P_s(0)=0$ gives $C=0$ so
\begin{equation}
P_s=\frac{1}{\tau}P_mt e^{1-t/\tau}
\end{equation}
Substituting this back in to the equation for $\dot{P}_s$ gives
\begin{equation}
\tau\dot{P}_s=P_m\left(1-\frac{t}{\tau}\right)e^{1-t/\tau}
\end{equation}
so the maximum occurs at $t=\tau$ and, calculating $P_s(\tau)$ shows that the maximum value is $P_m$. 

If $P_s(0)=P_0>0$ as would happen if one spike followed another, we would have $C=P_0$ and hence
\begin{equation}
P_s=\frac{1}{\tau}P_mt e^{1-t/\tau}+P_0e^{-t/\tau}
\end{equation}
and
\begin{equation}
\tau\dot{P}_s=P_m\left(1-\frac{t}{\tau}-\frac{P_0}{e P_m}\right)e^{1-t/\tau}
\end{equation}
In this case, the maximum occurs when
\begin{equation}
1-\frac{t}{\tau}-\frac{P_0}{P_me}=0
\end{equation}
or $t=(1-f)\tau$ where $f=P_0/P_me$. Substituting back into the equation for $P_s$ we get the amazing result
\begin{equation}
P_s=P_me^f
\end{equation}
so if $P_0=0$ the maximum is $P_m$ as before, if $P_0=P_m$ then the maximum $P_me^{1/e}\approx 1.44 P_m$.

\end{enumerate}


\end{document}
