% LaTeX blank document for exam papers
% Anything after a % on a line is a comment -- does not appear in
% final document
\documentclass[12pt]{article}
\usepackage{tcdexam}
\usepackage{mathrsfs}
\usepackage{graphicx}
\usepackage{amsmath}

\newcommand{\Xx}[1]{\mathscr{#1}}

\begin{document}
\slcode{XMA         }  % insert XID code for the exam, e.g., XMA3211
\course{Module          }  % insert course e.g. Module 321
\examiners{} % insert e.g. Dr. R. Timoney
\groups{
}  % insert e.g. JF Engineers \\ JF MSISS
\term{Trinity Term 2009 }  % insert e.g. Trinity Term 2009
\day{}   % insert e.g. Tuesday, June 2
\time{9.30 --- 12.30}  % insert e.g. 9.30 --- 12.30  or 2.00 --- 5.00
\place{}  % insert e.g. Exam Hall
\instructions{
Attempt three questions.
                    % insert e.g. ATTEMPT SIX QUESTIONS
   \\[12pt]
Log tables are available from the invigilators, if required.\\[12pt]
Non-programmable calculators are permitted for this
examination,---please indicate the make and model of your calculator
on each answer book used.}

\maketitle


\begin{enumerate} %% for questions

\item %1
\begin{enumerate}
\item (5) Derive the typical voltage scale of neuron.
\item (4) The Hodgkin-Huxley equation is
\begin{equation}
c_m\frac{dV}{dt}=-i_m
\end{equation}
where $i_m$ is the membrane current
\begin{equation}
i_m=g_L(V-E_L)+g_K(V-E_K)+g_{Na}(V-E_{Na})+\ldots
\end{equation}
The $E_L$, $E_K$ and $E_{Na}$ are {\sl reversal potentials}, in the
case of $E_K$ and $E_{Na}$ explain what is meant by reversal
potential.
\item (4) The potassium channel is a persistent channel with
\begin{equation}
g_{K}\propto n^4
\end{equation}
Draw a Potassium channel and explain where the exponent comes from. Likewise, the sodium channel is a transient channel with
\begin{equation}
g_{Na}=m^3h
\end{equation}
Draw a transient channel and explain the meaning of $m$ and $h$.
\item (7) What equation is satisfied by $n$, $m$ and $h$? What do the terms in this equation mean? Give a rough account of the dynamics that produces spikes.
\end{enumerate}

\vfill

\item %3
\begin{enumerate}
\item (6) Draw a synapse and label the {\sl terminal button}, {\sl synaptic cleft} and {\sl dendritic spine} along with the {\sl vesicles} and {\sl ligand gated ion channels}. Give a description of the dynamics of a synapse in response to a spike arriving.
\item (7) In a useful model of synaptic conductance, the conductance is $g_s=g_0P_s$ where 
\begin{equation}
\tau_s\frac{d}{dt}P_s=-\beta_sP_s
\end{equation}
where $\beta_s$ is the closing rate for the synaptic ionic gates and
\begin{equation}
P_s\rightarrow P_s+P_0(1-P_s)
\end{equation}
whenever there is a presynaptic spike. Show how this model can be justified. 
\item (7) Another model of the synaptic conductance has an auxiliary function $z$ and satisfies
\begin{eqnarray}
\tau_s\dot{P}_s&=&eP_mz-P_s\cr
\tau_z\dot{z}&=&-z
\end{eqnarray}
with the rule that $z$ is set to one whenever a spike arrives. $P_m$
is a constant. Solve this for the response to single spike with
$\tau_s=\tau_z$ and otherwise. Speculate on the physiological meaning
of $z$.

\vfill

\item %4
\begin{enumerate} 
\item (6) Write down the equation governing a leaky integrate and fire neuron and describe the model. Is it linear?
\item (7) Calculate the firing rate for a leaky integrate and fire neuron with a constant injected current.
\item (7) What is refractoriness? Suggest how it might be modelled?
\end{enumerate}


\item
\begin{enumerate}
\item (5) What is {\sl synaptic plasticity}, {\sl long term depression} and {\sl long term potentiation}. What is {\sl Hebb's rule}.
\item (10) Describe the rate based model of synaptic plasticity and show how it can be approximated by
\begin{equation}
\tau_w\frac{d}{dt}{\bf w}\approx Q{\bf w}
\end{equation}
where ${\bf w}$ is a vector of synapse strengths and $Q$ is a measure of the correlation of the input. 
\item (5) Show how including a threshold in this model gives the approximate dynamics
\begin{equation}
\tau_w\frac{d}{dt}{\bf w}\approx C{\bf w}
\end{equation}
where $C$ is a measure of covariance.

\end{enumerate}
\end{document}
