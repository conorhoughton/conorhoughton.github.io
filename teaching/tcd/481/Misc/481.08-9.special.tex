% LaTeX blank document for exam papers
% Anything after a % on a line is a comment -- does not appear in
% final document
\documentclass[12pt]{article}
\usepackage{tcdexam}
\usepackage{mathrsfs}
\usepackage{graphicx}
\usepackage{amsmath}

\newcommand{\Xx}[1]{\mathscr{#1}}

\begin{document}
\slcode{XMA         }  % insert XID code for the exam, e.g., XMA3211
\course{Module          }  % insert course e.g. Module 321
\examiners{} % insert e.g. Dr. R. Timoney
\groups{
}  % insert e.g. JF Engineers \\ JF MSISS
\term{Trinity Term 2009 }  % insert e.g. Trinity Term 2009
\day{}   % insert e.g. Tuesday, June 2
\time{9.30 --- 12.30}  % insert e.g. 9.30 --- 12.30  or 2.00 --- 5.00
\place{}  % insert e.g. Exam Hall
\instructions{
Attempt three questions.
                    % insert e.g. ATTEMPT SIX QUESTIONS
   \\[12pt]
Log tables are available from the invigilators, if required.\\[12pt]
Non-programmable calculators are permitted for this
examination,---please indicate the make and model of your calculator
on each answer book used.}

\maketitle


\begin{enumerate} %% for questions

\item %1
\begin{enumerate}
\item (5) Draw a labelled diagram of a neuron indicating the {\sl soma},
  {\sl axon} and {\sl dentrites} along with the regions with high concentrations
  of sodium and potassium concentration.
\item (5) Derive the typical voltage scale of neuron.
\item (10) The Nernst equation was derived under the assumption that the
  membrane potential was negative and the ion being considered had
  positive charge. Rederive this result for a negatively charged ion
  and for the case when $E$ is positive to verify that it applies in
  all these cases.
\end{enumerate}

\vfill

\item %3
\begin{enumerate}
\item (6) Draw a synapse and label the {\sl terminal button}, {\sl synaptic cleft} and {\sl dendritic spine} along with the {\sl vesicles} and {\sl ligand gated ion channels}. Give a description of the dynamics of a synapse in response to a spike arriving.
\item (7) In a useful model of synaptic conductance, the conductance is $g_s=g_0P_s$ where 
\begin{equation}
\frac{d}{dt}P_s=-\beta_sP_s
\end{equation}
where $\beta_s$ is the closing rate for the synaptic ionic gates and
\begin{equation}
P_s\rightarrow P_s+P_0(1-P_s)
\end{equation}
whenever there is a presynaptic spike. Show how this model can be justified. 
\item (7) Another model of the synaptic conductance has an auxiliary function $z$ and satisfies
\begin{eqnarray}
\tau_s\dot{P}_s&=&eP_mz-P_s\cr
\tau_z\dot{z}&=&-z
\end{eqnarray}
with the rule that $z$ is set to one whenever a spike arrives. $P_m$
is a constant. Solve this for the response to single spike with
$\tau_s=\tau_z$ and otherwise. Speculate on the physiological meaning
of $z$.

\vfill

\item %4
\begin{enumerate} 
\item (5) Write down the equation governing a leaky integrate and fire neuron and describe the model. 
\item (5) How does the leaky integrate and fire neuron differ from a Hodgkin-Huxley neuron?
\item (10) Consider the effect of a linearly increasing current on the integrate and fire neuron: 
\begin{equation}
I_e=At
\end{equation}
What is the {\sl latency}; that is, the time to the first spike.
\end{enumerate}


\item
\begin{enumerate}
\item (5) What is the spike triggered average?
\item (5) Describe the linear rate response model?
\item (10) What is the relationship between the spike triggered average and the linear rate response model?
\end{enumerate}

\end{enumerate}
\end{document}
