% LaTeX blank document for exam papers
% Anything after a % on a line is a comment -- does not appear in
% final document
\documentclass[12pt]{article}
\begin{document}
Sample paper 2008: two hour exam, do three questions.

\begin{enumerate}
\item % 1st Question
\begin{enumerate}
\item (9 marks) For discrete random variables $X$ and $Y$ define
\begin{itemize}
\item The entropy $H(X)$.
\item The mutual information $I(X;Y)$.
\item The conditional entropy $H(X|Y)$.
\end{itemize}
\item (11 marks) Given the conditional distribution
$$
\begin{tabular}{c|ccc}
&a&b&c\\
\hline\\
1&1/3&1/12&1/12\\
2&1/12&0&1/24\\
3&1/24&1/3&0
\end{tabular}
$$ for $X\in{\cal X}=\{1,2,3\}$ and $Y\in{\cal Y}=\{a,b,c\}$, find
$H(X)$, $H(Y)$, $H(X|Y)$, $H(Y|X)$ and $I(X;Y)$.

\end{enumerate}

\item % 2rd Question
For discrete random variables $X$, $Y$ and $Z$
\begin{enumerate} 
\item (5 marks) prove
$$H(X,Y)=H(X)+H(Y|X)$$
\item (5 marks) prove
$$I(X;Y)=H(Y)-H(Y|X)$$
\item (5 marks) prove
$$H(X,Y|Z)\ge H(X|Z)$$
\item (5 marks) prove
$$I(X;Z|Y)=I(Z;Y|X)-I(Z;Y)+I(X;Z)$$ 
\end{enumerate}


\item % 3nd Question
\begin{enumerate}
\item (4 marks) What is meant by a Markov chain $X\rightarrow
  Y\rightarrow Z$ and 
\item (3 marks) Show that $X\rightarrow Y\rightarrow Z$ implies
  $Z\rightarrow Y\rightarrow X$.
\item (8 marks) State and prove the data processing inequality.
\item (5 marks) Suppose that a Markov chain starts in one of $n$ states, necks down to $k<n$ states and then fans back out to $m>k$ states. Show that the dependence of the first and last variables, $X$ and $Z$ is limited by the bottleneck by showing $I(X,Z)\le\log{k}$.
\end{enumerate}


\item % 4th Question
\begin{enumerate}
\item (4 marks) Define a source code and an instantaneous code.
\item (5 marks) State the Kraft inequality.
\item (4 marks) Define the expected length $L(C)$ of a source code $C(x)$.
\item (8 marks) Prove that the expected length $L$ of any instantaneous $D$-ary code for a random variable $X$ is greater than or equal to the entropy $H_D(X)$.
\end{enumerate}
\end{enumerate}




% end of document should have this next line
\end{document}
