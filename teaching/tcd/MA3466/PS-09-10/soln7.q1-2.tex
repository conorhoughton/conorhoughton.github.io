\documentclass[12pt]{article}
\usepackage{a4wide, amsfonts, epsfig}
\newcommand{\soln}{\noindent\textit{Solution:}}

\begin{document}
\begin{center}
{\bf MA3466 Tutorial Sheet 7, q1-2 solutions\footnote{Conor Houghton, {\tt
      houghton@maths.tcd.ie}, see also {\tt
      http://www.maths.tcd.ie/\char126 houghton/MA3466}}}\\[1cm]{} 6
April 2010
\end{center}
\begin{enumerate}

\item (C\& T 5.2) Let $p(S_i)=p_i$ for some set of outcomes $\{S_1,S_2,\ldots,S_n\}$. The $S_i$'s are uniquely encoded into strings from a $D$-symbol alphabet in a uniquely decodable manner. If $n=6$ and the code word lengths are $(l_1,l_2,\ldots,l_n)=(1,1,2,3,2,3)$ find a good lower bound on $D$.

\soln Well this question is easy to do by checking the first few cases. $D=2$ clearly doesn't work, since $l(S_1)=l(S_2)=1$ then we need to have $c(S_1)=0$ and $c(S_2)=1$ or visa versa, and then the length two code word for $S_3$, which needs to be 00, 01, 10 or 11, is ambiguous, since it could also be decoded as a pair from $S_1$ and $S_2$. However, $D=3$ does work, for example
\begin{center}
\begin{tabular}{c|cccccc}
&$S_1$&$S_2$&$S_3$&$S_4$&$S_5$&$S_6$\\
\hline\\
&0&1&20&220&21&221
\end{tabular}
\end{center}
is a prefix code with the correct lengths.


\item (C\& T 5.4) Slackness in the Kraft inequality. An instantaneous code has word lengths $l_1$ to $l_m$ satisfying the strict inequality
\begin{equation}
\sum_{i=1}^m{D^{-l_i}}<1
\end{equation}
Show there are arbitrarily long sequences of code symbols in ${\cal D}^*$ which cannot be decoded into sequences of codewords: that is, not all sequences of symbols in $D$ form a sentence.

\soln So the trick here is that, if 
\begin{equation}
\sum_{i=1}^m{D^{-l_i}}<1
\end{equation}
then, with $l_x$ with maximum length,
\begin{equation}
\sum_{i=1}^m{D^{l_x-l_i}}<D^{l_x}
\end{equation}
and, we know, that $D^{l_x}$ is the number of nodes at level $l_x$,
which the sum on the right hand side gives the number of nodes at that
level which are codewords or their descendents. Thus, the strict
inequality means that there are nodes at level $l_x$ which are not
codewords and which are not the descendents of codewords. Taking one
these nodes: the node will correspond to a codeword and could of been
added to the existing prefix code to extend it to a further symbol,
this code would still be uniquely decodable. This means that strings
in the codeword corresponding to this node do not correspond to any of
the existing symbols.

\end{enumerate}


\end{document}
