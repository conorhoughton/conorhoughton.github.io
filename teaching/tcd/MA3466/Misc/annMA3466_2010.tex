% LaTeX blank document for exam papers
% Anything after a % on a line is a comment -- does not appear in
% final document
\documentclass[12pt]{article}
\usepackage{tcdexam}
\begin{document}
\slcode{XMA}  % insert XID code for the exam, e.g., XMA3211
\course{Course MA3466}  % insert course e.g. Course 321
\examiners{Conor Houghton} % insert e.g. Dr. R. Timoney
\groups{JS/SS Maths\\JS/SS TP\\MSC in HPC}  % insert e.g. JF Engineers \\ JF MSISS
\term{2008}  % insert e.g. Trinity Term 1987
\day{       }   % insert e.g. Tuesday, June 2
\time{}  % insert e.g. 9.30 --- 12.30  or 2.00 --- 5.00
\place{}  % insert e.g. Exam Hall
\instructions{
 Credit will be given for the best three answers
   \\[12pt]
Log tables are available from the invigilators, if required.\\[12pt]
Non-programmable calculators are permitted for this
examination.}

\maketitle

% begin test of exam use \question{1} to produce a bold face number 1.
% in the left margin.  Use \part{(a)} or \part{(i)} to number parts
% of questions. Put a blank line before these if you want it to come
% on a new line.

\begin{enumerate}
\item % 1st Question
\begin{enumerate}
\item [5 marks] Define
\begin{enumerate}
\item The Shannon Entropy.
\item The conditional entropy.
\item The mutual information.
\end{enumerate}
\item [5 marks] Prove
$$H(X,Y)=H(X)+H(Y|X)$$
and
$$I(X;Y)=H(X)-H(X|Y).$$
\item [4 marks] If ${\cal X}={a,b,c}$ and ${\cal Y}=\{r,s\}$ with
\begin{tabular}{l|ll}
&$a$&$b$\\
\hline
$r$&1/4&1/4\\
$s$&1/2&0
\end{tabular}
calculate $I(X;Y)$. You can write the answer in terms of $\log{3}$ and so on.
\item [6 marks] Let $X_1$ and $X_2$ be identically distributed but not necessarily independent. Let
$$\rho=1-\frac{H(X_2|X_1)}{H(X_1)}.$$
\begin{enumerate}
\item Show that $$\rho=\frac{I(X_1;X_2)}{H(X_1)}.$$
\item Show that $$0\le \rho\le 1.$$
\item When is $\rho=1$?
\item When is $\rho=0$?
\end{enumerate}

\item % 2nd Question
\begin{enumerate}
\item [4 marks] What is meant by a Markov chain $X\rightarrow
  Y\rightarrow Z$.
\item [8 marks] A random variable $X$ takes values $1$ and $0$ with
  probability $p$ and $1-p$ determined by a distribution $P$. Let $X_1$ to $X_k$ be a set of independent variables each of which has the same distribution as $X$. If 
$$N=\sum_{i=1}^k X_i$$
it is clear $P\rightarrow (X_1,\ldots,X_k)\rightarrow N$, argue that
$$P\rightarrow N\rightarrow (X_1,\ldots,X_k)$$ 
and say what is meant by a {\sl sufficient statistic}.
\item [8 marks] Show $H(X_0|X_n)$ is non-decreasing with $n$ for a Markov chain
$$X_0\rightarrow X_1\rightarrow \ldots \rightarrow X_n.$$

\end{enumerate}

\item % 3rd Question
\begin{enumerate} 
\item [8 marks] State and prove Fano's inequality for an estimator $\hat{X}\in\hat{\cal X}$ for a random variable $X\in{\cal X}$.
\item [5 marks] How can the inequality be sharpened when ${\cal X}=\hat{{\cal X}}$.
\item [7 marks] Let $\mbox{Pr}\,(X=i)=p_i$ for $i=1,2,\ldots,m$ with $p_1\ge p_2\ge \ldots \ge\p_m$. In the absence of any other information, the best estimate for $X$ is $\hat{X}=1$ with probability of error $P_e=1-p_1$. Maximize $H({\bf p})$ subject to the constraint $1-p_1=P_e$ to give a bound on $P_e$ in terms of $H$; a version of Fano's inequality with no conditioning.
\end{enumerate}

\item % 4th Question
\begin{enumerate}
\item [5 marks] Define 
\begin{enumerate}
\item A source code.
\item The expected length of a code.
\item A nonsingular code.
\item A uniquely decodable code.
\item A prefix code.
\end{enumerate}
\item [5 marks] Which of the following codes are {\sl uniquely decodable} and {\sl prefix}:
\begin{enumerate}
\item $C_1=\{00,01,0\}$
\item $C_2=\{00,01,100,101,11\}$
\item $C_3=\{0,10,110,1110,\ldots\}$
\item $C_4=\{0,00,000,0000\}$
\item $C_5\{1,00,101\}$.
\end{enumerate}
\item [10 marks] Work out Huffman codes for
\begin{enumerate}
\item $p(A)=.5$, $p(B)=.2$, $p(C)=.1$, $p(D)=.1$ and $p(E)=.1$ with $D=2$.
\item $p(A)=.25$, $p(B)=.15$, $p(C)=.1$, $p(D)=.1$, $p(E)=.1$, $p(F)=.1$, $p(G)=.1$, $p(H)=.05$ and $p(I)=.05$ with $D=3$.
\end{enumerate}
Work out the average code length and the entropy in each case.
\end{enumerate}



% end of document should have this next line
\end{document}
