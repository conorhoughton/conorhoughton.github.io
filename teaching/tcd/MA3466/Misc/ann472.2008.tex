% LaTeX blank document for exam papers
% Anything after a % on a line is a comment -- does not appear in
% final document
\documentclass[12pt]{article}
\usepackage{tcdexam}
\begin{document}
\slcode{XMA}  % insert XID code for the exam, e.g., XMA3211
\course{Course 472}  % insert course e.g. Course 321
\examiners{Conor Houghton} % insert e.g. Dr. R. Timoney
\groups{JS/SS Maths\\JS/SS TP\\MSC in HPC}  % insert e.g. JF Engineers \\ JF MSISS
\term{2008}  % insert e.g. Trinity Term 1987
\day{       }   % insert e.g. Tuesday, June 2
\time{}  % insert e.g. 9.30 --- 12.30  or 2.00 --- 5.00
\place{}  % insert e.g. Exam Hall
\instructions{
 Credit will be given for the best three answers
   \\[12pt]
Log tables are available from the invigilators, if required.\\[12pt]
Non-programmable calculators are permitted for this
examination,---please indicate the make and model of your calculator
on each answer book used.}

\maketitle

% begin test of exam use \question{1} to produce a bold face number 1.
% in the left margin.  Use \part{(a)} or \part{(i)} to number parts
% of questions. Put a blank line before these if you want it to come
% on a new line.

\begin{enumerate}
\item % 1st Question
\begin{enumerate}
\item [5 marks] Define the Kullback-Leibler divergence for two
  probability distributions $p(x)$ and $q(x)$. This quantity is also
  called the relative entropy or Kullback-Leibler distance.
\item [5 marks] State Jensen's inequality.
\item [5 marks] Prove the Kullback-Leibler divergence is non-negative. When is it zero?
\item [5 marks] Show the Kullback-Leibler divergence is not symmetric in $p$ and $q$. Is it ever symmetric when $p\not=q$?
\end{enumerate}

\item % 2nd Question
\begin{enumerate}
\item [4 marks] What is meant by a Markov chain $X\rightarrow
  Y\rightarrow Z$.
\item [4 marks] Show that $X\rightarrow Y\rightarrow Z$ implies
  $Z\rightarrow Y\rightarrow X$.
\item [12 marks] The Data-processing inequality tells us that if
  $X\rightarrow Y\rightarrow Z$ then $I(X;Y)\ge I(X;Z)$. Consider a
  snakes-and-ladders like board game where a fair coin if flipped for
  each go and a counter is moved forwards either zero or one square
  according to whether the coin gives a heads or a tails. $X$, $Y$ and
  $Z$ denote the positions after one, two and three goes. Calculate
  $I(X;Y)$ and $I(X;Z)$ and argue that the result is consistient with
  the data-processing inequality.
\end{enumerate}

\item % 3rd Question
\begin{enumerate} 
\item [8 marks] State and prove Fano's inequality for an estimator $\hat{X}\in\hat{\cal X}$ for a random variable $X\in{\cal X}$.
\item [5 marks] How can the inequality be sharpened when ${\cal X}=\hat{{\cal X}}$.
\item [7 marks] Consider joint variables $(X,Y)$ where ${\cal X}=\{1,2,3\}$ and ${\cal Y}=\{a,b}$ with $p(1,a)=p(2,b)=p(3,b)=1/4$ and all other probabilities equal $1/12$. Find a minimal probability estimator $\hat{X}(Y)$ and evaluate both sides of Fano's inequality for this problem.
\end{enumerate}

\item % 4th Question
\begin{enumerate}
\item [5 marks] Define 
\begin{itemize}
\item A source code.
\item The expected length of a code.
\item A nonsingular code.
\item A uniquely decodable code.
\item A prefix code.
\end{itemize}
\item [8 marks] State and prove the Kraft inequality for a prefix code with codewords of lengths $l_1,l_2,\ldots,l_m$.
\item [7 marks] Find a tenary (D=3) Huffman code for ${\cal X}=\{1,2,3,4,5,6,7\}$ with $p(1)=.25$, $p(2)=p(3)=p(4)=.15$, $p(5)=.12$ and $p(6)=p(7)=.09$. Find a 5-ary (D=5) Huffman code in this case.
\end{enumerate}




% end of document should have this next line
\end{document}
