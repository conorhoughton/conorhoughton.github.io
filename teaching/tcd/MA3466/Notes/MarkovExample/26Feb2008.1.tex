\documentclass[12pt]{article}
\usepackage{a4wide, amsfonts, epsfig}


\begin{document}
\begin{center}{\Large
{\bf The Markov Property}\footnote{Conor Houghton, {\tt houghton@maths.tcd.ie} please
send me any corrections.}}\\[1cm] 26 February 2008\\[1cm]
\end{center}
The random variables $X$, $Y$ and $Z$ are said to {\sl form a Markov chain in that order}
\begin{equation}
X\rightarrow Y\rightarrow Z
\end{equation}
if and only if $p(x|y,z)=p(x|y)$ for all $x$, $y$ and $z$. This is equivalent to requiring $X$, $Z$ are conditional independent on $Y$:
\begin{equation}
p(x,z|y)=p(x|y)p(z|y)
\end{equation}

As an extremely simple example consider snakes and ladders: imagine a
game of snakes and ladders with no snakes and no ladders, for each
turn you flip a coin and move one or two squares depending on whether
you get a head or a tail. This is a Markov chain: the probability
distributions of positions at the $n$th throw depends only on your
current position and not on how you got their. To make this more
definite let $X_1$ be your position after one throw, $X_2$ after two
and $X_3$ after three. Thus
\begin{equation}
p_{X_1}(1)=p_{X_1}(2)=1/2
\end{equation}
Now $X_3$ is not independent of $X_1$. If $X_1=1$ then two heads gives $X_3=3$, a head and a tails or visa versa, gives $X_3=4$ and two tails puts $X_3=5$. 
\begin{eqnarray}
p_{X_3|X_1}(3|1)=p_{X_3|X_1}(5|1)&=&1/4\cr
p_{X_3|X_1}(4|1)&=&1/2
\end{eqnarray}
whereas, if $X_1=2$ you are starting one further along and 
\begin{eqnarray}
p_{X_3|X_1}(4|2)=p_{X_3|X_1}(6|2)&=&1/4\cr
p_{X_3|X_1}(5|2)&=&1/2
\end{eqnarray}
we can add to get
\begin{eqnarray}
p_{X_3}(3)&=&1/8\cr
p_{X_3}(4)&=&3/8\cr
p_{X_3}(5)&=&3/8\cr
p_{X_3}(6)&=&1/8\cr
\end{eqnarray}
but, more importantly, clearly, as stated above and as we would guess, the conditional distributions are different for different values of $X_1$. 

Now, $X_2$ and $X_3$ are also dependent, however, if you know $X_2$
knowing $X_1$ will not tell you any more about $X_3$, all the
dependence of $X_3$ on $X_1$ comes through $X_2$ and $X_1\rightarrow
X_2 \rightarrow X_3$. For completeness, here are the conditional probabilities
\begin{eqnarray}
p_{X_3|X_2}(3|2)=p_{X_3|X_2}(4|2)&=&1/2\cr
p_{X_3|X_2}(4|3)=p_{X_3|X_2}(5|3)&=&1/2\cr
p_{X_3|X_2}(5|4)=p_{X_3|X_2}(6|4)&=&1/2
\end{eqnarray}
The important point is, that if, say $X_2=3$ we know $X_3$ is equally
likely to be four or five. There are also two possible values of
$X_1$, for $X_2$ to be three, we could have had $X_1=1$ or $X_1=2$,
but knowing which it was does not affect the distribution for
$X_3$. In fact $p(x_1,x_3|X_2=3)=1/4$ for each of the possible values of
$x_1$ and $x_3$, just as $p(x_1|X_2=3)=1/2$ and $p(x_3|X_2=3)=1/2$ for each
possible value of $x_1$ and $x_3$.
\end{document}
